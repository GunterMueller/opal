% Local Variables: 
% mode: latex
% TeX-master: "opalreport"
% outline-regexp: "\\\\chap\\|\\\\\\(sub\\)*section"
% End: 

\documentclass[11pt]{report}
% LAST EDIT: Fri Jan 26 15:49:08 1996 by Christian Maeder (andromache!maeder) 
%
% Mar  2 1994 MS: monadic structure of the I/O system
%
% Apr  2 1993 AF: semicompleteness changed in appendix A (minor technical)
% Apr  1 1993 AF: TYPEs are not inherited from sign to impl (major technical)
% Mar 30 1993 AF: some minor technical corrections in appendix A
% Jan 17 1993 WG: article -> report, added acknowledgement
%
% Die folgenden Punkte habe ich gemaess Besprechung vom 10.12.92 geaendert:
%   1. Titelseite
%   2. non-terminals werden kursiv gesetzt (\newcommand{\nt}[1]{...})
%   3. Environments example, examples. Beide ruecken nicht mehr ein.
%   4. Bibliography an das Ende verschoben.
%   5. Anhang "Acknowledgements" (leer) eingefuegt
%   6. WORDs weiche Trennzeichen (-) entfernt
%   7. Headings eingefuehrt (\pagestyle{headings})
%      (Die Eintraege koennen mittels markleft, markright, markboth
%       geaendert werden.)
%   8. \paragraph{} statt \enumerate
%      (All paragraph labels are empty!)
%   9. Die Syntaxregeln von den semantischen Bedingungen getrennt
%      (environment "synrule" vs "concon")
%  10. "\" fuer die Mengendifferenz
%  11. Einrueckung der verbatim-Codesequenzen einheitlich auf 4 Spaces
%  12. Querverweise angepasst
%  13. Der Satz der Syntaxregeln kann jetzt gesteuert werden mittels
%        - \nt{non-terminal},
%        - \tm{terminal} und
%        - \ms{metasymbol}
%      durch Aenderung der entsprechenden "\newcommand" Befehle
%      Continuation lines (Umbruch) in Syntaxregeln werden durch
%        "\contsynrule" eingeleitet.
%  14. Spacing der example(s)-environment geregelt
%
% Die folgenden Punkte habe ich noch ___nicht___ geaendert:
%   1. Inhaltlich nichts geandert, insbesondere keine Aenderung der
%      Libarybeschreibung, der Einleitung, der Acknowledgements, ...

\usepackage{latexsym}
\addtolength{\textheight}{3cm}
\addtolength{\topmargin}{-1.5cm}
\addtolength{\textwidth}{2cm}
\addtolength{\oddsidemargin}{-1cm}
\addtolength{\evensidemargin}{-1cm}

\pagestyle{headings}

% The command "\tm" is used to set terminal symbols in syntax rules
\newcommand{\tm}[1]{``\texttt{#1}''} % '|' may not appear in the argument!
% The command "\nt" is used to enclose non-terminals
\newcommand{\nt}[1]{\mbox{\em #1}}
% The command "\ms" is used to set metasymbols in syntax rules
\newcommand{\ms}[1]{\mbox{#1}}

% The command "\vb" is used to set source text in-line and in syntax rules
\newcommand{\vb}[1]{\texttt{#1}} % '|' may not appear in the argument!
% The command sets a backslash
\newcommand{\bs}{\mbox{$\backslash$}}
% The command "\ul" abbreviates underlining
\newcommand{\ul}[1]{\underline{#1}}
% The command "\dul" underlines twice
\newcommand{\dul}[1]{\underline{\underline{#1}}}

% The environment "example(s)" serves to set examples
\newenvironment{example}{\begin{list}{}{\leftmargin0pt\labelwidth0pt\labelsep0pt}%
\item[{\em Example:}]\ }{\end{list}\nopagebreak\vspace*{-24pt}\hspace*{\fill}$\Box$\vspace*{8pt}}
\newenvironment{examples}{\begin{list}{}{\leftmargin0pt\labelwidth0pt\labelsep0pt}%
\item[{\em Examples:}]\ }{\end{list}\nopagebreak\vspace*{-24pt}\hspace*{\fill}$\Box$\vspace*{8pt}}

% The environment "synrule" serves to set syntax rules
\newenvironment{synrule}{\renewcommand{\baselinestretch}{1.5}%
\begin{list}{}{\leftmargin0pt\labelwidth0pt\labelsep0pt}\item[]\medskip\large}%
{\normalsize\end{list}\renewcommand{\baselinestretch}{1}}
% The command "\contsynrule" is used to indent continuation lines of syntax rules
\newcommand{\contsynrule}{\vspace*{-1ex}\hspace*{4em}}

% The environment "concon" serves to set context conditions
% Items without the optional label are preceded by \conconsymbol
\newcommand{\conconsymbol}{\mbox{\copyright}}
\newenvironment{concon}[1]{\par\begin{list}{\conconsymbol}{\sloppy}#1}{\end{list}\par}

\begin{document}

\title{{\sc The Programming Language Opal}\\
       {\small 5$^{\mbox{\small th}}$ corrected edition}}
\author{The {\sc Opal} Group\\
        Fachbereich Informatik\\
        Technische Universit\"at Berlin\\[1ex]
        {\small Edited by PETER PEPPER}}
\date{November 1997 \bigskip
}

\bigskip


\maketitle

% \renewcommand{\thefootnote}{\fnsymbol{footnote}}
% \footnotetext[2]{The current members of the {\sc Opal} Language Group
%                  are Peter Pepper, Wolfgang Grieskamp, Andreas Fett, 
%                  and Carola Gerke}
\renewcommand{\thefootnote}{\arabic{footnote}}

\tableofcontents


% Befehle f�r Sprachbeschreibung
\newcommand{\cce}{{\ms{::=}}}%                                          ::=
\newcommand{\opt}[1]{\ms{[} #1 \ms{]}}%                                 [ #1 ]
\newcommand{\bra}[1]{\ms{(} #1 \ms{)}}%                                 ( #1 )
\newcommand{\plu}{\ms{$^+$}}%                                           +
\newcommand{\kle}{\ms{$^*$}}%                                              *
\newcommand{\bb}{\ms{$\|$}}%                                               ||
\newcommand{\alt}{\ms{$|$}}%                                               |
\newcommand{\snl}{\\ \contsynrule}%
% The command "\contsynrule" is used to indent continuation lines of syntax rules
% \newcommand{\contsynrule}{\vspace*{-1ex}\hspace*{4em}}
% The environment "concon" serves to set context conditions
\newcommand{\cosi}[1]{\item[\hfill\copyright\hfill]{\underline{#1}:}}
\newcommand{\codo}[1]{\item[\hfill\copyright\hfill]{\underline{\underline{#1}}:}}
\newcommand{\opal}{{\sc Opal}}
\newcommand{\cs}{$\bullet$}
\newcommand{\Da}{$\Downarrow$}

\chapter{Introduction}

The {\sc Opal} project is an experiment that sets out to explore
the optimal compilation of purely 
applicative programming languages. Hence, the language {\sc Opal}
comprises those features of 
applicative languages that we consider essential
for our research on optimal compilation. If 
our intention had been to design a language
for actual software production, some design 
decisions would have turned out differently.
We hope to produce (at some future date) a 
follow-up version of the language, including all
the nice syntactic features that make a 
language truly comfortable to work with. It should not come
as a surprise that {\sc Opal}
resembles other applicative languages such as {\sc Ml}
or --- even more closely --- {\sc Hope}.

In order to understand our main design choices more clearly,
one should bear in mind the
context into which we would like to embed {\sc Opal}.
We envisage a programming environment
in which program development starts
from high-level algebraic specifications. These
specifications are then transformed
and refined in a stepwise process until they are brought 
into a constructive form. This is where {\sc Opal} should take over.

This methodological background suggests
that a language like {\sc Opal} should comprise three
components: a signature part, a specification part,
and an implementation part. The specification aspects,
in particular the compatibility between specification
and implementation, will be 
the subject of a research project that we intend to
conduct (in cooperation with others) in the 
near future. {\em The version of {\sc Opal}, as presented
in this report, deals only with the
constructive aspects of the language, i.e., the signature part
and the implementation part.}

When using more traditional development methods,
the programmer has to produce classical 
code from a specification, employing either transformation
or verification techniques, such 
that the result runs efficiently and meets the specification.
Our hope is that, for a certain class 
of constructive specifications, this task
can be performed automatically by a compiler.

It was these considerations that motivated the following design choices:
\begin{itemize}
  \item The overall appearance of {\sc Opal} programs
        is {\em strongly algebraic}. In particular, we have
        signatures and recursive equations.
  \item {\em Modularization} is an important element in modern specification
        and programming languages. Hence, {\sc Opal} programs are built up from
        structures (each consisting of a signature and
        an implementation part) that are organized in a hierarchical usage
        relation, providing import and export interfaces.
  \item Applicative languages should treat functions as
        ``first-class citizens''. Hence, {\sc Opal}
        provides {\em higher-order functions}, i.e., functions with functions
        as arguments and/or results.
  \item The orientation towards algebraic specification
        languages as well as the interest in
        efficient implementation leads to a {\em strong typing discipline}.
        This does not, however, rule out attractive features
        like parameterization and overloading.
  \item The need to produce truly efficient code does
        --- as far as we are aware --- make eager
        evaluation mandatory. In other words, we employ
        a {\em call-by-value} mechanism.
  \item Input and output are realized by means of
        referentially transparent input/output {\em continuations}.
\end{itemize}

\section*{Overview}

An {\sc Opal} program consists of a collection of {\em structures}.
A structure is the {\sc Opal} counter-part
of what is usually referred to as ``module'',
``cluster'', ``package'', ``encapsulation'', ``class'', etc.
Semantically speaking, structures are algebras.
The structures of an {\sc Opal} program are in an
acyclic hierarchical dependency relation, the ``uses'' or
``import'' relation.

An {\sc Opal} structure consists of a visible
{\em signature part} and a hidden {\em implementation part}.
These are discussed in sections \ref{StructSign} and \ref{StructImpl}.

Structures can be parameterized by sorts and operations.
Parameterized structures are 
introduced in section \ref{parameterization}.

Overloading and parameterization call for flexible
naming of {\sc Opal} objects. These naming
facilities are explained in section \ref{NameScope}.

A description of how to construct an {\sc Opal} program
rounds off this report.

A formal syntax, an availability notice, acknowledgements and a bibliography
are given in the appendices.

\section*{Remarks on the Syntax}

When reading this report, the following syntactic conventions
should be kept in mind.

{\sc Opal} distinguishes four kinds of lexical symbols:
\begin{itemize}
  \item alphanumeric symbols such as: {\tt hallo x3 1}
  \item graphic symbols such as: \vb{++ \% -- ==>}
  \item separators: \nt{blanks} {\tt , ( [ ) ] '}
  \item strings such as: {\tt "Hello World!"}
\end{itemize}

Two alphanumeric, two graphic symbols, or two strings
have to be separated from each other 
by suitable separator symbols; however, the direct
juxtapositioning of different kinds of 
symbols is allowed. Moreover two different symbols can be combined by an
underscore. 
\begin{examples}
The following compositions are legal without separators:
\begin{verbatim}
    x++y       x-1       nat**bool
\end{verbatim}

A composition of different symbols by an underscore which are interpreted as one
lexical symbol looks like : 
\begin{verbatim}
    no_%        -_1     add_one
\end{verbatim}

The following composition requires a separator:
\begin{verbatim}
    <| :text**text->bool
\end{verbatim}
\end{examples}

Note that the {\em question mark} plays a special role:
it may be appended directly to both
alphanumeric and graphic symbols.

{\sc Opal} makes substantial use of keywords.
{\em Keywords} are written in {\tt CAPITAL} letters. They are reserved
words.

There are two ways of writing {\em comments}. The first supports
readability of layouts in print or
on terminals; this kind is started by the symbol \vb{--}
(i.e., a double dash) and it is terminated at
the end of the line. The second is more structure-oriented;
it starts with the symbol {\tt /*} and
terminates with the symbol {\tt */}. This allows, in particular,
``nested comments''. The comment symbols are keywords!

{\sc Opal} includes {\em denotations} which are given as strings,
i.e., character sequences enclosed in double quotes, e.g.,
\vb{"Hello World!"}.

{\sc Opal} {\em identifiers} are arbitrary graphic or alphanumeric character
sequences (except for the reserved words); for example,
{\tt 2}, {\tt +} and {\tt OPAL} are identifiers.

The items in a {\em tuple} (e.g., parameters of a function)
are separated by commas and, if
necessary, enclosed in parantheses. By contrast,
mere {\em collections} of items whose order is
irrelevant are enumerated without commas.
Tuples are used only where necessary, e.g., for
function parameters; collections, on the other hand,
are used wherever possible.

\chapter{Structure Signature}
\label{StructSign}

The major programming units of {\sc Opal} are {\em structures}.
They define data elements as well as
operations.  The {\em interface} of a structure, i.e.,
the components that are visible to the outside 
world, is given by the {\em structure signature} (which plays
a similar role to the ``definition
modules'' in MODULA2). It consists of the following items:
\begin{itemize}
  \item the {\em signature} describes the ``syntactic aspects'',
    i.e., the sorts and the operation (together with their functionalities);
  \item the {\em free types} give a certain structural appearance
    to the data elements.
\end{itemize}

\begin{example}
The following structure \vb{Text} is intended to give an
initial intuitive introduction to 
the {\sc Opal} syntax. It introduces sequences of characters
and operations on them.
\begin{verbatim}
    SIGNATURE Text
    IMPORT  Char  ONLY char                     -- imported
            Nat   ONLY nat                      --  signature
    -- free type
    TYPE text == empty                          -- empty text
                 ::(first: char, rest: text)    -- appending a character
    -- additional operations
    FUN ++     : text ** text -> text           -- concatenation
        revert : text         -> text           -- revert
        length : text         -> nat            -- length
        =      : text ** text -> bool           -- equality
\end{verbatim}
\end{example}

\section{Signature}

The two components of a signature are:
\begin{itemize}
  \item a set S of {\em sorts}, where a sort is a name for a set of values;
  \item a set O of {\em operations}, where an operation is a name for
    a constant or a function. Each operation $o\in$ ) has a
    {\em functionality}.
\end{itemize}

The sorts and operations of a structure come from two sources:
either they are listed explicitly 
in the signature (including free types), or they are inherited
from the imported structures.

The scope of the sorts and operations is the whole structure.

{\sc Opal} is a strongly typed language which does, however, allow
overloading and parameterization. The basic requirement, therefore,
is that each name can be unequivocally identified. The 
detailed conditions that follow from this principle
are listed in section \ref{NameScope}.

\subsection{Sorts}

{\em Sorts} are simply names. Semantically, they denote carrier sets.
These carrier sets may consist 
of elementary values such as the natural or real numbers,
or they may consist of complex data 
structures such as sequences, trees, graphs, or arrays. 

\begin{examples}
The sorts of the booleans, the natural numbers,
and texts are denoted by
\begin{verbatim}
    SORT  bool  nat  text
\end{verbatim}
\end{examples}

\subsection{Operations}

{\em Operations} are names for elementary values or functions.
Elementary values are elements of 
carrier sets; functions are mappings from arguments to results.
Each operation has a functionality, which describes
its sort or its domain and range.

Functions are ``first-class citizens'' in {\sc Opal}.
Hence, functionalities exhibit a deeper structure
than the simple ``argument-tuple-plus-result'' pattern:
we have Cartesian products and
mappings, built on top of sorts.

\begin{examples}
\begin{verbatim}
    FUN 0  1  2  3  4  5  6  7  8  9:nat     -- constants
    FUN pi:real                              -- constant
    FUN divmod:nat**nat->nat**nat            -- function
    FUN filter:(char->bool)->(text->text)    -- higher-order function
\end{verbatim}
\end{examples}

Cartesian products may not be nested, e.g. products such as {\tt
A**(B**C)**D} are not permitted. The mapping operator \vb{->}
associates to the right; i.e., \vb{A->B->C} is equivalent to
\vb{A->(B->C)}. Finally, the empty tuple is only allowed as the domain
of a function, thus characterizing it as a nullary function.

\section{Free Types}
\label{FreeType}

Recursive type definitions are an elegant means of
defining many kinds of data structures. 
Moreover, they support clear function definitions
by means of structural induction (``call-by-pattern'';
see section \ref{operation}). However, the principle of information hiding
requires that the internal structure of data elements should not
transit the interface. {\sc Opal} therefore provides
the concept of {\em free types}, allowing a data type to be viewed as
if defined by a recursive type 
definition, irrespective of the actual implementation
which has to be {\em behaviourally equivalent} (see section \ref{DataType}).

Only one free type may be given for each sort.
Each free type induces its signature (see below).

\begin{example}
The free type \vb{text} is recursively defined
with {\tt empty} and \vb{::} as the free constructor
operations, and \vb{empty?} and \vb{::?} as its discriminator operations.
The selector operations are \vb{first} and \vb{rest}.
\begin{verbatim}
    TYPE text == empty
                 :: (first:char, rest:text)
\end{verbatim}

The type \vb{text} automatically induces the following signature:
\begin{verbatim}
    SORT text                               -- texts
    FUN empty  : text                       -- the empty text
        ::     : char**text->text           -- prefixing a character
        empty? : text->bool                 -- test for emptiness
        ::?    : text->bool                 -- test for nonemptiness
        first  : text->char                 -- the first character
        rest   : text->text                 -- the remaining characters
\end{verbatim}

The operations (have to) obey the following laws:
\begin{eqnarray*}
 \forall e,s & : &  \vb{empty?}(\vb{empty}) \wedge \neg\vb{empty?}(e::s) \\
 \forall e,s & : &      \vb{::?}(e::s) \wedge \neg\vb{::?}(\vb{empty}) \\
 \forall e,s & : &      \vb{first}(e::s) \equiv e \\
 \forall e,s & : &      \vb{rest}(e::s) \equiv s \\
 \forall s   & : &      \vb{::?}(s) \Rightarrow
                                  \vb{first}(s)\vb{::}\vb{rest}(s) \equiv s \\
 \forall s   & : &      \vb{empty?}(s) \vee \vb{::?}(s)
\end{eqnarray*}
\end{example}

Free types are a counterpart of data types; see also section \ref{DataType}.

\section{Export and Import}

\paragraph{}

The {\em export} of a structure is its complete signature part.
Exported sorts and operations of a structure can be {\em imported}
by another structure.

The import becomes part of the signature.
Hence, all sorts and operations that are imported in
the signature part are automatically re-exported.
This has the effect that it is possible, say, to
import from the structure \vb{Text} the sort \vb{nat} from \vb{Nat}.

\paragraph{}

The basic syntactical form of
an import from a structure is the {\em selected import}: only the
explicitly listed sorts and operations are imported.
When an overloaded identifier is listed, all
its (matching) instances are imported (see section \ref{OverloadResolution}).
The alternative form of an import is
the {\em complete import} which is only allowed
in the implementation part of a structure.

\begin{example}
In the structure \vb{Text}, we import from \vb{Nat}
only the sort \vb{nat}.
\begin{verbatim}
    IMPORT Nat ONLY nat
\end{verbatim}
However, if we wish to import everything from \vb{Nat}
in the implementation part, we can write
\begin{verbatim}
    IMPORT Nat COMPLETELY
\end{verbatim}
\end{example}

The same structure may be imported several times.
If the selections are not disjoint, the
common sorts and operations are actually imported only once.

\paragraph{}

Every structure implicitly imports all sorts and
operations from the predefined structures \vb{BOOL} and \vb{DENOTATION}.

\chapter{Structure Implementation}
\label{StructImpl}

Implementations provide the actual definitions
of the various items of a structure.  The two 
components of an implementation are:
\begin{itemize}
  \item a set of {\em carrier sets},
    which are defined by means of data definitions;
  \item a set of {\em values} and {\em functions},
    which are defined by means of (recursive) equations.
\end{itemize}

These carrier sets, values and functions must be
in a one-to-one relationship with the sorts and 
operations listed in the signature of the structure. 

Continuing our example, the implementation part
of the structure \vb{Text} may be defined as follows:
\begin{verbatim}
    IMPLEMENTATION Text
    IMPORT  Char  ONLY =                        -- additionally needed
            Nat   COMPLETELY                    --  signature
    -- data definition
    DATA text == empty
                 ::(first:char, rest:text)
    -- concatenation
    DEF t1++t2 == IF t1 empty? THEN t2          -- simple definition
                  ELSE first(t1)::(rest(t1)++t2) FI
    -- length
    DEF length(empty) == 0                      -- pattern-based
    DEF length(_ ::t) == succ(length(t))        --  definition
    -- revert
    DEF revert(t) == rev(t,empty)
    FUN rev:text**text->text
    DEF rev(empty,z) == z
    DEF rev(c::t,z)  == rev(t,c::z)
    -- equality
    DEF empty=empty == true
    DEF (_ :: _)=empty == false
    DEF empty=(_ :: _) == false
    DEF (c1::t1)=(c2::t2) == IF c1=c2 THEN t1=t2 ELSE false FI
\end{verbatim}

\section{Signature Extension}

The implementation part inherits the signature (including the induced signature
of the free types) of the corresponding signature part, but it can be enriched
by hidden signatures and free types which may also come from additional imports.

\begin{example}
The implementation part of the structure \vb{Text}
has the additional import of \vb{Nat} and
\vb{Char}, and the hidden function \vb{rev}.
\end{example}

\section{Data Types}
\label{DataType}

The types (here: free types and data types) of {\sc Opal}
correspond to the classical mathematical
constructions of direct product and direct sum.
As a result, we obtain the canonical operations 
such as projection functions (in lieu of special language features), etc.
This is of particular benefit in connection with the direct sum,
where we have a very clear and precise conception.

The syntax of type definitions is chosen in such a way
that it comprises all the necessary 
information in a very compact way.
The signature of these operations is automatically derived 
from the type definition.

\subsection{The Definition of Data Types}

A {\em data definition} introduces a {\em sort}
(as its left-hand side) together with a structural description 
of the corresponding data elements (as its right-hand side):
\begin{itemize}
  \item A data definition consists of the direct sum of one or more variants.
  \item Each {\em variant} is either a constant or a direct product of
    one or more components.
  \item Each {\em product} consists of the description of
    a constructor operation and a selector operation for every component.
  \item For each variant, we have a {\em discriminator} operation
    that tests whether a given data element belongs
    to the variant or not. The identifier for this discriminator operation is
    obtained by simply appending a question mark
    to the corresponding constructor.
\end{itemize}

The variant names, the constructor operation as well as
the selector operations and the sorts of 
the components. The signature of these operations
is automatically deduced from the data definition.

\begin{examples}
The data type \vb{text} is recursively defined
as a sum: \vb{empty} and \vb{::} are the
constructor operations, and \vb{empty?} and \vb{::?} its
discriminator operations. The selector operations are \vb{first}
and \vb{rest}.
\begin{verbatim}
    DATA text == empty
                 :: (first:char, rest:text)
\end{verbatim}

The signature of the induced canonical operations
is like the one for the equivalent free type 
(see section \ref{FreeType}).

The following (enumeration) type simply introduces two constants:
\begin{verbatim}
    DATA bool == true  false
\end{verbatim}

A data type \vb{player} is a typical instance of a product type.  Note
that the sort and the constructor operations may have identical
identifiers (but this need not be so). Note also that functions are
admitted as components of products.
\begin{verbatim}
    DATA player == player(name:text,
                          age:nat,
                          statistics:game->points)
\end{verbatim}

The same selector identifier may appear in different variants of a sum. It is then actually one 
selector:
\begin{verbatim}
    DATA address == short(name:text, prename:text, street:text)
                    long( name:text, prename:text, street:text,
                          state:text, tel:nat)
\end{verbatim}
Note that the same constructor name may not be used for short and
long, because the resulting discrimination operations would have the
same name (see section \ref{OverloadResolution}).
\end{examples}

Except for the canonical operations, no additional operations are
automatically defined on types: neither an equality nor an order or
anything else. The main reason for such a rigorous decision is that
{\sc Opal} types can incorporate functions, and there is no computable
equality on functions.

\subsection{Free Types and Data Types}

A data definition looks very similar to a free type. Both concepts are
strongly connected, but there are also essential differences. Both a
data definition and a free type automatically induce the signature,
consisting of the appropriate sort and the canonical constructor,
selector and discriminator operations. However, a data definition also
constitutes a concrete implementation for the data elements and for
the operations. This is not the case for free types. It is therefore
the programmer's task to explicitly provide definitions of the sorts
and the canonical operations of the free types.

Free types must be implemented by behaviourally equivalent data types
(see \cite{Sanella87}). It should not be possible to distinguish
between the free type and the data type by evaluating exported
functions.

\begin{example}
  Consider an alternative data definition of texts, where texts
  are implemented by a sequence of bounded character arrays. For
  details of (parameterized) sequences and arrays, see section
  \ref{parameterization}.
\begin{verbatim}
    DATA text  == text (chunks:seq[chunk])
    DATA chunk == chunk(chars:array[char], from:nat, to:nat)
\end{verbatim}

The operations of the free type now have to be programmed on the basis
of the given data type; we can do this using the above type structure:
\begin{verbatim}
    DEF empty == text(<>)
    DEF c::t == IF chunks(t) <>?      THEN text(singleton:: <>)
                ELSE IF from(fst) = 0 THEN text(singleton::chunks(t))
                                      ELSE text(new::rest(chunks(t)))
                     FI WHERE fst == first(chunks(t))
                              new == chunk((chars(fst),from(fst)-1):=c,
                                                   from(fst)-1,to(fst))
                FI WHERE singleton == chunk(init(10,c),9,9)
    -- etc.
\end{verbatim}

From the user's point of view, only the free type as given in section
\ref{FreeType} is visible. Hence, the free type and the data type are
indistinguishable. But note that, for the above implementation, the
laws of the free type that state equivalences on texts only hold for
behavioural equivalence.

The equivalences
\vspace*{-7pt}
\begin{eqnarray*}
  \forall e,s & : & \vb{rest}(e::s) \equiv s \\
  \forall s   & : & \vb{::?}(s) \Rightarrow \vb{first}(s)\vb{::rest}(s) \equiv s \\
\end{eqnarray*}
\vspace*{-4pt}
actually stand for the behavioural equalities
\begin{eqnarray*}
\forall e,s & : & \vb{first}(\vb{rest}(e\vb{::}s)) \equiv \vb{first}(s)
                  \wedge \vb{length}(\vb{rest}(e\vb{::}s))
                  \equiv \vb{length}(s) \wedge \\
            & & \vb{empty?}(\vb{rest}(e\vb{::}s)) \equiv \vb{empty?}(s)
                \wedge \vb{::?}(\vb{rest}(e\vb{::}s)) \equiv \vb{::?}(s) \\
\forall s   & : & \vb{::?}(s) \Rightarrow \vb{first}(\vb{first}(s)\vb{::rest}(s)) \equiv \vb{first}(s) \wedge \\
                        & & \vb{length}(\vb{first}(s)\vb{::rest}(s)) \equiv \vb{length}(s) \wedge \\
                        & &     \vb{empty?}(\vb{first}(s)\vb{::rest}(s)) \equiv \vb{empty?}(s) \wedge \\
                        & & \vb{::?}(\vb{first}(s)\vb{::rest}(s)) \equiv \vb{::?}(s)
\end{eqnarray*}
\end{example}

\vspace*{-4pt}
Only free types allow function definition by pattern matching. If
there is no free type defined, then the data type automatically
induces a corresponding free type.

\section{Operations}
\label{operation}

In {\sc Opal}, the notion of {\em operations} includes {\em functions}
as well as {\em elementary values} (``constants''). Owing to the
algebraic orientation of the {\sc Opal} programming style, function
definitions are not written as lambda abstractions, but rather in the
form of special equations. For these equations, there are two
notational variants: {\em standard definitions} and {\em pattern-based
definitions}.

\subsection{Standard Definitions}

\paragraph{}

The standard form of a function definition has the following appearance:
\begin{quote}
    \vb{f(x$_1$,\ldots,x$_m$) \ldots\ (z$_1$,\ldots,z$_n$) ==} \nt{expression}
\end{quote}
That is, the left-hand side is the application of a function
identifier to zero or more tuples of {\em parameters}, and the
right-hand side is an arbitrary expression. (Function applications are
described in section \ref{FunctionApplication}.) The functionalities
of the parameters are induced by the functionality of f as listed in
the signature. Empty parameter tuples, denoted by \vb{()}, are
admissible.

\begin{examples}
The following examples implement constants based on suitably defined functions.
\begin{verbatim}
    DEF 1  == succ(0)
    DEF pi == computePi()
\end{verbatim}
The concatenation function on texts, with \vb{++} as its identifier
and \vb{t1}, \vb{t2} as parameters, is defined as follows:
\begin{verbatim}
    DEF t1 ++ t2 == IF empty?(t1) THEN t2
                    ELSE first(t1) :: (rest(t1)++t2) FI
\end{verbatim}
A typical higher-order function is given in the following example: the
function \vb{filter} extracts from a text those elements that fulfil a
certain predicate.
\begin{verbatim}
    FUN filter: (char->bool)->text->text
    DEF filter(p)(t) ==
        IF t empty? THEN empty
        OTHERWISE
        IF p(first(t)) THEN first(t)::filter(p)(rest(t))
                       ELSE           filter(p)(rest(t)) FI
\end{verbatim}
\end{examples}

\paragraph{}

Since higher-order functions are allowed, it may --- under certain
circumstances --- be possible to omit some of the final parameter
tuples.

\begin{example}
    Given the function \vb{filter} (see above) and a predicate
    \vb{isCaps} that tests whether a character is a capital letter, we
    can define a function that extracts all capital letters from a
    given text. In the signature, we have to write
\begin{verbatim}
    FUN Caps: text->text
\end{verbatim}
and the definition can be written equivalently in either of the forms
\begin{verbatim}
    DEF Caps(T) == filter(isCaps)(T)
    DEF Caps    == filter(isCaps)
\end{verbatim}
\end{example}

\paragraph{}

The scope of the parameters is the function definition. Therefore, the
expression that forms this function body can be built up from the
global names of the structure under consideration (i.e.\ import plus
signature) together with the parameters.

\subsection{Pattern-Based Definitions}

Functions are often defined inductively on the structure of the underlying
types. This paradigm is supported in {\sc Opal} by pattern-based definitions,
which, essentially, look like normal definitions, with the exception that the
left-hand side may also contain patterns in addition to parameters. A {\em
  pattern} is a constructor operation from a free type applied to arguments,
which are either {\em pattern variables} or, again, patterns (nested patterns
are allowed).  All of the variables in a pattern must be distinct (because there
is no predefined equivalence on sorts). Their scope is the equation. A special
pattern variable {\em wildcard ( \verb"_" ) \/} is a placeholder for pattern
variables which are unreferenced at the right hand side of the definition.

\begin{example}
The function \vb{filter} can be alternatively defined by patterns:
\begin{verbatim}
    DEF filter(_)(empty) == empty
    DEF filter(p)(x::t)  == IF p(x) THEN x::filter(p)(t)
                                    ELSE    filter(p)(t) FI
\end{verbatim}
Here, \vb{empty} and \vb{x::t} are patterns, and \vb{x} and \vb{t} are
pattern variables. 
\end{example}

The set of equations need not be exhaustive; there may be values that
do not match any of the given patterns. In this case, the function is
simply undefined for such arguments (see also section
\ref{CaseDistinction}).

The case analysis is realized by reference to a best-fit criterion: if
a pattern is an instance of another pattern, the more specific pattern
is checked first; otherwise, even if patterns overlap, the order in
which they are checked is left open (see also section
\ref{CaseDistinction}).

Using the abstract properties of free types, pattern matching is
nothing more than a shorthand notation for an expression with explicit
test and selector applications.

\begin{example}
The equality function (see introductory example to section \ref{StructImpl})
\begin{verbatim}
    DEF empty=empty == true
    DEF (_ :: _)=empty == false
    DEF empty=(_ :: _) == false
    DEF (c1::t1)=(c2::t2) == IF c1=c2 THEN t1=t2 ELSE false FI
\end{verbatim}
is translated into the following code (according to the given
strategy):
\label{TranslatedEqExample}
\begin{verbatim}
    DEF s1=s2 ==
         IF s1 empty? THEN IF s2 empty? THEN true
                           IF s2 ::?    THEN false FI
         IF s1 ::?    THEN IF s2 empty? THEN false
                           IF s2 ::?    THEN
                                LET (c1,t1) == (first(s1),rest(s1))
                                    (c2,t2) == (first(s2),rest(s2))
                                IN IF c1=c2 THEN t1=t2 ELSE false FI
                           FI
         FI
\end{verbatim}
\end{example}

References to the complete argument in the equation's right-hand side
are facilitated by giving it a name in addition to the pattern.

\begin{example}
The function fac may be written as follows:
\begin{verbatim}
    DEF fac(0)            == 1
    DEF fac(m AS succ(n)) == m * fac(n)
\end{verbatim}
\end{example}

{\em Note}: If we define a pattern as a pattern variable or a
constructor operation from a free type applied to patterns, we can
subsume the standard form of function definition as a trivial case of
pattern-based definition. This is done from now on. Note that all
parameters on the left-hand side are then viewed as pattern variables!

\section{Expressions}

{\em Expressions} are used to define the right-hand sides of function
definitions. In general, they are evaluated each time the defined
operation is applied. There is one exception, however: {\em constant
expressions}, i.e., expressions that contain no free variables are
evaluated during program initialization.

There are essentially six different forms of expressions:
\begin{itemize}
  \item {\em atomic expressions}, i.e., applied occurrences of names or denotations,
  \item {\em tuples} of expressions,
  \item {\em function applications},
  \item {\em lambda abstractions},
  \item {\em case distinctions}, i.e., collections of guarded expressions and conditionals, and
  \item {\em extended expressions}, i.e., expressions with 
    auxiliary names introduced in {\tt LET} or {\tt WHERE} clauses.
\end{itemize}

All expressions in {\sc Opal} are strongly typed.

\subsection{Atomic Expressions}

An {\em atomic expression} is either an applied occurrence of a name,
or a denotation. In the former case, the sort of the atom is the sort
of the name; in the latter case, it is the basic sort \vb{denotation}.

\begin{examples}
Applied occurrences of names are
\begin{verbatim}
    pi       1       +       x
\end{verbatim}

Denotations are
\begin{verbatim}
    "Hello World!"      "3.14159"      "2137"      "\t foo \n bar"
\end{verbatim}
\end{examples}

It should be noted that {\sc Opal} does not possess predefined number
systems and the like. For example, the only way to represent numbers
in the familiar decimal notation is to write them as denotations (and
to apply an explicitly programmed conversion function).

\subsection{Tuples}

A {\em tuple} of expressions, enclosed in parantheses and separated by
commas, is itself an expression. An empty tuple () is possible as the
argument of a function application. The functionality of a tuple is
the Cartesian product of the functionalities of its component
expressions.

\begin{example}
The following expression consists of an argument value, a function and the
corresponding result value:
\begin{verbatim}
    (pi, sin, sin(pi))
\end{verbatim}

Hence, its functionality is \vb{real**(real->real)**real}.
\end{example}

{\em Note}: Tuples may be nested, but this is only a syntactic
feature. The functionality of a tuple is always that of the unnested
version.

\subsection{Function Applications and Mixfix Notation}
\label{FunctionApplication}

\paragraph{}

The {\em application} of a function to an argument expression is
itself an expression. The resulting expression may again be a function
and, thus, be applied in turn to an argument.  This way, we obtain
iterated function applications. The arguments of a function
application must always be enclosed in parentheses. Function
application associates to the left; that is \vb{f(x)(y)} is the same as
\vb{(f(x))(y)}.

\paragraph{}

The semantics of the function application is {\em call-by-value}. In
other words, both the function expression and the argument expression
are evaluated before the actual application takes place. The
functionality of a function application is the result functionality of
the function as given in the signature.

\paragraph{}

To improve readability of {\sc Opal} programs, it is possible to write
function applications in infix or postfix notation. The function
symbol may be inserted at any point in the first argument list, i.e.,
if a function \vb{@} has the functionality:
\begin{quote}
    \vb{FUN @ : A ** B ** C ->} \ldots
\end{quote}
then all the following expressions denote the same function application:
\begin{verbatim}
    @(a,b,c)  (a)@(b,c)  (a,b)@(c)  (a,b,c)@
\end{verbatim}
If the argument list to the left or right of an infix or postfix
application has arity one, the parentheses may be omitted. Thus, we
could also write
\begin{verbatim}
    a@(b,c)  (a,b)@c
\end{verbatim}
in the above example. Infix and postfix applications have lower
precedence than the standard prefix application. 

\begin{examples}
A standard situation of a function application is
\begin{verbatim}
    +(pred(x),succ(y))
\end{verbatim}
which can be made more readable by using the familiar infix notation
\begin{verbatim}
    (pred(x))+(succ(y))
\end{verbatim}
and, if we use the binding rules, is even more readable
\begin{verbatim}
    pred(x)+succ(y)
\end{verbatim}
With higher-order functions, we obtain situations like
\begin{verbatim}
    filter(even)(S)
\end{verbatim}
The application to the first argument list can also be written in
postfix notation:
\begin{verbatim}
    (even filter)(S)
\end{verbatim}
\end{examples}

We employ a restricted infix notation to allow more than one infix
application of one operator in sequence. For a given sequence of operators and
operands during context analysis the compiler tries to deduce an unambiguous
structure on operators and operands dependent on the functionality of the
operator. If the deduced type is ambiguous then for the operators, which must be
identitical, right-associativity is assumed (universal right-associative rule). 

\begin{example}
  The functionality of prepending a data with a list and an infix notion of
  concat (here instantiated) would look like: 
  \begin{verbatim}
  :: : nat ** seq[nat] -> seq[nat]
  1 :: 2 :: 3 :: <>
  \end{verbatim}

  The only type correct interpretation is: 
  \begin{verbatim}
  1 :: ( 2 :: ( 3 :: <>))
  \end{verbatim}

  The functionality of concatenating two lists and an infix notion of concat
  (again instantiated) would look like: 
  \begin{verbatim}
  ++ : seq[nat] ** seq[nat] -> seq[nat]
  (1 :: <>) ++ (2 :: <>) ++ (3 :: <>) 
  \end{verbatim}

  Here, two bracketings are possible. But due to the universal
  right-associative rule the following bracketing is deduced: 
  \begin{verbatim}
  (1 :: <>) ++ ((2 :: <>) ++ (3 :: <>))
  \end{verbatim}
\end{example}

\subsection{Lambda Abstractions}
\label{LambdaAbstraction}

The general form of lambda abstraction is
\begin{quote}
  \bs\bs \nt{names} . \nt{expression}
\end{quote}
where \nt{names} is a (possibly empty) list of names for {\em
lambda-bound variables}.  Instead of a \nt{name} a \nt{wildcard} (\verb"\_" ) can
be used to denote a placeholder for unused let-bound variables. 
The body of the abstraction is
\nt{ex\-press\-ion}. The scope of the lambda-bound variables is the
whole lambda abstraction.

The resulting expression is an unnamed function with the Cartesian
product of the functionalities of the lambda-bound variables as its
domain and the functionality of the body as its co-domain.

\begin{example}
    A curried version of the addition function \vb{+} from
    \vb{Nat} can be written in the following way:
\begin{verbatim}
    \\x.\\y.x+y
\end{verbatim}
Its functionality is \vb{nat->nat->nat}.
\end{example}

\subsection{Sections}

Sections are a short-hand notion for delayed evaluation of function
applications. In an application arguments can be replaced by a section
placeholder (\verb"_") to indicate the missing argument value. The behaviour
of such an expression is explained best in giving the syntactical
transformation: a lambda abstraction is built around the function application
and for each placeholder a new lambda-bound variable is generated. 

\begin{example}
  An expression \verb"a + _" is (syntactically) transformed into 
  \verb" \\ new. a + new" where \verb"new" is a new variable.  Multiple
  placeholders are 
  possible, as in \verb"f(_,a,b,_)" which is transformed into 
  \verb" \\ new1,new2. f(new1,a,b,new2)". And an application like
  \verb"f(\_,a)(b,\_)"
  is transformed into \verb" \\ new1. \\ new2. f(new1,a)(b,new2)"
\end{example}

\subsection{Case Distinctions}
\label{CaseDistinction}

The basis for case distinctions is the guarded expression, which is of the form
\begin{quote}
  \vb{IF} \nt{guard} \vb{THEN} \nt{expression}
\end{quote}

The \nt{guard} is a boolean expression. The \nt{expression} is only
evaluated if the \nt{guard} yields true. Its functionality is the
functionality of the whole guarded expression.

Several guarded expressions can be assembled to form a {\em case
distinction} of the form
\begin{quote}
  \vb{IF}  \nt{guard$_1$}  \vb{THEN}  \nt{expression$_1$} \\
          \vdots \\
  \vb{IF}  \nt{guard$_n$}  \vb{THEN}  \nt{expression$_n$}  \vb{FI}
\end{quote}

The operational semantics of the case distinction is defined as
follows: The order in which the various guards are checked is left
open (but fixed at compile time). When the first guard yielding true
is encountered, its corresponding expression is evaluated. If none of
the guards yields true, the case distinction is undefined (i.e., the
program execution aborts).

{\em Note}: Each case distinction is {\em deterministic} in the sense
that, once compiled, it is evaluated every time in the same order
(even though this order is left to the compiler's discretion). For
instance, we have the following property: the equation
\begin{verbatim}
                      f(x) == IF x>=0 THEN g(x)
                              IF x<=0 THEN h(x) FI
\end{verbatim}
is equivalent to the operation
\begin{eqnarray*}
    x>0 & \Rightarrow &  f(x) \equiv g(x) \\
    x<0 & \Rightarrow &  f(x) \equiv h(x) \\
    x=0 & \Rightarrow &  f(x) \equiv g(x) \vee f(x) \equiv h(x)
\end{eqnarray*}
This equivalence characterizes the semantics of ``nondeterminism'' as
realized by the {\sc Opal} compiler.

Sometimes we wish to evaluate certain guards {\em before} evaluating
others.  Then, we write
\begin{quote}
\vb{IF}  \nt{guard$_1$}   \vb{THEN}  \nt{expression$_1$} \\
        \vdots \\
\vb{IF}  \nt{guard$_m$}   \vb{THEN}  \nt{expression$_m$} \\
\vb{OTHERWISE}\\
\vb{IF}  \nt{guard$_{m+1}$} \vb{THEN}  \nt{expression$_{m+1}$} \\
        \vdots \\
\vb{IF}  \nt{guard$_n$}   \vb{THEN}  \nt{expression$_n$}  \vb{FI}
\end{quote}
Here, the guards of the second block are only checked after all guards
in the first block have yielded false. Of course, a case distinction
can contain several sequential blocks separated by \vb{OTHERWISE}.

Finally, we can conclude a case distinction --- with or without
\vb{OTHERWISE} constructs --- by an \vb{ELSE} branch, which is only
evaluated if none of the guards yields true.
\begin{quote}
\vb{IF}  \nt{guard$_1$}   \vb{THEN}  \nt{expression$_1$} \\
        \vdots \\
\vb{IF}  \nt{guard$_n$}   \vb{THEN}  \nt{expression$_n$} \\
               \vb{ELSE}  \nt{expression$_{n+1}$}  \vb{FI}
\end{quote}
In all cases, the keyword \vb{FI} concludes the case distinction, thus
allowing the unambiguous nesting of several case distinctions.

\begin{examples}
The lexicographic order \vb{<=} of texts is based on a case distinction like
\begin{verbatim}
    DEF R<=S ==
          IF R empty?          THEN true
          OTHERWISE
          IF S empty?          THEN false
          OTHERWISE
          IF first(R)<first(S) THEN true
          IF first(R)>first(S) THEN false
                               ELSE rest(R)<=rest(S) FI
\end{verbatim}
Our case distinctions also include the familiar \vb{IF-THEN-ELSE-FI}
construct as a special case:
\begin{verbatim}
    IF x<0 THEN -(x) ELSE x FI
\end{verbatim}
\end{examples}

The {\em guards\/} can be sequentialized in using the keywords \vb{ANDIF} or
\vb{ORIF}. These are only shortcuts for syntactical transformations as shown in
the example. 

\begin{example}

  \begin{quote}
  \vb{IF} \nt{guard$_1$} \vb{ANDIF} \nt{guard$_2$} \vb{THEN} \nt{expression$_1$} 
  \dots
  \end{quote}
  is transformed into 
  \begin{quote}
    \vb{IF} \vb{IF} \nt{guard$_1$} \vb{THEN} \nt{guard$_2$} \vb{ELSE} \nt{false}
    \vb{FI} \vb{THEN} \nt{expression$_1$} \dots
  \end{quote}
  and 
  \begin{quote}
  \vb{IF} \nt{guard$_1$} \vb{ORIF} \nt{guard$_2$} \vb{THEN} \nt{expression$_1$} 
  \dots
  \end{quote}
  is transformed into 
  \begin{quote}
    \vb{IF} \vb{IF} \nt{guard$_1$} \vb{THEN} \nt{true} \vb{ELSE} \nt{guard$_2$}
    \vb{FI} \vb{THEN} \nt{expression$_1$} \dots
  \end{quote}

Sequential guards can be composed without bracketing. \vb{ANDIF} and \vb{ORIF}
bind to the right. 

\end{example}
     

\subsection{Extended Expressions}
\label{ExtendedExpression}

\paragraph{}

In order to allow a further structuring of expressions, it is possible
to name the values of subexpressions and to use these names in other
expressions. (This is particularly useful for ``common
subexpressions''.) In this way, we obtain {\em extended expressions}
that comprise \vb{LET} and \vb{WHERE} clauses. The general notation
is:
\begin{quote}
  \nt{expression}  \vb{WHERE}  \nt{declarations} \\
  \vb{LET}  \nt{declarations}  \vb{IN}  \nt{expression}
\end{quote}

\paragraph{}

Here, the \nt{declarations} are sets of equations, the left-hand sides
of which are (tuples of) names of {\em let-bound variables}. Thus, we
have either of the two forms for a declaration:
\begin{quote}
  \nt{name} == \nt{expression} \\
  (\nt{name$_1$},\ldots,\nt{name$_n$}) == \nt{expression}
\end{quote}
Instead of \nt{name} a \nt{wildcard} can be used as a placeholder to denote
unused values. The order in which these equations are written down is of no
relevance, but cyclic dependencies are not allowed; in other words, it
must be possible to put the declarations in a sequential order. (This
is the compiler's job.)

\paragraph{}

The scope of let-bound variables is the largest expression with which
the \vb{LET} or \vb{WHERE} clause can be combined in a syntactically
meaningful way. For instance, in \vb{f(y)(y) WHERE} \vb{y==\ldots}, the
scope does not only extend to the final \vb{(y)}, but (at least) to
the whole term \vb{f(y)(y)}. By contrast, in \vb{f(y, y WHERE
y==\ldots}), the scope only extends to the latter \vb{y}. In addition,
the scope of a let-bound variable extends over the whole list of
declarations to which it belongs. The declared names inherit their
functionalities from their corresponding right-hand sides.

The semantics of \vb{LET} and \vb{WHERE} clauses is {\em strict},
i.e., the equations are evaluated before the associated expression is
evaluated.
\begin{examples}
    In the translated equality test in section
    \ref{TranslatedEqExample}, we have a typical use of \vb{LET}
    clauses.
    
    Local declarations are often used to name the individual
    components of a function with several results:
\begin{verbatim}
    WHERE (quotient,rest) == x divmod y
\end{verbatim}
According to the scoping rules, the same name may, for example, appear
in both branches of a conditional:
\begin{verbatim}
    IF ... THEN LET xnew == f(xold) IN h(xnew,xold)
           ELSE LET xnew == g(xold) IN k(xnew,xold) FI
\end{verbatim}
\end{examples}

\chapter{Parameterization}
\label{parameterization}

Sequences of natural numbers are not much different from sequences of
characters (= words) or even sequences of sequences of characters (=
sentences). This fact is expressed --- as it usually is in mathematics
or informatics --- by a proper parameterization.

\section{Parameterized Structures}
\label{ParameterizedStructures}

Below, we define the structure of sequences via a suitably chosen
parameter, which in this case is a sort. However, we need not restrict
this parameterization facility to sorts only. For instance, bounded
sequences have an appropriate constant as their parameter, and ordered
sequences are based on a suitable order relation on the element sort. 
The parameters are declared in the signature and can be used in the
body of the structure. Thus, the scope within which the parameters are
visible is the whole structure.

\begin{examples}
    The structure \vb{Seq} represents a classical example of
    parameterization. Here, \vb{data} is the parameter, and it is
    declared to be a sort. Hence, it can be used in the structure body
    like any other sort.
\begin{verbatim}
    SIGNATURE Seq[data]
    -- Parameters
    SORT data
    -- Imports
    IMPORT  Nat  ONLY nat
    -- free type
    TYPE seq == <>
                ::(ft:data, rt:seq)
    -- additional operations
    FUN ++ : seq**seq->seq                 -- concatenation
    FUN #  : seq->nat                      -- length
    -- etc.

    IMPLEMENTATION Seq[data]
    -- additional import
    IMPORT  Nat  COMPLETELY
    -- data definition
    DATA seq == <>
                ::(ft:data, rt:seq)
    -- concatenation
    DEF <> ++ t2      == t2
    DEF (c::t1) ++ t2 == c::(t1++t2)
    -- length
    DEF #(<>)   == 0
    DEF #(_ :: t) == succ(#(t))
    -- etc.
\end{verbatim}

A structure for bounded sequences could be given in the form
\begin{quote}
    \vb{SIGNATURE BoundedSeq [data,bound]} \\
    \vb{SORT data} \\
    \vb{FUN bound:nat} \\
    \vb{-- rest of signature part}
\end{quote}

Again, \vb{data} can be used in the body like any normal sort, and
\vb{bound} can be used like any other constant. Note, however, that
the sort \vb{nat} must be known in the structure by means of an
import, or it must be a parameter! (Using here a sort that is declared
in the structure itself would clearly lead to a vicious circle.)

A structure for ordered sequences could be given in the form
\begin{quote}
  \vb{SIGNATURE OrderedSeq [data,<]} \\
  \vb{SORT data} \\
  \vb{FUN < :data**data->bool} \\
  \vb{ -- rest of signature part}
\end{quote}

As before, \vb{data} can be used like a normal sort in the body, and
\vb{<} can be used like a normal operation. Note that a parameter
which is a sort may be used to express the functionalities of other
parameters.
\end{examples}

The rules for signatures, implementations and free types also apply to
parameterized structures.

\section{Instantiation}

If a structure is parameterized, then all its constituents become {\em
generic}. Every sort and operation now stands for a whole family of
carrier sets, values and functions, respectively.

If we import a parameterized structure, we have to provide sorts and
operations for the parameters. In case of overloaded imports, e.g.,
\vb{Seq[nat]} and \vb{Seq[real]}, we have to use additional facilities
to make the names unequivocal (see section \ref{NameScope}).

\begin{example}
We may now define the sort \vb{text} from \vb{Text} using the
structure \vb{Seq}.
\begin{quote}
    \vb{IMPORT  Seq[char]  COMPLETELY} \\
            \vdots \\
    \vb{DATA text == text(text:seq, len:nat)}
\end{quote}
\end{example}

In the implementation part of a structure, it is also possible to
import a parameterized structure {\em uninstantiated}. This can be
viewed as an import of the structure with all necessary
instantiations.

\chapter{Names and Scopes}
\label{NameScope}

In an {\sc Opal} program, there are many objects such as sorts,
operations and parameters that must be referred to in the program
text. Hence, all these objects have {\em names}. The use of each name
is restricted to certain parts of the program text, called the scope
of the name.

The naming facilities in {\sc Opal} are very flexible. This is
practically unavoidable in connection with parameterization, but it is
also convenient in other cases where overloading of identifiers is
desirable. The classical example here is a function like addition on
integers and real numbers.

\section{Names}

Names in {\sc Opal} do not only consist of an identifier, but also of
an {\em origin} and a {\em kind}. For every name application, its
origin and/or kind can be given in addition to the identifier in order
to describe the name exactly. This is optional if the missing
information can be deduced from the context.

\subsection{Origin}

The origin of an object is the identifier of the structure in which it
is declared.

\begin{example}
    Suppose that there are two structures \vb{DirGraph} and
    \vb{UndirGraph} for directed and undirected graphs, respectively. 
    Both sorts are called {\em graph}. To write a structure that uses
    both kinds of graphs, we have to import both structures:
\begin{verbatim}
    IMPORT  DirGraph    ONLY graph
    IMPORT  UndirGraph  ONLY graph
\end{verbatim}
But we now encounter the problem of how to refer unequivocally to the
sort of directed graphs. Hence, we may write \vb{graph'DirGraph} (to
be read as: ``graph from DirGraph'') to refer to the sort \vb{graph}
of the structure \vb{DirGraph}.
\end{example}

Note that, in the case of transitive import, the origin is the
structure in which the object is originally declared.

\begin{example}
The structure \vb{Text} re-exports the sort \vb{nat'Nat}. Thus, we can write
\begin{verbatim}
    IMPORT Text ONLY nat
\end{verbatim}
But the origin of nat remains the structure \vb{Nat}.
\end{example}

\subsection{Kind}

Since overloading is also allowed in the internal signature of a
structure, the origin is not always sufficient to distinguish two
objects. The {\em kind} of an object consists of its {\em class}
(i.e., whether it is a sort or an operation) and --- in case of an
operation --- its {\em functionality}.

\begin{example}
    Suppose that, in a structure \vb{Rat} for rational numbers, the
    sort as well as two conversion functions are all called \vb{rat};
    their identifiers are therefore overloaded:
\begin{verbatim}
    SIGNATURE Rat
    IMPORT  Nat  ONLY nat
    IMPORT  Int  ONLY int
    SORT rat
    FUN rat:nat->rat
    FUN rat:int->rat
\end{verbatim}
To use one of these identifiers unambiguously, it is generally
necessary to explicitly denote its kind. Thus, one has to write
\vb{rat:SORT}, \vb{rat:nat->rat} and \vb{rat:int->rat} in order to
distinguish them from each other.

Note: If the kind can be deduced from the context, it can be omitted. 
For instance, it is sufficient to write \vb{rat(-(1))}.
\end{example}

Fortunately, the typing mechanism of {\sc Opal} is powerful enough to
deduce the kinds in the majority of situations. Hence, the programs
are not usually burdened with too many explicit denotations of origins
and kinds. As a matter of fact, such annotations are the programmer's
last resource on the few occasions where some ambiguity remains with
respect to overloaded identifiers.

\subsection{Instantiation}

So far, we have looked at simple names. In case of parameterized
structures, we have to distinguish between objects from different
instantiations of a structure. Thus, the origin divides into the {\em
origin identifier} and the (origin) {\em instantiation}. Either of
these two parts can be omitted, if it is deducible from the context.

\begin{example}
    If we import both \vb{Seq[int]} and \vb{Seq[real]}, we have to
    distinguish between the two instantiations of the sort \vb{seq} by
    using their origins:
\begin{verbatim}
    IMPORT  Seq[int]   ONLY seq
            Seq[real]  ONLY seq
    FUN intsToReals: seq'Seq[int]->seq'Seq[real]
\end{verbatim}
If the origin identifier can be deduced from the context, we may leave
it out:
\begin{quote}
\vb{FUN intsToReals: seq[int]->seq[real]}
\end{quote}
\end{example}

\subsection{Overload Resolution}
\label{OverloadResolution}

In general, an object must be identified unequivocally. This is done
by using the explicitly written parts of the name together with the
information deduced from the context. If this ``maximal'' information
still matches several visible names, we have a context error.

There is one exception to this rule: Where an overloaded identifier is
imported, all its matching variations are imported. (If this is to be
prohibited, one has to qualify the desired variant by appropriate
annotations.)

\begin{example}
    Suppose that the structure Seq defines four overloaded
    versions of the function \vb{++}, namely, concatenation of two
    sequences, of sequence and element, of element and sequence, and,
    finally, of two elements. If we wish to import only one of them,
    we may write
\begin{verbatim}
    IMPORT  Seq  ONLY  ++ :seq**seq->seq
\end{verbatim}
\end{example}

\section{Scopes}

The scope of a name depends on the place of its declaration. In some
cases, the scope can be extended and there may be {\em holes} in the
scope.

\subsection{Global Names in a Structure}

\paragraph{}

The sorts and operations in the signature are {\em global names}. 
(Note that this includes the imported sorts and operations as well.)
\begin{itemize}
  \item For the global names that are introduced in the {\em
    signature} part of a structure, the scope is the signature part and
    the implementation part.
  \item For the global names that are additionally introduced in the
    {\em implementation} part, the scope is only the implementation part.
\end{itemize}
There may be {\em holes} in the scope caused by local names (see
section \ref{LocalName} below).

\paragraph{}

Global names may be declared several times; the repeated declarations
introduce only one global name. (For instance, a repetition of the
signature part in the implementation part would be legal; in practice,
repeated declarations may occur through imports.)

\subsection{Local Names}
\label{LocalName}

\paragraph{}

Pattern variables, lambda- and let-bound variables are {\em local
names}. Local names consist of their identifiers and their kind only;
there is no origin. The scope of a pattern variable is the function
definition in which it is declared (see section \ref{operation}); the
scope of a lambda- or let-bound variable is the whole expression in
which it is declared (see sections \ref{LambdaAbstraction} and
\ref{ExtendedExpression}).
\begin{example}
\begin{verbatim}
    FUN quickSort:seq[nat]->seq[nat]
    DEF quickSort(s) == 
      IF s <>? THEN <>
      IF s ::? THEN
         LET compare == ft(s)
             smaller == filter(\\x.x<compare)(rt(s))
             equal   == filter(\\y.y=compare)(s)
             larger  == filter(\\z.z>compare)(rt(s))
         IN quickSort(smaller)++(equal++quickSort(larger))
      FI
\end{verbatim}
The pattern variable \vb{s} can be used in the whole expression,
whereas the let-bound variables \vb{compare}, \vb{smaller}, \vb{equal}
and \vb{larger} can only be used inside the extended expression, and
the lambda-bound variables \vb{x}, \vb{y} and \vb{z} only in the
corresponding lambda abstraction.
\end{example}

\paragraph{}

Local names may cause holes in the scopes of global names. This
happens where the local name has the same identifier as a global
operation. In other words, there is no overloading between global and
local names, except for sorts; their identifier may coincide with a
local name.
\begin{example}
    In the following definition, the function \vb{length} cannot be
    used in the definition of \vb{volume} because of its argument \vb{length}.
\begin{verbatim}
    FUN length: object->real
    FUN volume: real**real**real->real
    DEF volume(length,width,height) == length*width*height
\end{verbatim}
\end{example}

\paragraph{}

For programming convenience, the definition of local names with the
same identifier in overlapping scopes is forbidden.
\begin{examples}
    In the following implementation, the let-bound variable \vb{z} must be
    different from the pattern variables:
\begin{verbatim}
    DEF f(x,y) == x*(IF y<0 THEN z WHERE z==y*y ELSE y FI)
\end{verbatim}
    However, the definition of two identical local names in
    non-overlapping (local) scopes is permitted: 
\begin{verbatim}
    DEF f(x) == IF x<0  THEN res WHERE res==x*neg(x)
                IF x>=0 THEN res WHERE res==x*x
                FI
\end{verbatim}
    In the case of pattern-based definitions, there may be identical
    pattern variables in different equations:
\begin{verbatim}
    DEF <> ++s == s
    DEF s++ <> == s
    DEF (ft::rt)++s == ft::(rt++s)
\end{verbatim}
\end{examples}

\chapter{Programming in {\sc Opal}}

This section describes some of the questions arising when running an
{\sc Opal} program.

\section{What is an {\sc Opal} Program?}

An {\sc Opal} {\em program} consists of a {\em top-level structure}
and the structures that are (transitively) imported by it. The import
relation must be acyclic. For all structures constituting a program, a
signature and an implementation part must exist. The top-level
structure must export (at least) one special constant operation, the
{\em top-level command} (see section \ref{IO}).

An {\sc Opal} program can be compiled and linked together with a small
runtime library to produce a self-contained, executable program. 
During the linking phase, the user is requested to specify the
top-level command that is to be interpreted when the program is
executed. This top-level command is interpreted in the runtime
environment, which usually leads to a complete sequence of
input/output actions.

\section{Input/Output}
\label{IO}

{\sc Opal} realizes input/output by {\em commands\/} of the sort \verb"com".
Typical -commends are requests such as ``read the next input'' and ``write this
afterwards''.  Whenever a command is executed by the runtime system, it returns
an answer of the sort \vb{ans}.  Both sorts are defined in the structure
\vb{Com}.  Obviously, executing a program is a (complex) request to the runtime
system.  Therefore, the sort of the top-level command of an {\sc Opal} program
must be of sort \vb{com}.

The synchronization of commands and answers can be achieved through
the introduction of continuations. A {\em continuation} simply
encapsulates the synchronization between the user and the operating
system. Commands can be built up by using the function \vb{;} provided
by the structure \vb{ComCompose}. They have two ``parts'': a simple
command which the runtime system will interpret, and a continuation
function that accepts the operating system answer and yields a new
command. The runtime system will call this function after it has
processed the request, and will pass as arguments the result of the
request. Since the result of the continuation is a new command, the
operating system will again interpret it, and so on.

This interface to I/O is monadic in the sense of
\cite{Peyton93,Wadler92}. Computations are distinguished from values in that
coomputations have parametrized type {\tt com}. Computations which just
return values are constructed by the (injection) function {\tt
yield}. Different computations are composed using the function {\tt ;} and
these functions observe the laws defining monads. To our knowledge {\sc
Opal} is the first programming language with an implemented monadic
interface to I/O, since it was completed in this form in 1990.

\subsection{A Simple Program}

Our first program is an {\sc Opal} program that echoes the user`s
input until an empty line is entered. The structure
\vb{MyFirstProgram} is the top-level structure with \vb{echo} as the
top-level command:
\begin{verbatim}
    SIGNATURE MyFirstProgram
    IMPORT  Void          ONLY void
            Com[void]     ONLY com
    FUN echo: com[void]                           -- top-level command

    IMPLEMENTATION MyFirstProgram
    IMPORT  Void          ONLY void nil
            Nat           ONLY nat 0            
            Char          ONLY char newline
            String        ONLY string empty?
            Com           ONLY com ans exit okay fail
            ComCompose    ONLY ;
            Stream        ONLY input stdIn readLine
                               output stdOut write
    DEF echo == readLine(stdIn) ; processline                   -- (1)
    FUN processline: ans[string]->com[void]
    DEF processline(okay(s)) ==
        IF s empty? THEN exit(0)                                -- (2)
        ELSE write(stdOut,s) ; (write(stdOut,newline) ;         -- (3)
                       (readLine(stdIn) ; processline)) FI      -- (4)
    DEF processline(fail(_)) ==
        write(stdOut,"Cannot read user input")                  -- (5)
\end{verbatim}

The following aspects deserve mention:

Firstly, the \vb{;} in line(3) is the standard case for a
``write output'' command that is followed by some other command
ignoring the value returned by the runtime system. Here, \vb{;} has
the functionality
\begin{verbatim}
    com[void]**com[void]->com[void]
\end{verbatim}

Secondly, the \vb{;} in line (1) with the functionality
\begin{verbatim}
    com[string]**(ans[string]->com[void])->com[void]
\end{verbatim}
is a typical way of building up a ``read input'' command followed by a
function that processes its result. If \vb{readLine:input->com[string]} is
successfully evaluated, the operating system returns the variant \vb{okay}
of the sort \vb{ans[string]} that is fed up into the continuation
function \vb{processline}. In case an error results, the \vb{fail}
variant is returned.

Thirdly, line (2) and (5) are simple commands: the program will
terminate.

\subsection{State-Preserving Beyond Input/Output Boundaries}

In general, commands may consist of input/output actions involving
arbitrary complex data structures. State-preserving beyond
input/output boundaries is realized by partial instantiation of
functions that must still have the functionality
\vb{ans[\ldots]->com[\ldots]} as their result functionality.

Our second program calculates the average of a sequence of numbers.  Here, the
first parameter group of the function \vb{average} is used to transfer the count
and sum of the already processed input. To shorten the presentation, we ignore
structure boundaries and import clauses and do not give the implementations of
\vb{readNat} and \vb{writeNat} (they are not part of the library).
\begin{verbatim}
    FUN prog2: com[void]                       -- top-level command
    DEF prog2 == (writeLine(stdOut,1stLn) ;
                  writeLine(stdOut,2ndLn)) ;
                 (prompt ; average(0,0))
        WHERE 1stLn=="Average: Please type a sequence of numbers"
              2ndLn=="         0 stops input and prints the result:"
    FUN prompt: com[nat]
    DEF prompt == write(stdOut,"> "!) ; readNat(stdIn)
    FUN average: nat**nat->ans[nat]->com[void]
    DEF average(count,sum)(okay(n)) ==         -- process correct input
        IF n=0 THEN write(stdOut,"= ") ;
                    (writeNat(stdOut,sum/count) ;
                     write(stdOut,newline))
               ELSE prompt ; average(count+1,sum+n) FI
    DEF average(count,sum)(fail(_)) ==         -- ignore errors
        prompt ; average(count,sum)
\end{verbatim}

\subsection{Not Inherently Sequential Input/Output}

With our fully referentially transparent commands, we are even able to
cope with the organization of input that is not inherently sequential.

Suppose we wish to write a function \vb{readLisp} that should
substitute each atom of a lisp-like tree by a number that is to be
interactively entered by the user. We define a command that preserves
the shape of the given lisp-like structure in the following way (note
that \vb{o} is the function composition function, and that \vb{\&} is a
variant of \vb{;} that calls the continuation function only on success
with the returned data):
\begin{verbatim}
    DATA lisp == atom(valOf:nat)
                 cons(car:lisp, cdr:lisp)
    FUN readLisp: input**lisp->com[lisp]
    DEF readLisp(in,atom(x)) ==
        readNat(in) & (yield o           -- build a command that yields
                        (okay o          -- the answer consisting of
                           atom))        -- an atom embedding
                                         -- the result of readNat
    DEF readLisp(in,lisp(l1,l2)) ==
        readLisp(in,l1) &                -- call readLisp of car
        (\\newl1.readLisp(in,l2) &       -- then call readLisp of cdr
           (\\newL2.yield(okay(cons(newl1,newl2)))
                                         -- then yield the new cons
\end{verbatim}

\section{The Library}

With the exception of \vb{bool'BOOL} and \vb{denotation'DENOTATION},
{\sc Opal} has no sorts built into the language. Consequently, every
data type that is to be used in a program has first to be defined by
means of a suitable structure. It obviously makes no sense, though, to
let every programmer start again from scratch. So there are some basic
structures available in the programming environment of the {\sc Opal}
compiler.

It should be noted, however, that these structures are not different
from any other structure. In particular, most of them are written in
{\sc Opal}. There are also some structures, though, that are
substituted by hand-written ones, having the same semantics as an {\sc
Opal} implementation, but more efficient in terms of storage and/or
time. (The {\sc Opal} user is not aware of the kind of structure he
uses.) 

The structure \vb{Seq}, for example, contains at least those functions
that are presented in section \ref{ParameterizedStructures}. A
detailed description of the library is given in "Bibliotheca Opalica - A
Document on Structured Use and Abuse", which is available in the
distribution of the {\sc Opal} system. 

\begin{appendix}

\chapter{{\sc Opal} Syntax}
\label{OpalSyntax}

The formal definition of the syntax is given in EBNF. For each grammar
rule, the corresponding context conditions and attributes are
specified informally. The context conditions are preceded by
`\copyright', and the attributes by `\cs'. Syntactic transformations are
preceded by \Da. The context conditions are additionally
supplied with an underlined short name reflecting the corresponding
error. Some conditions that are considered by the identification
function are double-underlined.  For convenience's sake the grammar
rules are presented in semantic sections, each beginning with a short
description. Furthermore, they are formulated with a view to providing
good readability for humans rather than facilitating parsing.

The metasymbols of EBNF are set as \ms{[}, \ms{]}, \ms{(}, \ms{)},
\ms{$|$}, \ms{$\|$}, \ms{$^+$} and \ms{$^*$}; nonterminals are set as
exemplified by \nt{SignaturePart} and terminals as \tm{IMPORT}.

The other extensions to normal BNF have the following meanings (see
\cite{Waite84}): \[
\begin{array}{ll}
    Abbreviation & Meaning \\
    \\ X \ms{::=} \alpha \ms{(} \beta \ms{)} \gamma & X \ms{::=}
    \alpha Y \gamma \\ & Y \ms{::=} \beta \\
    \\ X \ms{::=} \alpha \ms{[} \beta \ms{]} \gamma & X \ms{::=}
    \alpha \gamma \ms{$|$} \alpha \ms{(} \beta \ms{)} \gamma \\
    \\ X \ms{::=} \alpha u\ms{$^+$} \gamma & X \ms{::=} \alpha Y
    \gamma \\ & Y \ms{::=} u \ms{$|$} Y u \\
    \\ X \ms{::=} \alpha u\ms{$^*$} \gamma & X \ms{::=} \alpha \ms{[}
    u\ms{$^+$} \ms{]} \gamma \\
    \\ X \ms{::=} \alpha \ms{$\|$} t & X \ms{::=} \alpha \ms{(} t
    \alpha \ms{)}\ms{$^*$}
\end{array}
\]

Here, $\alpha$, $\beta$ and $\gamma$ are arbitrary right-hand sides of
rules, $Y$ is a new nonterminal, $u$ is either a single symbol or a
parenthesized right-hand side, and $t$ is a terminal symbol.

\section{Definitions and general context conditions}

{\em Names} are used to refer to objects in a structure. To distinguish
overloaded objects, names have two components besides the {\em identifier}: the
origin and the kind. The {\em origin} reflects the (instance of a) structure in
which the name is declared; the {\em origin identifier} is the name of this
structure, the {\em instantiation} is a list of names (called {\em instance
  names\/}) refering to the formal or actual parameters of the structure. The
{\em kind} of a name distinguishes sorts and operations, and for operations it
reflects the operation's functionality based on the names of the used sorts.

There are two kinds of names, depending on their scope. {\em Global names} refer
to sorts and operations of a structure.  A global name can be given {\em
  attributes} indicating that it is the {\em n$^{\mbox{th}}$ parameter\/} of a
structure, it is a {\em free constructor\/} and its n$^{\mbox{th}}$ instance
name must be {\em known\/}. {\em Local names\/} refer to pattern variables,
lambda-bound and let-bound variables, all of which are operations; they consist
only of an identifier and a kind. There are three general context conditions for
names:
\begin{concon}
  \item \dul{local name as actual parameter}: The instantiation of a name must
    be a list of global names.
  \item \dul{compound object}: The functionality of an operation's
    name may not be a Cartesian product.
  \item \ul{local sort}: A local name may not be a sort.
\end{concon}
A {\em signature} is a set of names. The {\em exported signature} of a
structure contains the names that can be imported by other structures.  The
{\em global signature} of a structure (or its signature or implementation
part) consists of the global names of the structure (or its parts). It is
divided into two disjoint sets: the {\em internal signature} contains the
names that are {\em declared\/} by an explicit declaration; the {\em
  external signature} contains the {\em imported\/} names.  The same global
name can be declared (or imported) twice.

A global name must be complete in order to be used. A global name is {\em
  complete} if all the names in its constituent parts are elements of the global
signature. For this, it is sufficient to demand that a name of the global
signature is semicomplete (and this is necessary for uninstantiated imports). A
global name is {\em semicomplete} if all names in its constituent parts are
elements of the global signature or un{\em known\/} instance names. There are
two general context conditions for the names of the global signature:
\begin{concon}
  \item \ul{incomplete name}: All names of the global signature must
    be semicomplete.
  \item \ul{recursive name}: No name of the global signature may
    contain itself unless it is a parameter.
\end{concon}
Function definitions and their parts have a {\em local signature},
which is the set of global and local names that can be used in them. 
Each global name of the global signature is contained in a local
signature if it is a sort, or there is no local name with the same
identifier in this local signature. A local name is contained in a
local signature if it is contained in the local signature of the
surrounding construct, or it is explicitly {\em added} to the local
signature. There are two general context conditions for local
signatures:
\begin{concon}
  \item \dul{overloaded local name}: There may not be two local
    names with the same identifier in a local signature.
  \item \dul{duplicate local name}: A local name may not be added
    twice to a signature.
\end{concon}
A {\em substitution} with respect to a list of {\em formal} names, and a list of
{\em actual\/} names of the same length, is a function that yields for a name
the original name with all occurrences of the formal names replaced by the
corresponding actual names. If a substitution replaces the n$^{\mbox{th}}$
instance name of the original name, the n$^{\mbox{th}}$ instance name of the
yielded name must be {\em known\/}. In case of a {\em free\/} substitution, the
n$^{\mbox{th}}$ instance name must not necessarily be known.

A substitution is {\em proper} if both lists have the same length and the
substitution of the kind of each formal name yields the kind of the
corresponding actual name. A substitution of a signature is the set of
substitutions of all its elements.

In the syntax, the denotation of a name may omit parts of it. Such a
denotation is called a {\em partial name}. A partial name {\em matches} a
name if all its given parts match. If parts of an origin are given, the
partial name does not match any local name. The {\em matching set\/} of a
partial name is the set of matching names in the global or local signature
under consideration.

A {\em selection} is a function which maps every partial name (except
those listed in selected imports) to an element of its matching set,
called the {\em selected name}. A selection is {\em more specific}
than another if they only differ at points where the first yields a
global name and the other a local name. An {\em identification} is a
selection, where all double-underlined context conditions are
fulfilled and there is no more specific selection that considers these
conditions.  The selected name for an identification is called the
{\em identified name}. There are two major context conditions:
\begin{concon}
  \item \ul{undefined identification}: An identification must exist,
    i.e., there is a function from partial names to names so that all
    double-underlined context conditions are fulfilled for the
    identified names.
  \item \ul{ambiguous identification}: The identification must be
    unique, i.e., there is a most specific function from partial names to
    names so that all double-underlined context conditions are fulfilled.
\end{concon}

An {\em application analysis\/} is a function which takes a sequence of
expressions and assigns each operator its arguments called {\em identified
arguments\/}. An argument is an {\em identified argument\/} iff it is type
correct regarding the functionality of the operator. If the analysis can find an
unambiguous and type correct application structure for all expressions of the
sequence it delivers the deduced structure with one operator as {\em top
operator\/} of the structure. If the analyzed structure is ambiguous the analysis
tries to build an unambiguous application considering right-associciativity for
the operator. An error is yielded if no umambiguous application regarding the
types and the right-associative rule can be deduced. 

Each name of the internal signature (except the parameters) can be
implemented. An {\em implementation} of a name is a syntactical part
of the program text such as a data type definition or a set of
function definitions. There are two general context conditions:
\begin{concon}
  \item \dul{parameter implementation}: There may not be an
    implementation of a parameter.
  \item \ul{duplicate implementation}: There may not be more than one
    implementation of a name.
\end{concon}

\section{Structures}

A program is a collection of structures. A structure consists of a
signature part and an implementation part. The former is the export
interface of the structure and can be compiled separately from the
latter.

\begin{concon}
  \item \ul{duplicate structure name}: Every structure in a program
    must have a unique identifier.
\end{concon}

\begin{synrule}
  \begin{tabular}{lll}
  \nt{SignaturePart} & \ms{::=} & \tm{SIGNATURE} \nt{Ident} \ms{[}
    \tm{[} \nt{Name} \ms{$\|$} \tm{,} \tm{]} \ms{]} \\
        & &  \ms{(} \nt{Signature} \ms{$|$} \nt{FreeType} \ms{)}\ms{$^*$}
  \end{tabular}
\end{synrule}

\begin{concon}
  \item[\cs] The global signature of the signature part consists of
    all names that are declared or imported in the signature part.
  \item[\cs] The n$^{\rm th}$ identified name of the optional \nt{Name} list is the
    n$^{\rm th}$ parameter.
  \item \ul{duplicate parameter}: The identified names must be
    different.
  \item \dul{imported parameter}: The identified names must be
    elements of the internal signature.
  \item \dul{invisible parameter kind}: The kind of the identified
    names may not contain names of the internal signature unless they
    are parameters themselves.
  \item[\cs] The actual origin of the structure consists of \nt{Ident} as
    origin identifier, and the list of identified names as
    instantiation.
  \item[\cs] The exported signature of the structure is the signature
    of the signature part without the parameters.
  \item \ul{empty export}: The exported signature may not contain only names
  with the origin \tm{BOOL} or \tm{DENOTATION}.
  \item \ul{possible name clash}: The internal signature of the
    signature part may not contain two names for which there are actual
    parameters, so that their substitutions with respect to the formal and
    actual parameters yield the same name.
\end{concon}

\begin{synrule}
  \begin{tabular}{lll}
    \nt{ImplementationPart}   &  \ms{::=} & \tm{IMPLEMENTATION} \nt{Ident}
      \ms{[} \tm{[} \nt{Name} \ms{$\|$} \tm{,} \tm{]} \ms{]} \\
     & &  \ms{(} \nt{Signature} \ms{$|$} \nt{FreeType} \\
     & & \ms{$|$} \nt{DataDefinition} \ms{$|$} \nt{FunDefinition} \ms{)}\ms{$^*$}
  \end{tabular}
\end{synrule}
\begin{concon}
  \item \ul{no signature part}: There must be a signature part with
    \nt{Ident} as structure identifier.
  \item[\cs] The global signature of the implementation part consists of the
    global signature of the corresponding signature part (with all free
    constructor attributes reset) together with all names that are declared or
    imported in the implementation part.
  \item \dul{different parameter list}: If a \nt{Name} list is given, the
    n$^{\rm th}$ identified name must be the n$^{\rm th}$ parameter, and all
    parameters must be given.
\end{concon}

\section{Signature}
\label{signature}

The signature provides sorts and operations of the internal and
external signature. Repeated declarations and imports are possible.

\begin{synrule}
  \nt{Signature} \ms{::=} \tm{SORT} \nt{Ident}\ms{$^+$} \ms{$|$}
    \tm{FUN} \ms{(} \nt{Ident}\ms{$^+$} \tm{:} \nt{Functionality} \ms{)}
    \ms{$^+$} \ms{$|$} \nt{Import} \ms{$|$} \nt{Law}
\end{synrule}
\begin{concon}
  \item[\cs] In case of a \tm{SORT} declaration, for each \nt{Ident} a
    name with the given \nt{Ident} as identifier, the actual origin as
    origin, and \tm{SORT} as its kind is declared to be element of the
    internal signature.
  \item[\cs] In case of a \tm{FUN} declaration, for each \nt{Ident} a
    name with the given \nt{Ident} as identifier, the actual origin as
    origin, and the given functionality as its operation's kind is
    declared to be element of the internal signature.
\end{concon}

\begin{synrule}
  \nt{Import} \ms{::=} \tm{IMPORT} \ms{(} \nt{Ident} \ms{[} \tm{[}
    \nt{Name} \ms{$\|$} \tm{,} \tm{]} \ms{]} \nt{Selection} \ms{)}\ms{$^+$}
\end{synrule}
\begin{concon}
  \item[\cs] The structures \tm{BOOL} and \tm{DENOTATION} are always
    imported completely.
  \item \ul{unknown structure}: \nt{Ident} must be the identifier of
    a structure in the program.
  \item \ul{cyclic import}: The structures must be in an acyclic
    import relation. (In particular, \nt{Ident} may not be the
    identifier of the actual structure.)
  \item \ul{uninstantiated import in signature part}: If an import
    of a parameterized structure occurs in a signature part, a \nt{Name}
    list must be given.
  \item[\cs] The identified names of the optional \nt{Name} list are
    called actual parameters.
  \item \dul{improperly instantiated import}: If a \nt{Name} list is
    given, the substitution with respect to the formal and actual
    parameter lists must be proper.
  \item[\cs] If a \nt{Name} list is given, all names of the structure's exported
    signature are substituted with respect to the formal and actual parameters.
    The substituted names constitute the import's signature. If no list is
    given, the union of all proper free substitutions of the names of the
    structure's export signature is the import's signature.
\end{concon}

\label{selection}
\begin{synrule}
    \nt{Selection} \ms{::=} \tm{ONLY} \nt{Name}\ms{$^+$}
      \ms{$|$} \tm{COMPLETELY}
\end{synrule}
\begin{concon}
%  \item \ul{complete import in signature part}: In a signature part,
%    only \tm{ONLY} imports may be used.
  \item[\cs] In case of an \tm{ONLY} import, for each \nt{Name} all
    matching names of the import's signature are imported.
  \item \dul{empty import}: In an \tm{ONLY} import, each \nt{Name}
    must match at least one name of the import's signature.
  \item \ul{instantiation in selection}: If \nt{Name} matches one
    substitution of an exported name in an uninstantiated import, it
    must match all its other proper substitutions, too.
  \item[\cs] In case of a \tm{COMPLETELY} import, all names of the
    import's signature are imported.
\end{concon}

\begin{synrule}
  \begin{tabular}{lll}
    \nt{Functionality} &\ms{::=}& \nt{ProductFct} \ms{$|$} \nt{FunctionFct}\\
    \nt{ProductFct}   &\ms{::=}& \nt{SimpleFct} \ms{$\|$} \tm{**} \ms{$|$}
      \tm{(} \nt{ProductFct} \tm{)}\\
    \nt{FunctionFct} &\ms{::=}& \nt{ProductFct} \tm{->} \nt{Functionality}
      \ms{$|$} \tm{(} \tm{)} \tm{->} \nt{Functionality}\\
    \nt{SimpleFct} &\ms{::=}& \nt{Name} \ms{$|$} \tm{(}
      \nt{FunctionFct} \tm{)} \ms{$|$} \tm{(} \nt{SimpleFct} \tm{)}
  \end{tabular}
\end{synrule}
\begin{concon}
  \item \dul{operation as sort}: The identified name must be a sort.
\end{concon}

\section{Free Types and Data Definitions}

Sorts are implemented by data definitions and can be specified by free types.

\begin{synrule}
  \nt{FreeType} \ms{::=} \tm{TYPE} \nt{Name} \tm{==} \nt{Variant}\ms{$^+$}
\end{synrule}
\begin{concon}
  \item[\cs] The name consisting of the identifier of the partial name
    \nt{Name}, the actual origin, and \tm{SORT} as its kind is declared
    to be element of the internal signature.
  \item \dul{improperly named free type}: \nt{Name} must match the
    declared name.
  \item \dul{parameter as free type}: \nt{Name} may not be a
    parameter.
  \item \ul{duplicate free type}: There may not be more than one
    \nt{FreeType} for the declared name.
  \item[\cs] All constructors that are declared in the \nt{Variant} list
    are free constructors.
  \item \ul{duplicate free constructor/discriminator}: All constructors and
    discriminators that are declared in the \nt{Variant} list must be
    different.
  \item \ul{parameter as free constructor/discriminator/selector}: No
    constructor, discriminator or selector may be a parameter.
\end{concon}

\begin{synrule}
    \nt{DataDefinition} \ms{::=} \tm{DATA} \nt{Name} \tm{==} \nt{Variant}\ms{$^+$}
\end{synrule}
\begin{concon}
  \item[\cs] The name consisting of the identifier of the partial name
    \nt{Name}, the actual origin, and \tm{SORT} as its kind is declared
    to be element of the internal signature.
  \item[\cs] \nt{DataDefinition} is an implementation of the declared
    name and of all selectors of the \nt{Variant} list.
  \item[\cs] \dul{improperly named data definition}: \nt{Name} must
    match the declared name.
  \item[\cs] If there is no free type for the declared name, all
    constructors that are declared in the \nt{Variant} list are said to be
    free constructors.
\end{concon}

\begin{synrule}
  \nt{Variant} \ms{::=} \nt{Name} \ms{[} \nt{Components} \ms{]}
\end{synrule}
\begin{concon}
  \item[\cs] The name consisting of the identifier of the partial name
    \nt{Name}, the actual origin, and the ``constructor functionality''
    as its kind is declared to be an element of the internal signature. 
    If \nt{Components} are given, the constructor functionality is the
    function functionality with the functionality of \nt{Components} as
    its domain and the implemented (or specified) sort as its co-domain,
    otherwise the constructor functionality is the implemented sort. The
    declared name is called constructor.
  \item \dul{improperly named constructor}: \nt{Name} must match the
    constructor.
  \item[\cs] The name consisting of the identifier of the partial name
    \nt{Name} with an appended question mark, the actual origin, and the
    ``discriminator functionality'' as its kind is declared to be an
    element of the internal signature. The discriminator functionality
    is the function functionality with the implemented (or specified)
    sort as its domain and the predefined sort \tm{bool} as its
    co-domain. The declared name is called discriminator.
  \item[\cs] \nt{Variant} is an implementation of the constructor and of
    the discriminator, if it is part of a \nt{SortImplementation}.
\end{concon}

\begin{synrule}
  \nt{Components}       \ms{::=} \tm{(} \ms{(} \nt{Name} \tm{:}
    \nt{Functionality} \ms{)} \ms{$\|$} \tm{,} \tm{)}
\end{synrule}
\begin{concon}
  \item[\cs] For each \nt{Name}, the name consisting of the identifier
    of the partial name \nt{Name}, the actual origin, and its ``selector
    functionality'' as its kind is declared to be an element of the
    internal signature. The selector functionality is the function
    functionality with the implemented (or specified) sort as its domain
    and the given functionality as its co-domain. The declared name is
    called selector.
  \item \dul{improperly named selector}: \nt{Name} must match the
    declared name.
  \item \ul{duplicate selector definition}: All selectors must be
    different.
  \item \ul{tupled component}: No \nt{Functionality} may be a
    Cartesian product.
  \item[\cs] The functionality of \nt{Components} is the Cartesian product of
    all functionalities.
\end{concon}

\section{Function Definitions}

Operations are implemented by a set of function definitions. The
left-hand side of a function definition determines the definition
target. There are three different kinds of expressions on the
left-hand side that are similar but differ in details: complete
left-hand sides, patterns and pattern constructors. {\em Complete
left-hand sides} are expressions that contain the definition target,
which is applied to {\em patterns}. A pattern is a {\em pattern
constructor} applied to patterns or a pattern variable, or a pattern
constructor. To shorten the syntax description, they are only
distinguished by the additional attribute {\em expression kind}.

\begin{concon}
  \item[\cs] The set of \nt{FunImplementation} with the same definition
    target is an implementation of the definition target.
\end{concon}

\begin{synrule}
  \nt{FunDefinition} \ms{::=} \tm{DEF} \nt{TopLeftHandSide} \tm{==} \nt{Expression}
\end{synrule}
\begin{concon}
  \item[\cs] \nt{LeftHandSide} is a complete left-hand side.
  \item[\cs] The pattern variables of \nt{LeftHandSide} are added to
    the local signature of the \nt{Fun\-Def\-i\-ni\-tion}.
  \item \dul{wrongly typed implementation}: The functionality of
    \nt{LeftHandSide} must equal \nt{Ex\-press\-ion}.
\end{concon}

\begin{synrule}
  \begin{tabular}{lll}
  \nt{TopLeftHandSide} & \ms{::=} & \nt{SimpleLhs} \ms{$|$} \nt{LhsTopInfix} \\
  \nt{SimpleLhs}    & \ms{::=} & \nt{LhsName} \ms{$|$} \nt{LhsTuple} \ms{$|$}
  \nt{LhsApply} \ms{$|$} \nt{Wildcard} 
  \end{tabular}
\end{synrule}

\begin{concon}
  \item[\cs] If \nt{SimpleLhs} is a \nt{Wildcard} then \nt{SimpleLhs} is a
  pattern. 
\end{concon}

\begin{synrule}
  \begin{tabular}{lll}
  \nt{LhsTopInfix}        & \ms{:= } & 
                       \nt{SimpleLhs LhsName} \ms{[} \nt{SimpleLhs} \ms{]}
                       \ms{$|$} \tm{(} \tm{)} \nt{LhsName} 
  \end{tabular}
\end{synrule}
\begin{concon}
  \item \ul{higher-order infix pattern constructor}: \nt{LhsTopInfix}
    may not be a higher-order pattern constructor.
  \item[\cs] Both \nt{SimpleLhs} are patterns. \nt{LhsName} is a
    pattern constructor if \nt{LhsTopInfix} is a pattern, and it is
    complete if \nt{LhsTopInfix} is complete.
  \item \dul{wrongly typed lhs infix}: The functionality of
    \nt{LhsName} must be a function functionality with the (flattened)
    Cartesian product of the functionalities of the given \nt{SimpleLhs}
    (or the empty Cartesian product) as its domain.
  \item[\cs] The functionality of \nt{LhsTopInfix} is the co-domain of the
    functionality of \nt{LhsName}.
\end{concon}

\begin{synrule}
  \begin{tabular}{lll}
  \nt{LeftHandSide} & \ms{::=} & \nt{SimpleLhs} \ms{$|$} \nt{LhsInfix}\\
  \nt{LhsInfix}   &  \ms{::=} & \nt{SimpleLhs LhsName} \\
           & & \ms{$|$} \nt{SimpleLhs} \ms{(} \nt{LhsName SimplLhs} \ms{)$^+$} \\
              & & \ms{$|$} \tm{(} \tm{)} \nt{LhsName} 
  \end{tabular}
\end{synrule}
\begin{concon}
  \item \ul{higher-order infix pattern constructor}: \nt{LhsInfix}
    may not be a higher-order pattern constructor.
  \item[\cs] All \nt{SimpleLhs} are patterns. \nt{LhsName} is a
    pattern constructor if \nt{LhsInfix} is a pattern, and it is
    complete if \nt{LhsInfix} is complete.
  \item[\cs] All \nt{LhsName} must be the same complete name. 
  \item[\cs] The binding of \nt{LhsName} is right-associative. 
  \item \dul{wrongly typed lhs infix}: The functionality of
    \nt{LhsName} must be a function functionality with the (flattened)
    Cartesian product of the functionalities of the given \nt{SimpleLhs}
    (or the empty Cartesian product) as its domain.
  \item[\cs] The functionality of \nt{LhsInfix} is the co-domain of the
    functionality of \nt{LhsName}.
\end{concon}

\begin{synrule}
  \nt{LhsTuple} \ms{::=} \tm{(} \ms{(} \ms{[} \nt{LocalName} \tm{AS} \ms{]}
   \nt{LeftHandSide} \ms{)} \ms{$\|$} \tm{,} \tm{)} \\ \contsynrule \ms{[} \tm{:}
    \nt{Functionality} \ms{]}
\end{synrule}
\begin{concon}
  \item[\cs] Each \nt{LeftHandSide} inherits its expression kind from
    \nt{LhsTuple}.
  \item \ul{improper tuple}: If \nt{LhsTuple} is not a pattern, it
    must not contain more than one \nt{LeftHandSide}.
  \item \ul{improper pattern synonym}: If \nt{LocalName} is given,
    \nt{LeftHandSide} must be a pattern.
  \item[\cs] The identified names for every \nt{LocalName} are pattern
    variables.
  \item \dul{wrongly typed pattern synonym}: The functionality of a
    pattern variable and the functionality of the corresponding
    \nt{LeftHandSide} must be equal.
  \item[\cs] The functionality of \nt{LhsTuple} is the (flattened)
    Cartesian product of the functionalities of all \nt{LeftHandSides}.
  \item \dul{wrong typing of lhs tuple}: The functionality of
    \nt{LhsTuple} must equal a given \nt{Functionality}.
\end{concon}

\begin{synrule}
  \nt{LhsApply} \ms{::=} \nt{SimpleLhs} \ms{(} \nt{LhsTuple} \ms{$|$}
    \tm{(} \tm{)} \ms{)}
\end{synrule}
\begin{concon}
  \item \dul{higher-order pattern constructor}: \nt{LhsApply} may
    not be a higher-order pattern constructor.
  \item[\cs] \nt{LhsTuple} is a pattern. \nt{SimpleLhs} is a pattern
    constructor if \nt{LhsApply} is a pattern, and it is complete if
    \nt{LhsApply} is complete.
  \item \dul{wrongly typed lhs apply}: The functionality of
    \nt{SimpleLhs} must be a function functionality with the
    functionality of the given \nt{LhsTuple} (or the empty Cartesian
    product) as its domain.
  \item[\cs] The functionality of \nt{LhsApply} is the co-domain of the
    functionality of \nt{SimpleLhs}.
\end{concon}

\begin{synrule}
  \nt{LhsName} \ms{::=} \nt{Name}
\end{synrule}

The context conditions and attributes for \nt{LhsName} differ for
complete expressions, patterns and pattern constructors. They are
therefore listed separately.

If \nt{LhsName} is a complete left-hand side:
\begin{concon}
  \item[\cs] The identified name is the definition target.
  \item \dul{improperly named function}: The identified name must be
    an element of the internal signature.
  \item \dul{sort defined by function definition}: The identified
    name must be an operation.
  \item[\cs] The functionality of \nt{LhsName} is the functionality of
    the identified name.
\end{concon}
If \nt{LhsName} is a pattern constructor:
\begin{concon}
  \item \dul{non-constructor used as constructor}: The identified
    name must be a free constructor.
  \item[\cs] The functionality of \nt{LhsName} is the functionality of
    the identified name.
\end{concon}
If \nt{LhsName} is a pattern, there are two possibilities:
\begin{concon}
  \item[\cs] If the identified name is a local name, it is a pattern
    variable.
  \item \dul{non-constant constructor used as pattern}: If the identified
    name is a global name, it must be a free constructor with a sort as its
    functionality.
  \item[\cs] The functionality of \nt{LhsName} is the functionality of
    the identified name.
\end{concon}

\section{Expressions}

Expressions are used to define the right-hand side of function definitions.

\begin{synrule}
  \begin{tabular}{lll}
  \nt{Expression} & \ms{::=} & \nt{SimpleExpression} \ms{$|$} \nt{Infix} \ms{$|$}
    \nt{Abstraction} \\ 
    & & \ms{$|$} \nt{Cases} \ms{$|$} \nt{Let} \ms{$|$} \nt{Where}\\
  \nt{SimpleExpression} & \ms{::=} & \nt{ValueDenotation} \ms{$|$} \nt{Tuple}
    \ms{$|$} \nt{Application} \\
  \nt{ValueDenotation} & \ms{::=} & \nt{Name} \ms{$|$} \nt{Denotation} \ms{$|$}
  \nt{SectionPattern}
  \end{tabular}
\end{synrule}
\begin{concon}
  \item \dul{sort as operation}: The identified name must be an
    operation.
  \item[\cs]      The functionality of \nt{Name} is the functionality of
    the identified name, the functionality of \nt{Denotation} is the
    predefined sort \tm{denotation}. The functionality of
    \nt{ValueDenotation} is the functionality of the given alternative.
\end{concon}

\begin{synrule}
  \nt{Tuple}    \ms{::=} \tm{(} \nt{Expression} \ms{$\|$} \tm{,} \tm{)} \ms{[} \tm{:} \nt{Functionality} ]
\end{synrule}
\begin{concon}
  \item[\cs] The functionality of \nt{Tuple} is the (flattened)
    Cartesian product of the functionalities of all \nt{Expressions}.
  \item \dul{wrong typing of tuple}: The functionality of \nt{Tuple}
    and a given \nt{Functionality} must be equal.
\end{concon}

\begin{synrule}
  \nt{Application}      \ms{::=}        \nt{SimpleExpression} \ms{(} \nt{Tuple} \ms{$|$} \tm{(} \tm{)} )
\end{synrule}
\begin{concon}
  \item \dul{wrongly typed application}: The functionality of
    \nt{Simple\-Ex\-press\-ion} must be a function with the
    functionality of the given \nt{Tuple} (or the empty Cartesian
    product) as its domain.
  \item[\Da] If \nt{Tuple} contains \nt{SectionPattern} then \nt{Application} is
   transformed as follows:
   \begin{center}
        \nt{SimpleExpression} \vb{(} $e_1,e_2,\dots, e_n$ \vb{)} $\wedge$ 
        $\exists e_i \stackrel{\ast}{\rightarrow}$ \nt{SectionPattern} \\
        \Da \\
        $\forall e_j \dots e_k \stackrel{\ast}{\rightarrow}$ \nt{SectionPattern}
        $\wedge$ $ne_j \dots ne_k$ are new variables $\wedge$ \\
        \verb"\\" $ne_j,\dots,ne_k$ . \nt{SimpleExpression} 
        \vb{(} $e_1,e_2,\dots, ne_j, \dots, ne_k, \dots ,e_n$ \vb{)} 
   \end{center}
  \item[\cs] The functionality of non-terminal \nt{Application} is the
    co-domain of the functionality of \nt{SimpleExpression}.
\end{concon}

\begin{synrule}
  \begin{tabular}{lll}
  \nt{Infix} & \ms{::=} & \nt{SimpleExpression Name} \\
 & & \ms{$|$} \nt{SimpleExpression} \ms{(} \nt{Name SimpleExpression} \ms{)$^+$} \\
 & & \tm{(} \tm{)} \nt{Name}
  \end{tabular}
\end{synrule}
\begin{concon}
  \item \dul{sort as infix operation}: All identified names must be
    an operation.
  \item[\cs] All identified names must be the same name. 
  \item \dul{wrongly typed infix}: The functionality of the
    identified name must be a function functionality with the
    (flattened) Cartesian product of the functionalities of the given
    \nt{SimpleExpression} (or the empty Cartesian product) as its
    domain.
  \item[\cs] The functionality of \nt{Infix} is the co-domain of the
    functionality of the identified {\em top operator\/}. 
\end{concon}

\begin{synrule}
  \nt{Abstraction} \ms{::=}  \tm{\bs\bs} \ms{[} \nt{LocalName} \ms{$\|$} \tm{,} \ms{]}
    \tm{.} \nt{Expression}
\end{synrule}
Since \nt{Abstraction} has a higher priority than \tm{WHERE}
expressions, \nt{Expression} cannot be a \tm{WHERE} expression.
\begin{concon}
  \item[\cs] The identified names for every \nt{LocalName} are lambda-bound
    variables.
  \item[\cs] The lambda-bound variables are added to the local
    signature of \nt{Abstraction}.
  \item[\cs] The functionality of \nt{Abstraction} is a function
    functionality with the Cartesian product of the functionalities of the
    lambda-bound variables as its domain and the functionality of
    \nt{Expression} as its co-domain.
\end{concon}

\begin{synrule}
    \nt{Cases}  \ms{::=}        \nt{Guard}\ms{$^+$} \ms{$\|$} \tm{OTHERWISE}
      \ms{[} \tm{ELSE} \nt{Expression} \ms{]} \tm{FI} \ms{[} \tm{:} \nt{Functionality} ]
\end{synrule}
\begin{concon}
  \item \dul{incompatible guards}: The functionalities of each
    \nt{Guard} must be equal.
  \item[\cs] The functionality of \nt{Cases} is the functionality of
    the first \nt{Guard}.
  \item \dul{incompatible else}: The functionality of \nt{Cases} and
    the functionality of a given \nt{Expression} must be equal.
  \item \dul{wrong typing of cases}: The functionality of \nt{Cases}
    and a given \nt{Functionality} must be equal.
\end{concon}

\begin{synrule}
  \nt{Guard} \ms{::=} \tm{IF} \nt{Expression} \ms{$\|$} \ms{(} \tm{ANDIF}
  \ms{$|$} \tm{ORIF} \ms{)} \tm{THEN} \nt{Expression}
\end{synrule}
\begin{concon}
  \item[\Da] \rule{0pt}{1pt}
    \begin{center}
        \nt{Expression$_1$} \tm{ANDIF} \nt{Expresion$_2$} \\
        \Da \\
        \tm{IF} \nt{Expression$_1$} \tm{THEN} \nt{Expression$_2$} \tm{ELSE}
        \vb{false} \tm{FI}
    \end{center}
    and 
    \begin{center}
        \nt{Expression$_1$} \tm{ORIF} \nt{Expresion$_2$} \\
        \Da \\
        \tm{IF} \nt{Expression$_1$} \tm{THEN} \vb{true} \tm{ELSE} 
        \nt{Expression$_2$} \tm{FI}
    \end{center}

  \item[\cs] \tm{ANDIF} and \tm{ORIF} are right-associative. 
  \item[\cs] The functionality of \nt{Guard} is the functionality of
    the second expression.
  \item \dul{wrongly typed condition}: The functionality of the first
    expression must be the predefined sort \tm{bool}.
\end{concon}

\begin{synrule}
  \nt{Let} \ms{::=} \tm{LET} \nt{Equation}\ms{$^+$} \tm{IN} \nt{Expression}
\end{synrule}

Owing to the binding rules of \tm{WHERE} expressions, only the last
\nt{Equation} may have a \tm{WHERE} expression as its right-hand side. 
Since \tm{LET} has a higher priority than \tm{WHERE}, \nt{Expression}
cannot be a \tm{WHERE} expression.
\begin{concon}
  \item[\cs] The let-bound variables of the \nt{Equation} list are
    added to the local signature of \nt{Let}.
  \item \ul{recursive let}: There must exist an order of all
    \nt{Equation\/}s, so that every \nt{Equation} in which a let-bound variable
    is used on its right-hand side is preceded by the \nt{Equation} in
    which this variable is used on the left-hand side.
  \item[\cs] The functionality of \nt{Let} is the functionality of
    \nt{Expression}.
\end{concon}

\begin{synrule}
  \nt{Where} \ms{::=} \nt{Expression} \tm{WHERE} \nt{Equation}\ms{$^+$}
\end{synrule}

All the following equations bind to the \tm{WHERE} expression.
\begin{concon}
  \item[\cs] The let-bound variables of the \nt{Equation} list are
    added to the local signature of \nt{Where}.
  \item \ul{recursive where}: There must exist an order of all
    \nt{Equation\/}s, so that every \nt{Equation} in which a let-bound
    variable is used on its right-hand side is preceded by the \nt{Equation}
    in which this variable is used on the left-hand side.
  \item[\cs] The functionality of \nt{Where} is the functionality of
    \nt{Expression}.
\end{concon}

\begin{synrule}
  \nt{Equation} \ms{::=} \ms{(}\nt{LocalName} \ms{$|$} \tm{(} \nt{LocalName}
    \ms{$\|$} \tm{,} \tm{)} \ms{)} \tm{==} \nt{Expression}
\end{synrule}
\begin{concon}
  \item[\cs] The identified names for every \nt{LocalName} are
    let-bound variables.
  \item \dul{wrongly typed equation}: The Cartesian product of the
    functionalities of the let-bound variables and the functionality of
    \nt{Expression} must be equal.
\end{concon}

\subsection{Bracketing of Infix-Expressions}
\label{bracketing}

The \opal{} compiler analyzes mixfix expressions in the context checking
phase. The compiler uses all available  
information for this task. However, there are situations, where there remain
ambiguities, for example in parsing arithmetic expression.

Let \(a \oplus b \otimes c\) the infix expression to be analyzed, and
suppose that both possible bracketings, \((a \oplus b) \otimes c\) and 
\(a \oplus (b \otimes c)\) are type correct.

The \opal{} compiler maintains two relations on functions, $\cal L$ and $\cal
R$. If $(\oplus, \otimes) \in {\cal L}$, the first possibility is chosen,
if $(\oplus, \otimes) \in {\cal R}$, the second possibility is chosen. 
If $(\oplus, \otimes) \not \in ({\cal L \cup R})$, and both functions are
equal, right bracketing is chosen as default.

Defining bracketings for other infix expressions is not yet part of the
language, but may be defined with pragmas.
Function pairs are inserted into $\cal L$ or $\cal R$ by means of the
following pragmas:


\begin{synrule}
\tm{/\$} \tm{BRACKET} \tm{LEFT} \ms{[} \nt{Name} \ms{]} \ms{[} \nt{Name}
\ms{]}  \tm{\$/}
\end{synrule}
\begin{synrule}
\tm{/\$} \tm{BRACKET} \tm{RIGHT} \ms{[} \nt{Name} \ms{]} \ms{[} \nt{Name}
\ms{]}  \tm{\$/}
\end{synrule}

\begin{concon}
  \item [\cs] $\cal L$ and $\cal R$ are extended to their transitive
    closure. 
  \item \ul{$\cal L$ and $\cal R$ must remain disjoint.}
\end{concon}

Bracketings are a superset of priority and left-/right-associativity
declarations as found in other languages:

\begin{tabular}{lcl}
``$*$ has higher priority than $+$'' & $\Leftrightarrow$ &
\vb{/\$ BRACKET LEFT [ * ] [ + ] \$/} \\
& & \vb{/\$ BRACKET RIGHT [ +] [ * ]} \\[1ex]

``$-$  associates to the left'' & $\Leftrightarrow$ & 
\vb{/\$ BRACKET LEFT [ - ] [ - ] \$/}
\end{tabular}



\section{Algebraic Properties}
\label{propertylanguage}

Local names refer to pattern variables, lambda- and let-bound variables as
well as to {\em quantor-bound variables\/}.

Laws and their parts have a {\em local signature\/} like that of function
definitions.

\subsection{Laws}

Laws yield the possibility to write properties of data types and functions.
They occur only in property parts.

\begin{synrule}
  \begin{tabular}{lll}
\nt{Property} & \cce\ & \nt{FunDefinition} \alt\ \nt{DataDefinition} \alt\ \nt{FreeType} \alt\ \nt{Law} \\
\nt{Law} & \cce\ & \tm{LAW} \opt{ \nt{Ident} \tm{==} } \nt{Formula}
  \end{tabular}
\end{synrule}

\begin{concon}
  \cosi{duplicate law} \nt{Ident} must be unique inside the corresponding
  property part.
\end{concon}

\subsection{Formulas}

Formulas are used to define laws.

\begin{synrule}
  \nt{Formula} \cce\ \nt{AtomicFormula} \alt\ \nt{PropositionalFormula} \alt\ 
  \nt{QuantifiedFormula} \\ \nt{AtomicFormula} \cce\ \nt{Definedness} \alt\ 
  \nt{Equality} \alt\ \nt{Inequality} \alt\ \tm( \nt{Formula} \tm) \\ 
  \nt{Definedness} \cce\ \tm{DFD} \nt{Expression} \\ \nt{Equality} \cce\ 
  \nt{Expression} \opt{ \tm{===} \nt{Expression} } \\ \nt{Inequality} \cce\ 
  \nt{Expression} \tm{<<=} \nt{Expression}
\end{synrule}

\begin{concon}
  \codo{wrongly typed equality} The functionality of both \nt{Expression}s
  in an \nt{Equality} or \nt{Inequality} must be equal. If only one
  \nt{Expression} is given, its functionality must be the predefined sort
  \tm{bool}.
\end{concon}

\begin{synrule}
  \begin{tabular}{lll}
\nt{PropositionalFormula} & \cce\ & \nt{Negation} \alt\ \nt{Conjunction} \\
                     & & \alt\ \nt{Disjunction} \alt\
\nt{Implication} \alt\ \nt{Equivalence} \\
\nt{Negation} & \cce\ & \tm{NOT} \nt{Formula} \\
\nt{Conjunction} & \cce\ & \nt{Formula} \tm{AND} \nt{Formula} \\
\nt{Disjunction} & \cce\ & \nt{Formula} \tm{OR} \nt{Formula} \\
\nt{Implication} & \cce\ & \nt{Formula} \tm{==>} \nt{Formula} \\
\nt{Equivalence} & \cce\ & \nt{Formula} \tm{<=>} \nt{Formula}
  \end{tabular}
\end{synrule}

The rules for \nt{PropositionalFormula} are ordered in descending priority and
right-associatity is assumed for equal formulas. 
Since \nt{QuantifiedFormula} has an even less priority the first formula of
\nt{Conjunction}, \nt{Disjunction}, \nt{Implication} and \nt{Equivalence}
cannot contain an unbracketed \nt{QuantifiedFormula}.

\begin{synrule}
  \nt{QuantifiedFormula} \cce\ \bra{ \tm{ALL} \alt\ \tm{EX} } \nt{LocalName}
  \plu\ \tm.  \nt{Formula}
\end{synrule}

\begin{concon}
\item The identified names for every \nt{LocalName} are quantor-bound
  variables. They are added to the local signature of
  \nt{QuantifiedFormula}.
\end{concon}


\section{Partial Names}

Partial names are the syntactical means for refering to names of the
signature under consideration.

\begin{synrule}
  \nt{Name} \ms{\ms{::=}} \nt{Ident} \ms{[} \tm{'} \nt{Ident} \ms{]} \ms{[} \tm{[} \nt{Name} \ms{$\|$} \tm{,} \tm{]} \ms{]} \ms{[} \nt{Kind} ]
  %@@@ is the bracketing correct
\end{synrule}
\begin{concon}
  \item[\cs] \nt{Name} is a partial name with the first \nt{Ident} as
    identifier and, if given, the second \nt{Ident} as origin identifier,
    the list of identified names as instantiation, and \nt{Kind} as kind.
\end{concon}

\begin{synrule}
  \begin{tabular}{lll}
    \nt{Kind}  & \ms{::=} & \tm{:} \tm{SORT} \ms{$|$} \tm{:} \nt{Functionality}\\
    \nt{LocalName} & \ms{::=} & \ms{(} \nt{Ident} \ms{$|$} \nt{Wildcard} \ms{)} 
                   \ms{[} \tm{:} \nt{Functionality} \ms{]} 
  \end{tabular}
\end{synrule}
\begin{concon}
  \item[\cs] \nt{LocalName} is a partial name with \nt{Ident} as identifier and,
  if given, \nt{Functionality} as kind. 
\end{concon}

\begin{synrule}
  \nt{Wildcard} \ms{::=} \tm{\_}
\end{synrule}
\begin{concon}
  \item[\cs] Wildcard is a fresh local name. 
\end{concon}

\begin{synrule}
  \nt{SectionPattern} \ms{::=} \tm{\_}
\end{synrule}
\begin{concon}
  \item[\cs] SectionPattern is a fresh local name. 
\end{concon}

\begin{synrule}
  \nt{Ident} \ms{::=} \nt{Ide} \ms{[} \nt{Pragma} \ms{]}
\end{synrule}
\section{Lexical Rules}

The syntax description treats the program text as a sequence of {\em
lexemes}, i.e., keywords and separators (denoted by their character
representation), identifiers, denotations and pragmas.  Comments and
layout are ignored. If there is no way to partition a program text
into a lexeme sequence, then there is a lexical error. The description
of the lexeme sequence itself treats the program text as a sequence of
{\em symbols}, which are ultimately defined on the basis of
characters.  The different kinds of lexemes and symbols are described
as sets of symbol sequences (or character sequences). Therefore, the
usual set operations (union, set difference) are used together with
sequencing operations (concatenation, repetition). Sets are built from
elements by enclosing the collection of elements (without any
separating comma) in curled brackets.

\subsection*{Lexemes}

The lexemes are defined on the basis of symbols, sometimes denoted by
their character representation. There are seven kinds of lexemes:
keywords, identifiers, denotations, separators, comments, pragmas and
layout.
\begin{tabbing}
  \hspace*{3cm} \= == \= \{ \= \kill

  \nt{Lexeme} \> = \> \nt{Keyword} $\cup$ \nt{Ide} $\cup$ \nt{Denotation}
                      $\cup$ \nt{Separator} $\cup$ \\
              \>   \> \nt{Comment} $\cup$ \nt{Pragma} $\cup$ \nt{Layout} \\
  \nt{Keyword} \> = \> \{ \vb{ALL AND ANDIF AS COMPLETELY DATA DEF DFD ELSE
    EX FI } \\
               \>   \> \> \vb{FUN IF IMPLEMENTATION IMPORT IN LAW LET NOT
                 ONLY} \\
               \>   \> \> \vb{OR ORIF OTHERWISE SIGNATURE SORT THEN TYPE WHERE } \\
               \>   \> \> \vb{ ** -> . : == \_ === <<= ==> <=> \char'134\char'134}~\} \\
  \nt{Ide}     \> = \> ( \nt{Alphanum} $\cup$ \nt{Graphic} ) \bs\ (
                       \nt{Keyword} $\cup$ \{ \vb{--  /* */  /\$ \$/}
\} ) \\
  \nt{Denotation} \> = \> \nt{String} \\
  \nt{Separator}  \> = \> \nt{Delimiter} \\
  \nt{Comment}    \> = \> \vb{--} (\nt{Symbol} \bs \{ nl \} )\ms{$^*$} nl
$\cup$ \\
                \> \> \vb{/*} ( \nt{Symbol} \bs\ \{ \vb{-- /* */}\} $\cup$ \\
              \>   \> \nt{Comment} ) \vb{*/} \\
  \nt{Pragma} \> = \> \vb{/\$} ( \nt{Lexeme} \bs \nt{Pragma} )* \vb{\$/} \\
  \nt{Layout} \> = \> \nt{Blank}\ms{$^+$}
\end{tabbing}

\label{dot-as-keyword}
Note that keywords are reserved words, they cannot be used as identifiers,
except for the keyword \vb{.} (single dot), which is only recognized after
one of the keywords \vb{\bs\bs\ ALL EX}. Note also that there are two kinds
of comments: a {\em line comment} starts with the symbol \vb{--} and skips
all symbols (and therefore all characters) until the next newline; a {\em
  nested comment} is enclosed within the symbols \vb{/*} and \vb{*/} and may
again contain comments. Line comments have a higher priority than nested
comments, i.e., the nested comment symbols \vb{/*} and \vb{*/} inside a line
comment are ignored.

\subsection*{Symbols}

There are six kinds of symbols: alphanumeric and graphic symbols,
delimiters and blanks, strings and wrong symbols. Note that wrong
symbols can occur in correct programs inside comments.
\begin{tabbing}
\hspace*{3cm} \= == \= \{ \= \kill

\nt{Symbol}       \> = \> \nt{Alphanum} $\cup$ \nt{Graphic} $\cup$
                          \nt{Delimiter} $\cup$ \nt{Blank} $\cup$
                          \nt{String} $\cup$ \nt{Wrong} \\
\nt{Alphanum} \> = \> \ms{(} \nt{Letgit} $\cup$ \vb{\_} \ms{)} \ms{$^+$} 
                      \vb{?}\ms{$^*$}\ms{(} \vb{\_} \ms{(} \nt{Alphanum} $\cup$
                      \nt{Graphic} \ms{))$^\ast$} \\
\nt{Graphic}  \> = \> \nt{Special}\ms{$^+$} \ms{(} \vb{\_} \ms{(} \nt{Alphanum}
                 $\cup$ \nt{Graphic} \ms{))$^\ast$} \\
\nt{Delimiter}\> = \> \nt{Extra} \bs\ \{ \vb{"} \} \\
\nt{Blank}    \> = \> \nt{White} \\
\nt{String}   \> = \> \vb{"} ( \nt{Char} \bs\ ( \nt{Other} $\cup$ \{ \vb{"
  tab nl} \} ) $\cup$ \vb{""} $\cup$ \nt{Escape} )\ms{$^*$} \vb{"} \\
\nt{Wrong}    \> = \> \nt{Other} $\cup$ \\
              \>   \> \vb{"} ( \nt{Char} \bs\ ( \nt{Other} $\cup$ \{ \vb{" tab nl} \} ) $\cup$ \vb{""} )\ms{$^*$}
\end{tabbing}

Strings are enclosed in quotes and may contain double quotes;
tabulators, newlines, and non-printable characters cannot be
used in strings, so a string can never extend over more than one line.

\label{escape}
The following escape sequences are recognized within strings and replaced by
one character.

\begin{tabular}{lrll}
\nt{Escape} &= \{ & \vb{\bs a} & (alarm) \\
& & \vb{\bs b} & (backspace) \\
& & \vb{\bs f} & (formfeed) \\
& & \vb{\bs r} & (carriage return) \\
& & \vb{\bs t} & (tabulator) \\
& & \vb{\bs v} & (vertical tab) \\
& & \vb{\bs \bs} & (single backslash) \\
& & \vb{\bs ?} & (questionmark) \\
& & \verb!\'! & (single quote) \\
& & \verb!\"! & (double quote) \\
& & \vb{\bs} \nt{oct} \ms{[} \nt{oct} \ms{[} \nt{oct} \ms{]} \ms{]} \\
& & \vb{\bs x} \nt{hex} \plu \\
  hex & = & \multicolumn{2}{l}{\{ \vb{0 1 2 3 4 5 6 7 8 9 a b c d e f A B C D E F} \}} \\
  oct & = & \multicolumn{2}{l}{\{ \vb{0 1 2 3 4 5 6 7} \}} \\

\end{tabular}

\begin{concon}
\item \ul{number too large} The octal or hexadecimal number may not exceed
  255.
\end{concon}

Wrong symbols are ``incorrect strings'' or non-printable and non-ASCII
characters.  The partitioning of a program text into a symbol sequence
p must obey the condition longest match: for all l,r$\in$\nt{Symbol}*
and u,v$\in$\nt{Symbol} with p=luvr, the following implications must
hold (v$_1$ denotes the first character of v):
\begin{tabbing}
\hspace*{6cm} \= == \= \{ \= \kill

u$\in$\nt{Letgit}\ms{$^+$}                  \> v$_1 \not\in$  ( \nt{Letgit} $\cup$ \{ \vb{?} \} ) \\
u$\in$\nt{Letgit}\ms{$^+$} \vb{?}\ms{$^+$}          \> v$_1 \neq$ \vb{?} \\
u$\in$\nt{Graphic}                  \> v$_1 \not\in$ \nt{Special} \\
u$\in$\nt{String}               \> v$_1 \neq$ \vb{"} \\
u$\in$ \nt{Wrong} \bs \nt{Other}\> v$_1 \not\in$ \nt{Other} $\cup$ \{ \vb{tab nl} \}
\end{tabbing}

\subsection*{Unicode Escapes}
\label{unicode}

The scanner recognizes unicode escapes.

\begin{synrule}
\nt{UnicodeEsc} =  \tm{\bs} \tm{u}\plu \nt{hex} \nt{hex} \nt{hex} \nt{hex} 
\end{synrule}

The unicode escape is replaced by the unicode character with the code number
represented by the four following hexadecimal digits.

Note that unicode escapes are processed before the proper scanner starts.

\begin{concon}
  \item \ul{The last ``\texttt{u}'' must be followed by at least four
    hexadecimal digits. }
  \item \ul{Character codes above 255 are not supported.}
\end{concon}


\subsection*{Characters}
\label{latin1}

The {\sc Opal} character set comprises all printable Latin-1 (ISO-8859-1)
characters as well as space, tabulator, and newline. Other characters can
only be 
used in comments. The character set divides into five classes:

\def\X{\phantom{\{}&}
\begin{tabular}{lcll}
\nt{Char} & = & \X{\nt{Letgit} $\cup$ \nt{Special} $\cup$ \nt{Extra} $\cup$ \nt{Blank} $\cup$ \nt{Other}} \\
\nt{Letgit} & = &  \{ & \vb{A B C D E F G H I J K L M N O P Q R S T U V W X Y Z}\\
            &   &  \X  \vb{a b c d e f g h i j k l m n o p q r s t u v w x y z}\\ 
            &   &  \X \texttt{\ss{} \`a \'a \^a \~a \"a \aa{} \ae{} \c{C} \`e \'e \^e \"e \`\i\ \'\i\ \^\i\ \"\i\ eth \~n \`o \'o \^o \~o \"o \o}\\
            &   &  \X\texttt{\`u \'u \^u \"u \'y thorn \"y \`A \'A \^A \~A \"A \AA\ \AE\ \c{C} \`E \'E \^E \"E \`I \'I \^I \"I Eth} \\
            &   &  \X\texttt{\~N \`O \'O \^O \~O \"O \O\ \`U \'U \^U \"U \'Y Thorn} \\
            &   & \X \vb{0 1 2 3 4 5 6 7 8 9} \} \\
%\nt{Special}& = & \{ \vb{! \# \$ \% \& * + - . / : ; < = > ? @ \ \^ \_ ` \{}
%                  \verb:| \} ~: \} \\
\nt{Special}& = & \{ & \vb{! \# \$ \% \& * + - . / : ; < = > ? @ 
                        \char'134\ \char'176\ \char'174\ \^\ \_ `}  \} \\
            &   & \X \texttt{ !` cent \pounds{} currency yen brokenbar \S \"{\ }\ \copyright{} $^{\mbox{\underline{a}}}$ $\ll$ $\neg$} \\
            &   & \X \texttt{softHyphen registered \={\ }  $^{\circ}$ $\pm$ $^2$ $^3$ \'{\ } $\mu$ \P{} $\cdot{}$   \c{\ } $^1$ $^{\mbox{\underline{o}}}$ $\gg$  }  \\
            &   & \X \texttt{ $\frac{1}{4}$ $\frac{1}{2}$ $\frac{3}{4}$ ?` $\times$ $\div$ } \\  
\nt{Extra}      & = & \{ & \vb{" ( ) , ' [ ]} \} \\
\nt{White}      & = & \{ & space  tab  nl noBreakspace\}  \\
\nt{Other}  & = & \X {all non-printable  characters}
\end{tabular}



%\chapter{The {\sc Opal} Compilation System}
%
%The {\sc Opal} compilation system is available by terms of free licence via
%anonymous ftp from {\tt ftp@cs.tu-berlin.de}. It resides in directory {\tt
%pub/local/uebb/ocs}.
%
% DEMNAECHST ETWAS MEHR INFORMATIONEN ZU OCS !!!

\chapter{Changes}

This chapter lists changes since the 4$^{th}$ edition.

\begin{itemize}
\item Laws are allowed in signature and implementation parts. (See
  Section~\ref{signature}.)  Property parts are no longer needed.
\item \tm{COMPLETELY} import allowed in signature parts. (See
  Section~\ref{selection}.) 
\item Bracketing rules for infix expressions have been added. (See
  Section~\ref{bracketing}.) 
\item A single dot is recognized as keyword only after one of the keywords 
      \vb{\bs\bs\ ALL EX}. In particular, \tm{.} is a valid name for a
      function. (See Section~\ref{dot-as-keyword}.)
\item Escape characters within denotation constants are defined. (See
  Section~\ref{escape}.) 
\item Unicode escapes are recognized. (See Section~\ref{unicode}).
\item The character set has been extended to Latin-1 (ISO-8859-1). (See
  Section~\ref{latin1}.) 
\end{itemize}

\chapter{Acknowledgement}

The language {\sc Opal} has been designed by Gottfried Egger, Andreas Fett,
Carola Gerke, Wolfgang Grieskamp, Michael Jatzeck, Peter Pepper, and Wolfram
Schulte.

Our colleagues and many students have contributed by critically assessing many
aspects of the design and implementation, by teaching {\sc Opal}, and last not
least by giving their invaluable feedback as users of {\sc Opal}.




\end{appendix}


\begin{thebibliography}{99}
  \bibitem{Bauer85} Bauer, F.L. et al.:
    {\em ``The Munich Project CIP''},
    Volume I: The Wide Spectrum Language CIP-L;
    Springer, Lecture Notes in Computer Science 183, 1985
  \bibitem{Bird88} Bird, R., Wadler, P.:
    {\em ``Introduction to Functional Programming.''};
    Prentice-Hall, 1988
  \bibitem{Diller88} Diller, A.:
    {\em ``Compiling Functional Languages.''};
    John Wiley, 1988
  \bibitem{Ehrig85} Ehrig, H., Mahr, B.:
    {\em ``Fundamentals of Algebraic Specifications 1.''};
    Springer Verlag, 1985
  \bibitem{Field88} Field, A.J., Harrison, P.G.:
    {\em ``Functional Programming.''};
    Addison-Wesley, 1988
  \bibitem{Henderson80} Henderson, P.:
    {\em ``Functional Programming: Application and Implementation.''};
    Prentice-Hall, 1980
  \bibitem{Hensen90} Henson, M.C:
    {\em ``Elements of Functional Languages.''};
    Blackwell Scientific Publications, 1987
  \bibitem{Hudak90} Hudak, P. et al.:
    {\em ``Haskell.''}
    Yale University Internal Report, 1990
  \bibitem{Milner88} Milner, R., Tofte, M., Harper, R.:
    {\em ``The Definition of Standard ML.''};
    MIT Press, 1990
  \bibitem{Perry88} Perry, N.:
    {\em ``Hope+. Dept. of Computing''};
    Imperial College London Internal Report
    IC/FPR/LANG  /2.5.1/7, 1988
  \bibitem{Perry87} Perry,  N.:
    {\em ``Hope+C, A Continuation extension for Hope+''};
    Dept. of Computing Imperial College London
    Internal Report IC/FPR/LANG/2.5.1/21, 1987
  \bibitem{Peyton87} Peyton Jones, S.L.:
    {\em ``The Implementation of Functional Programming Languages''};
    Prentice-Hall, 1987
  \bibitem{Peyton93} Peyton Jones, S.L., Wadler, P.:
    {\em ``Imperative functional programming''};
    in: Proceedings of 20th Symposium on Principles of Programming
        Languages, 1993
  \bibitem{Reade89} Reade, C.:
    {\em ``Elements of Functional Programming''};
    Addison-Wesley, 1989
  \bibitem{Sanella87} Sanella, D., Tarlecki, A.:
    {\em ``On Observational Equivalence''\/}; 
    in: Journal of Comp. and Sys.\ Science, n$^o$ 34, 1987
  \bibitem{Wadler92} Wadler, P.:
    {\em ``The essence of functional programming''};
    in: Proceedings 19th Symposium on Principles of Programming
        Languages, 1992
  \bibitem{Waite84} Waite, W.\ M., Goos, G.:
    {\em ``Compiler Construction\/}; Springer Verlag, 1984
\end{thebibliography}

\end{document}


