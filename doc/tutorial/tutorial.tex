% LAST EDIT: Wed May  4 11:08:50 1994 by Juergen Exner (hektor!jue) 
% LAST EDIT: Tue May  3 13:15:52 1994 by Juergen Exner (hektor!jue) 
% LAST EDIT: Mon Apr 18 17:40:25 1994 by Juergen Exner (hektor!jue) 
% LAST EDIT: Tue Feb 15 18:21:50 1994 by Juergen Exner (hektor!jue) 
% LAST EDIT: Mon Feb 14 17:45:23 1994 by Juergen Exner (hektor!jue) 
% LAST EDIT: Sun Feb 13 18:36:24 1994 by Juergen Exner (hektor!jue) 
% LAST EDIT: Mon Feb  7 12:27:58 1994 by Juergen Exner (hektor!jue) 
%\documentstyle[12pt,a4,drop,drafthead,alltt,doublespace]{report}
\documentstyle[12pt,a4,drop,alltt]{report}

% LAST EDIT: Fri Oct  8 17:06:26 1993 by Juergen Exner (hektor!jue) 
% LAST EDIT: Mon Jun 28 10:17:22 1993 by Juergen Exner (hektor!jue) 
% LAST EDIT: Thu Feb 25 13:04:36 1993 by Juergen Exner (hektor!jue) 
% LAST EDIT: Wed Jan 27 12:15:04 1993 by Juergen Exner (hektor!jue) 
% LAST EDIT: Wed Jan 20 14:13:15 1993 by Juergen Exner (hektor!jue) 
\newcommand{\novice}{\medskip\drop{\fbox{\Huge$ \cal N$}}}
\newcommand{\advanced}{\medskip\drop{\fbox{\Huge$ \cal A$}}}
\newcommand{\experienced}{\medskip\drop{\fbox{\Huge$ \cal E$}}}
\newcommand{\important}{\medskip\drop{\fbox{\Huge !}}}
\newcommand{\opal}{{\sc Opal}}


\newcommand{\back}{\char92}
\newcommand{\pro}[1]{{\small\tt #1}}
\newcommand{\key}[1]{{\tt\bf\small #1}}
\newcommand{\spec}[1]{$\it #1 $}

\newenvironment{prog}{\begin{alltt}\small\tt}{\end{alltt}}
% Local Variables: 
% mode: latex
% TeX-master: "main"
% End: 


\title{\Huge \bf The O{\LARGE\bf PAL} Tutorial}
\author{\Large The \opal\ Group\\[2ex]
%{\em }\\
{\bf J\"urgen Exner}\\[8ex]
Technische Universit\"at Berlin\\
Sekretariat FR 5-13\\
Franklinstr. 28--29\\
D--10587 Berlin\\[3ex]
\verb+jue@cs.tu-berlin.de+}
\date{
\vspace*{1cm}
May 1994\\[10ex]
  Report No. 94-9}

\begin{document}
\maketitle

\begin{abstract}
This tutorial describes the functional programming language \opal\
which was     
developed by the \opal\ Group at the Technische Universit\"at
Berlin.

It explains how to use \opal\ from an intuitive approach. No prior
knowledge of functional programming will be assumed.
Although this tutorial explains \opal\ in depth, it does not define
the language.
 In cases of ambiguity, you should consult the reference manual ``The Programming
Language \opal''. 
\end{abstract}

\tableofcontents
%\listoffigures
%\listoftables

% LAST EDIT: Wed May  4 11:39:31 1994 by Juergen Exner (hektor!jue) 
% LAST EDIT: Tue May  3 14:10:06 1994 by Juergen Exner (hektor!jue) 
% LAST EDIT: Wed Apr 27 15:20:51 1994 by Juergen Exner (hektor!jue) 
% LAST EDIT: Fri Apr 22 11:58:14 1994 by Juergen Exner (hektor!jue) 
% LAST EDIT: Mon Apr 18 17:52:13 1994 by Juergen Exner (hektor!jue) 
% LAST EDIT: Tue Feb 15 10:24:13 1994 by Juergen Exner (hektor!jue) 
% LAST EDIT: Sun Feb 13 14:08:42 1994 by Juergen Exner (hektor!jue) 
% LAST EDIT: Mon Nov 15 21:20:52 1993 by Juergen Exner (hektor!jue) 
% LAST EDIT: Wed Oct 20 13:37:31 1993 by Juergen Exner (hektor!jue) 
% LAST EDIT: Mon Oct 18 15:07:49 1993 by Juergen Exner (hektor!jue) 
% LAST EDIT: Mon Jan 11 16:28:25 1993 by Juergen Exner (hektor!jue) 
\chapter{Introduction}
\label{chap:intro}

\hfill\begin{minipage}[t]{7cm}
{\em
A language that does not affect the way you think about programming
is not worth knowing.}\\{\small \mbox{} \hfill \mbox{Author unknown}}
\end{minipage}

\bigskip

In the past programming has been dominated by the traditional style of
imperative programming.
Programming languages like Fortran, Cobol, Algol, Pascal, C and even
Assembler are familiar examples of the imperative 
programming paradigm.

These languages have been oriented towards the internal architecture of the
well-known 
\mbox{von-Neumann} Computer. This implies the main disadvantage of imperative
languages: programming must be oriented towards the architecture of the
computer instead of towards the structure of the problem to be solved.

As early as 1978 J. Backus asked in his Turing Award Lecture ``Can
Programming be Liberated from the von Neumann Style?'' 
Since then increasing effort has been invested in the development of
alternatives to 
imperative programming languages. 
Some of the results are known nowadays by the catchwords `logic
programming` (e.g. Prolog), `object oriented programming` (e.g.\ 
Smalltalk, C++) and `functional (or applicative) programming` (e.g. pure
LISP, ML, HOPE, Haskell).

The programming language \opal\  lies somewhere between  other modern functional
programming languages like ML, HOPE and  Miranda.
\opal\  is a pure functional language without any imperative relics.
In addition to higher-order functions, lambda abstraction and pattern-matching, \opal\ offers a comfortable modularization
(``programming in the large'') and a powerful, orthogonal type system
which includes
generic functions (realized by parameterized structures) and free types. 
 \opal\ also  supports overloading of names (together with a
concise and flexible method for annotation), object declarations,
non-deterministic case-distinctions and, in addition, a new way of
handling infix- and postfix-operators.

In the past functional programming languages have been accused of 
inefficiency with respect to time and space. 
The \opal\ research project has defeated this legend.
By using innovative techniques during compilation the runtime of the
generated object code is of the same magnitude as for hand-written C-code.
This has been proved in  several benchmark tests and sometimes, in very
special cases, the generated code is even more efficient than a
comparable hand-written C-program. 

In any case, the generated code is much faster than that of traditional
functional languages. There are orders of magnitudes between the
execution times of \opal\ and e.g.~ML
or HOPE (see \cite{SchGr} for details).
\medskip


This combination of features seems to be unique and we would
therefore like
to re\-commend the use of \opal\ in research, application programming and
education.
 

\section{Aim of this Tutorial}
\label{sec:aim}

In this tutorial we will not assume the reader to be familiar with
functional programming or any other programming paradigm.
In fact, being familiar with imperative languages , for example, may
be disadvantage, because you will have to alter your way of thinking
about programming,  whereas a novice user is spared this handicap.

Nevertheless, it might be helpful  to have some basic
knowledge about the theory of programming languages or at least about
programming in general. 

\smallskip

With a view to accommodating users from all fields  the goal of this
tutorial will be twofold:
\begin{itemize}
\item On the one hand, we will present a short, but complete
  introduction to the techniques and methods of functional
  programming.
\item On the other hand, we will give a complete introduction to
  programming with \opal.
\end{itemize}
After reading this tutorial a novice user without prior knowledge of
functional programming should be able to develop programs in
functional style using all the usual features of functional programming,
and to implement these programs in \opal.

But remember, you cannot learn a programming language only by reading
about it, you have to write your own programs and learn by use. 
We therefore advise the reader to try the examples in this tutorial, to
modify and enlarge them, and then to write original programs.


\section{Structure of this Tutorial}
\label{sec:structure}

Each chapter of this tutorial treats one aspect of \opal\  in depth.
In general we will use bottom-up methodology, i.e.~we will
start with the smallest parts of a program (the names) and finish by
combining complete libraries to programming systems.

The advantage of this scheme---the information on each topic will be
concentrated in one place and can be easily found by reference to  the
table of contents---unfortunately also implies a great 
disadvantage, particular for an  introductory tutorial.
A novice reader will be overwhelmed by a flood of information he
certainly does not need in the initial stages.

We try to avoid this  by using a second dimension in the structure. 
Each paragraph will be marked with a sign that indicates the target group
of this paragraph:

\novice A section marked with an $\cal N$ like this one addresses
a novice user with no prior knowledge of \opal\  or functional programming.
It contains fundamental information describing the most basic
features of \opal, which are  essential for trivial programs. 

After reading these sections a novice user without prior knowledge
should be able to write, compile and 
execute simple (indeed very simple) \opal\ programs.

These novice sections are really low-level. 
They do not explain any of those features which determine the
power of functional programming, or advanced features of \opal.

Nevertheless they contain vital information about \opal\  and should
therefore {\em not} be skipped by experienced users of functional languages.

\advanced The more advanced features will be explained in sections for
advanced users, marked with an $\cal A$ like this. 
A novice user should skip these sections on the first run, whereas a
user already familiar with functional programming may read them first
time round.

The advanced sections  contain all the information needed to harness 
the full power of functional programming and \opal. 
The concepts and features will be explained in detail, and restrictions
and circumventions will be noted.
In addition, topics only touched on in the novice sections will be discussed in
depth. 

In these sections we presume the reader to be familiar with the
basic concepts and notations of \opal. 
 Often there will be cross-references to other sections and topics,
 since, due to the relations between the different language concepts,
 individual parts of a programming language cannot always be explained
 in isolation
Therefore the reader should be aware that he will sometimes have to read a different
section first before being able to understand the current one.

This concept might be considered disadvantageous, but it is the only
way to get a concise and also complete reference for a
programming language.

\experienced The third kind of sections are those for experienced
users, marked with an $\cal E$ like this.
These sections can be skipped  by the normal user altogether, as they
are only for the experts.
They contain additional background information about special topics and
hints for very special features which won't be used by the average
application programmer.

\important Sometimes a paragraph will be preceded by an exclamation
mark like this. These paragraphs contain important information
and warnings.

\section{Notational Conventions and Terminology}
\label{sec:notation}
\novice 
In order to make this paper easier to read, we will use some
conventions for notations and terminology.
All these conventions will be detailed a second time as soon as  a notion or
notation is used in the following chapters.
So this section serves mainly as a kind of glossary and may
be skipped on a first reading.


\subsubsection{Notations}
\novice 
We will use different fonts to distinguish different objects.

\hspace{1em} The normal font, as used in  this chapter, will be used for flow
text, explanations, remarks and so on.

 Program text will be written in a tty-like font \pro{like this}.
We will also use this font in flow text for program fragments
(e.g.~names of functions), if they belong to a concrete program. 
% For convenience keywords are written in \key{boldface} tty-like style.

Note that the leading numbers  of programs in examples are {\em not}
part of the program. They are only used to reference lines
in the explanations.

In interactive examples the output of the computer will likewise be
denoted in \pro{tty-like style} while the user's input will also be
\underline{\pro{underlined}}.

When arguing about concepts we won't use the tty-like style, but prefer a
more mathematical notation in \spec{a\ font\ like\ this}. 


\subsubsection{Notions}
\novice 
 The notions declaration, definition, signature,
implementation and specification are sometimes applied imprecisely in
literature.
Let us explain their meanings as used in this tutorial by an example:

We want to write a function which doubles its argument.
This is already an (informal) specification.
A {\em specification\/} describes {\em what\/} should be done, e.g.~
doubling the argument.
Specifications are essential for arguing about programs, especially
for program verification.

The {\em signature declares\/} the formal frame:
\begin{prog}
\key{FUN} double : nat -> nat
\end{prog}

The function \pro{double} will take one natural number as argument and
deliver a natural number as the result.
The signature does not describe what a function does or how this will
be done.

The {\em implementation defines\/} how the function works:
\begin{prog}
\key{DEF} double(n) \key{==} n + n
\end{prog}

Sometimes we will use  declaration and definition as synonyms for
signature and implementation.

We won't deal with this topic in detail here, but will return to it
repeatedly in following chapters.

%\advanced The current release of \opal\  does not support specifications.
%We will add them only as comments.
%
%\advanced In  classical imperative languages signature and
%implementation are often mixed together, e.g.~in Modula-2 the procedure
%\begin{prog}
%\key{PROCEDURE} double(n:\key{CARDINAL}):\key{CARDINAL};\\
%\key{BEGIN}\\
%  \key{RETURN} n + n;\\
%\key{END} double;
%\end{prog}
%declares and defines the function \pro{double} in one step.


\section{Release Notes}
\label{sec:releasenotes}
\experienced This  tutorial describes \opal,  Version 2.1, released in
Spring 1994. 
There are some features in this new release, which are not supported
by former versions of \opal.  
These upgrades include:
\begin{itemize}
\item sections
\item enhanced infix notation
\item underscore as wildcard in pattern-matching
\item underscore as combinator for alpha-numerical and graphical
  identifiers
\item sequential guards
\item compiler now assumes right associate operators if brackets are missing
\end{itemize}

\noindent Consult the appropriate sections if you want to know more
about the new features.

 The examples in this tutorial are based on ``Bibliotheca Opalica''
of Spring 1994, as distributed together with the compiler.
This library has been considerably restructured and enhanced. 
See \cite{Di} for upgrading old programs to the new library.


% Local Variables: 
% mode: latex
% TeX-master: "tutorial"
% End: 


% LAST EDIT: Tue May  3 13:16:12 1994 by Juergen Exner (hektor!jue) 
% LAST EDIT: Wed Apr 27 17:08:06 1994 by Juergen Exner (hektor!jue) 
% LAST EDIT: Tue Feb 15 10:46:39 1994 by Juergen Exner (hektor!jue) 
% LAST EDIT: Sun Feb 13 17:51:26 1994 by Juergen Exner (hektor!jue) 
% LAST EDIT: Tue Jan 12 16:07:49 1993 by Juergen Exner (hektor!jue) 
\chapter{A First Example}
\label{chap:example}
\novice
Before starting with the precise description of \opal,  let us first present
a short overview using two introductory examples.

The first is the famous ``HelloWorld'' program. 
The second (a little bit more complex and incorporating simple interactive I/O)
calculates the rabbit numbers invented by the Italian mathematician
Fibbonacci.

These examples are intended only as  short survey and so we won't
discuss all the details; this will be left to the following chapters.

\advanced A third example---not included in this tutorial---can be found in
``The Programming Language \opal\ '' \cite[p.~27]{Pe}. 
It displays the contents of a named file on the terminal.

 A fourth example (``expression'') will be included in the
appendix. 
It simulates a small pocket calculator and illustrates user-defined
types, higher-order functions and more complex interactive I/O.

\medskip
\novice
{\mbox{\bf Note:}} All these examples are included in the
\opal\ distribution and can be found in the subdirectory \pro{examples}.

\section{``Hello World''}
\novice
In the world of programming, writing the words ``Hello World'' on the
terminal seems to be absolutely imperative.
An \opal\ program which does  this would probably look like Figures 2.1
and 2.2.
\begin{figure}[htbp]
  \leavevmode
  \begin{prog}
    1    SIGNATURE HelloWorld
    2    IMPORT  Void            ONLY void
    3            Com[void]       ONLY com
    4    FUN hello : com[void]    -- top level command\end{prog}
  \caption{HelloWorld.sign}
\end{figure}

\begin{figure}[htbp]
  \leavevmode
  \begin{prog}
    1    IMPLEMENTATION HelloWorld
    2    IMPORT  DENOTATION      ONLY denotation
    3            Char            ONLY char newline
    4            Denotation      ONLY ++ \% 
    5            Stream          ONLY output stdOut 
    6                      write : output ** denotation -> com[void]
    7    
    8    -- FUN hello:com[void] (already declared in Signature-Part)
    9    DEF hello ==
    10        write(stdOut, "Hello World" ++ (\%(newline)) )
  \end{prog}
  \caption{HelloWorld.impl}
\end{figure}

In this example the program consists of one structure named
\pro{HelloWorld}, which  is stored physically in two files
(\pro{HelloWorld.sign} and \pro{HelloWorld.impl}). 
The files have to be named using the name of the structure plus the
extension \pro{.sign} or \pro{.impl}.
So the possible names for structures are restricted due to the naming
conventions of the  file system used.

The signature part declares the export interface of a structure.
In the case of a program this must be a constant (e.g.,~a function
without arguments) of sort
\pro{com[void]}\footnote{The type system will be explained in Chapter
\ref{chap:types}, instantiations ([\dots]) in Chapter
\ref{chap:large} and the I/O-system in  Chapter \ref{chap:IO}} whereby
the sorts \pro{com} and \pro{void} must be imported from their
corresponding structures \pro{Com} and \pro{Void}.

In the implementation part we need some additional sorts (\pro{denotation},
\pro{output} and \pro{char}) and operations
(\pro{stdOut}, \pro{write}, \pro{\%},\pro{++} and \pro{newline}) which
are also imported from their corresponding structures.

Line 8 is a comment line, indicated by a leading ``\pro{--}''.

The definition of the constant \pro{hello}, which was declared in the
signature part, defines this function to write a text to \pro{stdOut}
(which is a predefined constant describing the terminal).

The text consists of the words ``Hello World'' and a trailing
newline character, which is converted into a denotation with the operation
\pro{\%} and appended to the text by the function \pro{++}.
 The function \pro{++} is used as an infix operator in this example,
but this is not essential.
%
%\advanced 
%Note, that the exclamation mark ``\pro{!}'' (in this example
%used as a postfix-operator) is essential as \pro{"Hello World"} is
%only a denotation, not a string.
%The function ``\pro{!}'' converts a denotation to a string as required by
%the function \pro{write}.
%More about the differences between denotations and strings can be
%found in Chapter \ref{denotations}.
\newline\rule{5cm}{1pt}

{\small \novice 
  To compile the program you should ensure that the
\opal\ Compilation System (OCS) is properly installed at your site and
that the OCS 
directory \pro{bin} is included in your search path.
The GNU \pro{gmake} must be available too.
If you don't know how to set up  your path or if OCS is not installed,
call a local guru. 


Within the proper environment---assuming the program \pro{HelloWorld}
resides in the current working directory---you just have to type 
\begin{prog}
  >  \underline{ocs -top HelloWorld hello}
\end{prog}
to compile and link the program \pro{HelloWorld} with
top-level-command \pro{hello} and you will receive an executable
binary named \pro{hello}.
You can start this program just by typing 
\begin{prog}
  >  \underline{./hello}
\end{prog}

For more information about using OCS try 
\begin{prog}
  >  \underline{ocs help}
\end{prog}
or 
\begin{prog}
  >  \underline{ocs info}
\end{prog} 
and consult the OCS-guide ``A User's Guide to the \opal\  Compilation
System'' \cite{Ma} and the man-pages.
\newline\rule{5cm}{1pt}

\novice
 A pseudo-interpreter (oi) for \opal\ programs is also available.
Although this interpreter is not intended for complete programs, it is
very helpful in the development of separate structures as it simplifies
testing considerably. For details, see ``The \opal\ Interpreter'' \cite{Le}.
}



\section{Rabbit Numbers}
\novice
Let us have a second example. Imagine a population of rabbits which
propagate according to the following rules:
\begin{itemize}
\item In the first generation\footnote{As we are good computer scientists we
    will start numbering at 0.} there is only one young couple of
  rabbits.
\item In each following generation the former young couples become grown-ups.
\item In each generation each already grown-up couple produces one couple of
  young rabbits.
\item Rabbits never die.
\end{itemize}

To calculate the total number of couples you may combine the two
functions and you will receive the so-called Fibbonacci-Numbers. 
We will just call them {\em rabbits\/}:
\[ \mbox{\em rabbits}(\mbox{\em gen}) = \left\{ \begin{array}{ll}
1 & \mbox{if\ {\em gen\/}} = 0\\
1 & \mbox{if\ {\em gen\/}} = 1\\
\mbox{\em rabbits}(\mbox{\em gen}-1) + \mbox{\em rabbits}(\mbox{\em gen}-2) & \mbox{if\ {\em gen\/}} > 1\\
\end{array}\right. 
\]


\begin{figure}[hbtp]
  \leavevmode
  \begin{center}
    \begin{prog}
   1    SIGNATURE Rabbits
   2
   3    IMPORT  Void            ONLY void
   4            Com[void]       ONLY com
   5
   6    FUN main : com[void]    -- top level command
    \end{prog}
    \caption{Rabbits.sign}
  \end{center}
\end{figure}

\begin{figure}[hbtp]
  \leavevmode
  \begin{prog}
   1    IMPLEMENTATION Rabbits
   2
   3    IMPORT  Denotation      ONLY  ++
   4            Nat             ONLY nat ! 0 1 2 - + > =
   5            NatConv         ONLY `
   6            String          ONLY string
   7            StringConv      ONLY `
   8            Com             ONLY ans:SORT
   9            ComCompose      COMPLETELY
  10            Stream          ONLY input stdIn readLine
  11                               output stdOut writeLine
  12                               write:output**denotation->com[void]
  13
  14    -- FUN main : com[void] -- already declared in signature part
  15    DEF main ==
  16        write(stdOut,
  17           "For which generation do you want 
                        to know the number of rabbits? ") &
  18        (readLine(stdIn)    & (\LAMBDA in.
  19         processInput(in`)
  20        ))
  21
  22    FUN processInput: denotation -> com[void]
  23    DEF processInput(ans) ==
  24              LET generation == !(ans)
  25                  bunnys     == rabbits(generation)
  26                  result     == "In the "
  27                                ++ (generation`)
  28                                ++ ". generation there are "
  29                                ++ (bunnys`)
  30                                ++ " couples of rabbits."
  31              IN writeLine(stdOut, result)
  32    -- ----------------------------------------------------------
  33
  34    FUN rabbits : nat -> nat
  35    DEF rabbits(generation) ==
  36            IF generation = 0 THEN 1
  37            IF generation = 1 THEN 1
  38            IF generation > 1 THEN rabbits(generation - 1)
  39                                   + rabbits(generation - 2) FI
 \end{prog}
    \caption{Rabbits.impl}
  \label{fig:rabbits}
\end{figure}

In the example (Figure \ref{fig:rabbits})
%, shortened to the
%essential parts; the whole program will be included in appendix
%\ref{prog:rabbits}) 
this formula can be found in the definition of the
function \pro{rabbits} (lines 35--39), which is a direct
1-to-1-translation of the mathematical notation.

\advanced This direct translation is typical for functional
programming. 
In contrast to traditional imperative languages, you don't need to
think about variables and their actual values (which 
must be supervised very carefully), about call-by-value or
call-by-reference parameters or about pointers to results and
dereferencing them.
 This also applies to real problems, not only to  such trivial
examples as the rabbit numbers.

\novice This example also illustrates local declarations as another
feature of \opal. 
In lines 24--30 three local declarations are established as
notational abbreviations:
\begin{itemize}
\item \pro{generation} stands for the number the user has typed,
\item \pro{bunnies} is the computed number of couples,
\item and \pro{result} is the final answer (as a text) of the program.
\end{itemize}

By using these abbreviations the logical structure of the program will be
emphasized and the main action can be notated in a very short form:
\begin{prog}
      write(stdOut, result)
\end{prog}
Local declarations will be  detailed in Chapter
\ref{chap:small}, Section \ref{subsec:objdecl}



% Local Variables: 
% mode: latex
% TeX-master: "tutorial"
% End: 

% LAST EDIT: Thu Apr 28 10:05:07 1994 by Juergen Exner (hektor!jue) 
% LAST EDIT: Sun Feb 13 18:09:17 1994 by Juergen Exner (hektor!jue) 
% LAST EDIT: Mon Nov 15 22:30:02 1993 by Juergen Exner (hektor!jue) 
% LAST EDIT: Fri Nov 12 15:43:03 1993 by Juergen Exner (hektor!jue) 
% LAST EDIT: Mon Oct 18 16:34:53 1993 by Juergen Exner (hektor!jue) 
% LAST EDIT: Wed Oct 13 15:14:32 1993 by Juergen Exner (hektor!jue) 
\chapter{Names in \opal}
\label{chap:names}
\novice
 Names are the basis for all programming because you need names to identify
the objects of your algorithm.
In \opal\ the rules for constructing names are more complex than in
most traditional languages for two reasons:

First, \opal\ also allows the construction of  identifiers with graphical
symbols like ``\pro{+}'', ``\pro{-}'', ``\pro{\%}'' or ``\pro{\#}'',
which can be used the same way as the established identifiers
made up of  letters and digits. 
This will be explained in the following section.

Second, \opal\ supports overloading and parameterization and thus requires
a method for annotations (see Section~\ref{sec:name}: ``What's the
name of the game?''). 


\section{Constructing Identifiers}
\label{sec:ide}
\novice
In this section we will explain which characters can be used to
build an identifier and which rules have to be fulfilled.

For constructing an identifier all  printable characters are divided
into three classes:
\begin{itemize}
\item upper-case letters,  lower-case letters and  digits
  (e.g., ``\pro{A}'', ``\pro{h}'', ``\pro{1}'')
\item  graphical symbols: these are all printable character with the
   exception of  letters, digits and separators\footnote{%
Although   question mark ``\pro{?}'' and  underscore
  ``\pro{\_}'' belong to this group too, they have special meanings
  (see below).}. 
Examples are ``\pro{+}'', ``\pro{\$}'', ``\pro{\char`@}'', ``\pro{\{}'',
``\pro{!}'' 
\item  separators: these are ``\pro{(}'', ``\pro{)}'', ``\pro{,}'',
  ``\pro{`}'', ``\pro{"}'', ``\pro{[}'', ``\pro{]}'' together with
   space,  tabulator and newline. 
   The last three are often called ``white space''.

\end{itemize}

You may construct identifiers from either of the first two classes.
This means ``\pro{HelloWorld}'',
``\pro{a1very2long3and4silly5identifier6with7a8lot9of0digits}'', \newline
``\pro{A}'', ``\pro{z}'', 
``\pro{2345}'', ``\pro{+}'', ``\pro{\#}'', ``\pro{---->}'',
``\pro{\%!}'' are legal identifiers in \opal.

But note that you cannot mix letters and digits with graphical
characters in one identifier, e.g., ``\pro{my1Value<@>37arguments}'' is a
list of  three identifiers  ``\pro{my1Value}'', ``\pro{<@>}'' and
``\pro{37arguments}''.

\begin{sloppypar}
    The case of a letter is significant, so ``\pro{helloworld}'',
``\pro{HelloWorld}'' and  ``\pro{HELLOWORLD}'' are three different
identifiers. 
\end{sloppypar}


\medskip
 Separators cannot be used in identifiers at all. They are
reserved for special purposes and delimit any identifier they are
connected to. 

Summarizing, one can say an identifier is the longest possible
sequence of characters,  either of letters and digits or graphical
characters.


\subsection{Question Mark and Underscore}
\label{sec:question mark}
\novice
Although question marks are graphical symbols, they can also be used
as trailing characters of an identifier based on letters or
digits. 
This exception was introduced because the discriminators of data types
are constructed by appending  a question mark to the constructor (for
details, see Chapter \ref{chap:types}: ``Types'').

\medskip
\advanced
An underscore-character ``\pro{\_}'' has a special meaning too.
First it is a member of both character classes, letters as well as
graphical symbols.
Hence it can be used to switch between the character classes
within one identifier. 
E.g.,~``\pro{my1Value\_<@>\_37arguments}'' is---in contrast to
above---only one identifier. 

Furthermore, a single underscore is a reserved keyword  with
two applications (as wildcard and as keyword for sections, see
\ref{sec:wildcard} and \ref{sec:sections} for details). 
Therefore you should be careful when using underscores.
%
%\begin{itemize}
%\item as a ``wildcard'' in pattern-based function definition (similar
%  e.g.~to Prolog); see Section \ref{sec:wildcard}: ``Using Wildcards in
%  Pattern-based Definitions'' for details.
%\item as a keyword for sections, i.e.~a shorthand notation for simple
%  lambda abstractions; see Section \ref{sec:sections} for details.
%\end{itemize}

\subsection{Keywords}
\label{sec:keywords}
\novice
Some identifiers are reserved as keywords, e.g.,~``\pro{IF}'',
``\pro{IMPLEMENTATION}'', ``\pro{FUN}'', ``\pro{:}'',``\pro{\_}'', ``\pro{->}''
(see ``The Programming Language \opal'' for a complete list of all keywords).

These keywords cannot be used as identifiers any more, but it is no
problem to use them as part of an identifier: ``\pro{myFUN}'',
``\pro{THENPART}'',``\pro{IFthereishope}'',``\pro{::}'' and
``\pro{-->}'' are legal identifiers.

Moreover, because upper and lower case are significant,
``\pro{if}'' and ``\pro{fun}'' are legal identifiers  and not keywords.

\important This is a common reason for curious errors.
A  programmer will write
\begin{prog}
      FUN help : ...
\end{prog}

\noindent to start the declaration of the function \pro{help}. 
The line 
\begin{prog}
      FUNhelp : ...
\end{prog}
obviously means something completely different (in this case it will
be an error), but a programmer will recognize this error at once. 

The same error---a missing space---written with graphical symbols is
much less obvious! 
You must take care to write
\begin{prog}
  FUN # : nat -> nat
\end{prog}
instead of 
\begin{prog}
  FUN #: nat -> nat
\end{prog}
So, if you receive a curious error message, first check that you have
included all necessary separators between identifiers and keywords.

\section{``What's the name of the game?''}
\label{sec:name}
\novice
In the previous section we discussed the construction of single
identifiers.
For several reasons---the two most important are
overloading\footnote{%
  Using the same identifier for different functions
  is called overloading. In traditional programming languages this
  feature is  common for built-in data types (e.g., the symbol \pro{+} for
  addition of natural and real numbers), but it's very rarely
  supported for user-defined functions.}  
and parameterized structures
(see Section \ref{sec:struct.param})---an identifier alone does not
suffice to really identify an object under all circumstances.
It is for this reason that an  identifier can be annotated with additional
information. 

 By carefully analyzing the environment of an identifier,  the
 compiler will nearly always detect by itself which operation to be 
used.
In very complex cases this detection may fail and you will receive an error
message saying something like ``ambiguous identification''.
In these cases you can annotate the identifier to help the compiler.

\advanced
In addition, you must annotate the instance of a parameterized structure
at least once, either at the import or at the application point (see
Chapter \ref{sec:struct.param} for details).

\medskip
 Annotations are always appended to an identifier, a white
space\footnote{Blanks, tabulators and newlines}  may be
added between identifier and annotations, but this is not necessary.

 The following annotations in particular are possible. You may omit
each of them, but if you supply two or more, they must be supplied in
the order presented below:
\begin{itemize}
\item{\bf Origin:} The origin of an object is the name of the
  structure in which this object was declared. 

The annotation of  the origin is introduced by a ``\pro{'}'', followed by
the name of the origin structure.

E.g., \pro{-'Nat} identifies the function \pro{-} from the structure
\pro{Nat}, whereas \pro{-'Int} identifies the function \pro{-} from the
structure \pro{Int}.

\item{\bf Instantiation:} The instantiation of an object of a
  parameterized structure (see \ref{sec:struct.param} for details)
  will be annotated by appending the parameters  to
  the identifier in square brackets.
E.g.,~if you want to use the function \pro{in} from the library
structure \pro{Set} with a set of natural numbers you may annotate
\pro{  in[nat,<]}---or in even more detail---\pro{  in'Set[nat'Nat,
  <'Nat]}. 

\item {\bf  Kind:} The kind of an object is either the keyword
  ``\pro{SORT}'' or the functionality of the object.
The kind is appended with a ``\pro{:}''. E.g.,~the name
\pro{-~:int->int}\/ identifies the unary minus, whereas
\pro{-~:int**int->int}\ identifies the usual dyadic minus\footnote{%
Remember the space between \pro{-} and \pro{:}.}. 
\end{itemize}





% Local Variables: 
% mode: latex
% TeX-master: "tutorial"
% End: 

% LAST EDIT: Wed May  4 11:40:52 1994 by Juergen Exner (hektor!jue) 
% LAST EDIT: Tue May  3 14:14:44 1994 by Juergen Exner (hektor!jue) 
% LAST EDIT: Fri Apr 29 15:07:10 1994 by Juergen Exner (hektor!jue) 
% LAST EDIT: Thu Apr 28 12:11:30 1994 by Juergen Exner (hektor!jue) 
% LAST EDIT: Tue Feb 15 14:23:58 1994 by Juergen Exner (hektor!jue) 
% LAST EDIT: Mon Feb 14 14:03:32 1994 by Juergen Exner (hektor!jue) 
% LAST EDIT: Sun Feb 13 19:55:39 1994 by Juergen Exner (hektor!jue) 
% LAST EDIT: Tue Nov 16 10:51:39 1993 by Mario Suedholt (kreusa!fish) 
\chapter{Programming in the Small}
\label{chap:small}
\novice
In functional languages the algorithms are expressed in terms of
functions---as the name ``functional programming'' already implies.
In this chapter we will explain how to declare and define functions
(Section \ref{sec:fun}) and how to use functions to express algorithms
(Section \ref{sec:expr.simple}).
We will also describe the rules for scoping and visibility of
names (Section \ref{sec:scope}).

\medskip
This chapter is called ``Programming in the Small'' because functions
are used to structure a program in a fine grain.
There are also features for  structuring a program in a coarse grain, which
means summarizing several functions (and data types) in structures
of their own. 
This will be explained in Chapter \ref{chap:large}: ``Programming in
the Large''.
\medskip

The examples in this chapter are generally taken from the two example 
programs,
\pro{HelloWorld} and \pro{Rabbits} (see Chapter
\ref{chap:example}:~``A first Example''), for the novice, and from
\pro{Expressions} (see Appendix \ref{app:expressions}) for the more
advanced features.
Before reading the advanced paragraphs you should first read about data
types  (see Chapter \ref{chap:types}: ``Types''), because the program
``\pro{Expression}'' uses a lot of data type definitions and the
algorithms are based on these.


\section{Declaration and Definition of Functions}
\label{sec:fun}
\novice
Functions are the basis of functional programming and they are the
only way to express algorithms in purely functional languages
like \opal.
Although traditional imperative languages  generally also  offer
functions, their usage is  restricted by several constraints.

In \opal\ functions are much more general and there is  in fact no
difference between functions and ordinary values as is the case in
imperative languages.
Both can be used in exactly the same way (this is sometimes
apostrophized as
``functions as first-class-citizens'').
From now on we will just say ``object'' if we don't want to
distinguish between functions and ordinary elements of data types.

\medskip
To write a function you perform two steps: you have to declare and you have
to define the function. 
Both steps will be explained in the following.


\subsection{Declaration of Functions}
\label{sec:fun.decl}
\novice The declaration tells about the {\em kind\/} of arguments and results
of a function.
It does not fix what the function will do (this is the task of the
specification)
% as \opal\ does not support specifications in the
%current release we won't mention them any further
or how this will be done (this is the object of the definition; see below).

\medskip
A function declaration is introduced by the keyword \pro{FUN},
followed by the name of the function, a colon and the functionality of
the function. 
 The declaration
\begin{prog}
        FUN rabbits : nat -> nat
\end{prog}
declares the function named \pro{rabbits}, which will take one natural
number as argument and deliver a natural number as result.

The argument (and also the result, see below) could be tuples:
\begin{prog}
        FUN add : nat ** nat -> nat
\end{prog}
This means the function \pro{add} will take two natural numbers as
arguments and deliver one natural number as result. 

Remember, a declaration does not say  anything about what the
function will do. 
The function \pro{add} might deliver as  result the minimum of the
two arguments; this would be rather contra-intuitive and should therefore
be avoided.

 The arguments may be missing altogether as in
\begin{prog}
        FUN main : com
\end{prog}
This means the function \pro{main} takes no argument, and in this case
we say \pro{main} is a constant of the sort \pro{com} (which means
\pro{command}; see Chapter \ref{chap:IO}: ``Input/Output in OPAL'' for
details about commands).

 The sorts of arguments and results need not to be the
same. The function
\begin{prog}
        FUN ! : string ** nat -> char
\end{prog}
as declared in the library structure \pro{StringIndex}  takes as
arguments a string and a natural number and delivers a character.

Remember that graphical symbols are allowed as identifiers and that
the space between ``\pro{!}'' and  ``\pro{:}'' is very
important (see Section \ref{sec:ide} if you'd  forgotten about that).

\medskip
As you could see above, the functionality is always expressed in terms
of sorts. 
These sorts must be known in the structure, i.e.~they must be either
 imported or declared in the structure.

\advanced  Arguments and results of a function could be much
more general than explained above.

Thus the result of a function could also be a tuple. 
This might be useful, e.g., in a function
\begin{prog}
        FUN divmod : nat ** nat -> nat ** nat
\end{prog}
which returns the quotient and the remainder of a division
simultaneously (on how to select the elements of a tupled result, see
Section \ref{subsec:objdecl}: ``Object Declarations'').

\medskip
\experienced Theoretically, the number of arguments is unlimited; in the current
implementation it is restricted to 16. 
This is not a serious restriction, because  16 is
quite a large amount.
IF you really do need more parameters, you can combine several
arguments into a new data type to reduce the number of arguments. 

\subsubsection{Higher-Order Functions}
\label{sec:higherOrderFunctions}

\advanced You can have functions as arguments and results too.
These functions are called higher-order functions.
Usually higher-order functions are supported only very rudimentarily
in traditional languages, if at all.

 A function to compute the integral may be declared like this:
\begin{prog}
        FUN integral : (real -> real) ** real ** real -> real
\end{prog}
which means that the first argument is the function to be
integrated and the second and third argument are real numbers defining
the lower and upper bound of the integral.

Another example can be found in the program \pro{Expression}:
\begin{prog}
        FUN doDyop : dyadicOp -> nat ** nat -> nat
\end{prog}
The function \pro{doDyop} takes an element of the sort \pro{dyadicOp}
and delivers a function which itself takes two numbers as arguments and
delivers one number as result.

The symbol \pro{->} is right-associative. This means the declaration
above is equivalent to 
\begin{prog}
        FUN doDyop : dyadicOp -> (nat ** nat -> nat)
\end{prog}
and the parentheses may be omitted.

This process of functions as arguments or results could be continued.
There is no limit to the ``nesting depth''.


\subsubsection{Currying}
\label{sec:currying}
\advanced
As known from theoretical computer science and mathematics, it is
always possible to transform a function which takes more than one
argument into a function which takes only one argument and delivers a
function as result, without altering the semantics of the function.
This process is called {\em currying\/}.
\begin{prog}
        FUN + : nat ** nat -> nat
        FUN + : nat -> nat -> nat
\end{prog}
The second  variant is the curried version of the first.

In some functional languages this transformation is done automatically and the
two declarations are recognized as identical.

In \opal\ the two declarations are distinguished and need their own
definitions. 
Each variant has its own advantages:
with the first the function \pro{+} could be used as
infix-operator, whereas with the second one you can define a function,
e.g.~\pro{+(3)}, a brand  new function which will add three to
its only remaining parameter.
This can be useful, for example in conjunction with other higher-order
functions, such as  apply-to-all on sequences.

On the other hand, if you have used the uncurried version, you can use
sections and lambda expressions (see 
\ref{sec:sections} and \ref{subsec:lambda}) to do a
``currying on the fly'', i.e.~to
define a temporary, auxiliary function with partially supplied
parameters.
It is therefore simply  a matter of taste, whether you prefer the
curried or the uncurried version.

\subsection{Definition of Functions}
\label{sec:fun.def}
\novice
The definition of a function defines how the function works, i.e.~it
represents the real algorithm.

The definition of a function consists of a header on the left side of
the ``\pro{==}'' and a body on the right side. The function
\pro{rabbits} will be defined by
\begin{prog}
        DEF rabbits (n) == {\em \verb+<<+ body \verb+>>+\/}
\end{prog}

The number of parameters (``\pro{n}'') must match the number of
parameters given in the declaration of the function (in this example
one argument). 
In the body the name \pro{n} will be visible (see next section). 
And due to the declaration of \pro{rabbits} it will stand for an
object of type \pro{nat}.
The parameters are used to reference  the arguments of a concrete call
of \pro{rabbits} in the body. 

The body of a function definition is an expression. We will explain
expressions in Section \ref{sec:expr.simple}.

The headers of the other examples from the previous section
(``Declaration of Functions'') will be something like
\begin{prog}
        DEF main == {\em \verb+<<+ body \verb+>>+\/}
        DEF !(str, n) == {\em \verb+<<+ body \verb+>>+\/}
\end{prog}

They define \pro{main} to have no parameters and \pro{!} to have two
parameters, named \pro{str} and \pro{n}, which can be used in the body
of the definition.
\medskip

The functionality%
\footnote{Functionalities are required for checking
  the correctness of expressions; see Section \ref{sec:expr.simple}
  below.}
of the parameters can be derived from the declaration of the
corresponding function.
Function ``\pro{!}'' is declared as \pro{FUN ! :string**nat->char}.
Therefore the first parameter, \pro{str}, has functionality \pro{string}
and the second, \pro{n}, has functionality \pro{nat}.
\medskip

\advanced There is a bunch of other ways to notate the header
of a function definition. First of all you can use infix-notation
which is quite similar to infix expression \ref{subsec:func.appl}. 
You could  also write
\begin{prog}
          DEF str ! n == {\em \verb+<<+ body \verb+>>+\/}
\end{prog}
\noindent instead of the definition above.

Furthermore, you can use pattern-matching to define functions. Pattern
matching depends on free types and will be explained in Section
\ref{sec:patternType}: ``Pattern-Matching''.

\medskip
Higher-order functions with functions as arguments are defined the
same way as first-order functions. 
The function \pro{integral} (see above) could be defined as
\begin{prog}
        DEF integral (f, low, high) == {\em \verb+<<+ body \verb+>>+\/}
\end{prog}
where \pro{f} denotes the function to be integrated.

The functionality of the parameters is naturally extended: \pro{f} has
functionality \pro{real->real}, \pro{low} and \pro{high} have
\pro{real} respectively.

Higher-order functions with functions as results will be defined with
additional parameters for the parameters of the result function.
The function \pro{FUN doDyop : dyadicOp -> nat ** nat -> nat} could be
defined as 
\begin{prog}
        DEF doDyop (op)(l, r) == {\em \verb+<<+ body \verb+>>+\/}
\end{prog}
In this case \pro{op} is a dyadic operand and \pro{l} and \pro{r} are
the natural numbers as arguments for the resulting function.

More concretely, this means you define this function as 
\begin{prog}
        DEF doDyop (op)(l, r) == IF op addOp? THEN +(l,r)
                                 \dots
\end{prog}
The functionality of the parameters again are naturally extended: \pro{op}
has functionality \pro{dyadicOp} and \pro{l} and \pro{r} have \pro{nat}.

But you can even shorten this header.
 If you want to define a function \pro{myadd : nat ** nat -> nat} with the
same semantics as the standard definition of addition (i.e. renaming
the function \pro{+}) you can do this by  writing:
\begin{prog}
        DEF myadd (a,b) == +(a,b)
\end{prog}
It is not true however, that for each parameter in the declaration there
must be a corresponding parameter in the definition.
As long as the type remains correct you can omit the parameters.
\begin{prog}
        DEF myadd == +
\end{prog}
On the right side there is only one identifier with the functionality
\pro{nat**nat->nat}, and the left side has the same. 
So this definition is correct. 

 Omission of parameters can only be done on whole tuples.
This means you cannot leave out \pro{b} alone. You must write
either all or none of the parameters of a tuple.
And---of course---you can only omit trailing tuples.

\medskip
This scheme is especially useful with higher-order functions. 
You can shorten the definition of the function \pro{ doDyop} to  
\begin{prog}
        DEF doDyop (op) == IF op addOp? THEN +
                           \dots 
\end{prog}
where the function yields---depending on the operation \pro{op}---just
one of the well-known functions \pro{+, -, *, /}.

\experienced
Omitting parameters entails one small catch.
In \opal\ all constants are evaluated only once during the initialization
phase of the program. 
Constants are  function {\em definitions\/} without parameters. 
Therefore the second definition of \pro{myadd} (see above) is a
constant definition (it does not depend on arguments), whereas the
first is a function definition (it has two arguments).

In general this won't make any difference and you can use the two as
you please.
 But you should note, that the definition of constants
must be acyclic, i.e.~you can't use a constant in its own definition,
neither direct nor transitive.

Moreover there are rare cases where the time of evaluation of
functions and arguments might be significant.
Sometimes it is even useful to add an empty argument tuple to delay
the evaluation of a function call until later:
\begin{prog}
        FUN f : nat ** nat -> () -> res
\end{prog}
In this case the function call \pro{LET g == f(1,2) IN \dots} won't be
evaluated, but yields a closure (i.e.~a new function) which could be
submitted  as an argument or stored in a data structure.

Only if this closure is applied with the empty tuple (e.g.~\pro{g()}), then
the call  \pro{f(1,2)} is evaluated.
Using this method you can simulate lazy execution within the strict
language \opal.


\section{Scoping and Overloading}
\label{sec:scope}
\novice
Scoping and overloading generally involve very complex rules.
Scoping means: I have declared an object somewhere in  the
program text and want to know if this object is known (accessible)
somewhere else.
If the object is accessible, we say it is {\em visible\/} at this
location.

Overloading means using the same identifier for different objects.

Concerning scoping and overloading in \opal\ you have to  distinguish
two different kinds of objects
\begin{itemize}
\item global objects
\item and local objects.
\end{itemize}
Global objects are all imported objects as well as all functions and sorts
declared in the structure, either in the signature or in the
implementation part. 

Global objects may have the same identifier as long as they can still be
distinguished by annotations (see Section \ref{sec:name} for details).
This means that if two global objects differ in at least one of their
identifiers, their origins, their instantiations  or their kinds (sort or
functionality) you can use both objects side by side.

If they are the same in identifier and all possible annotations, the
compiler will perceive them as identical.
This means it is possible, for example, to import the same object several times.

Local objects are parameters, lambda- and let-bound variables (see
Section~\ref{sec:expr.simple}).
They cannot be annotated with origin or instantiation.

\important The names of local objects must be disjoint within their visibility
region, i.e.~they cannot be overloaded at all.

\medskip
\novice The rules for scoping (i.e.~visibility) are fairly simple in \opal.
A signature part cannot have local objects.
In the signature part the only  visible objects are those which are 
imported or declared in the signature part.
\medskip

Throughout  the implementation part, all global objects from the signature part
and the implementation part are visible.

\medskip
Parameters of function definitions are visible throughout the function
definition; lambda- and let-bound variables are visible only within
their expressions. 

\important If a local and a
global object have the same identifier, the global object will be invisible
as long as the local object is visible, {\em even if the identifier is
  annotated\/}. In this case the annotation on local objects is ignored!
This does not apply to sort-names, because these can be identified by their
position in the program text.

\section{Expressions}
\label{sec:expr.simple}
\novice 
In this  section we will explain how to construct
the body of a function definition.
First we introduce the fundamental expressions essential even for
trivial programs; these are atomic expressions, tupling of expressions,
function applications and case distinctions. 
Then we will discuss the more elaborate features which improve the
power of  functional programming and the readability of \opal\ 
programs: lambda abstractions, sections and local declarations. 

\medskip
A basic issue regarding correctness of expressions is the
functionality of an expression.
The main context condition for expressions demands that
functionalities must fit together.
Therefore we will also discuss the functionality of each expression
and the conditions it must satisfy.


\subsection{Atomic Expressions}
\label{subsec:atomic}
\novice
The most simple expressions are atomic expressions. 
There are two kinds of atomic expressions: identifiers and
denotations.

Identifiers denote objects (i.e.~functions, local objects  and
elements of data types) which have to be visible\footnote{see
  explanation of visibility above.} when the identifier
is used. 

Examples are \pro{rabbits}, \pro{0}, \pro{=}, \pro{+} and
\pro{generation} (all are taken from the example program
\pro{rabbits}). 
In the case of a global identifier the functionality is
declared in the identifiers declaration; the
functionality of a local identifier can  be derived either from the declaration of the
corresponding function (for parameters; see Section
\ref{sec:fun.def}) or from the context (for local declarations and
lambda-bound variables; see Sections \ref{subsec:objdecl} and \ref{subsec:lambda} for details).

\medskip
Denotations are special notations for denoting arbitrary objects.
They are enclosed in quotation marks and are often used to
represent text, e.g. ``\pro{Hello World}'', but you can write conversion
routines to represent just about every data object using denotations.
The corresponding conversion functions for natural numbers, integers and reals are
predefined in the standard library.

These conversion functions are usually named ``\pro{!}''.
For example \pro{"1234"!'Nat} represents the natural number 1234
(remember,  \pro{'Nat} is an annotation defining the origin of
``\pro{!}'' being the structure \pro{Nat}).
More about denotations can be found in Section \ref{denotations}.

Denotations always have functionality ``\pro{denotation}''.

\subsection{Tuples}
\label{sec:tuples}
\novice
Expressions (any expression, not only atomic ones) can be grouped
together as tuples by enclosing them in parentheses:
\begin{prog}
        ("Hello World", 1, +(5,3))
\end{prog}

\noindent is a tuple with three elements, a denotation, a number and a function
application.
Tuples are used mainly to unite the arguments of a function
application, e.g. in \pro{+(3, 5)}.

Tuples are always flat in \opal.
That is, you can write a nested tuple like \pro{(a, b, (c, d), e)} which
seems to consist of four elements, the third being a tuple itself.
But in \opal\ this is identical with the tuple \pro{(a, b, c, d, e)}.

A tuple may contain only one element while the empty tuple is allowed
only as an argument of function calls.

\advanced  Flat tuples allow some compact notations on using
functions which return tuples as results. 
Suppose you have a function \pro{f:nat**nat**nat->nat} 
which takes three arguments.
Remember the function \pro{divmod:nat**nat->nat**nat} which
returns a tuple of two numbers.

You can write \pro{f(3, divmod(22,5))} which is the same as
\pro{f(3, (4, 2))} (the function call \pro{divmod(22,5)} being
evaluated and---because tuples are flat---this is a correct call of 
\pro{f(3, 4, 2)}). 


\subsection{Function Applications}
\label{subsec:func.appl}
\novice
A function application consists of two expressions, the second being
a tuple, e.g.~\pro{+(3,4)} or \pro{rabbits(generation-1)}.
The tuple consists of the arguments of the functions call.

The functionality of the first expression must be a function
functionality consisting of parameters and a result.
The functionality of the tuple must match the functionality parameters.
The functionality of the whole function application is the result.

In the example above the first expression is ``\pro{+}'', which has a 
functionality of \pro{nat**nat->nat}.
The second expression is the tuple ``\pro{(3,4)}'' with
functionality \pro{nat**nat}, which matches the parameters of
``\pro{+}''.
The functionality of the application is \pro{nat}, the result part of the
functions functionality.

\advanced There are several variations on the notation of function
applications.
Each function application could also be written as a postfix-operation,
i.e. just exchange the two expressions: \pro{(3,4)+}.

In this case the parentheses around a tuple containing only one
element may be omitted.
Postfix-operations are often used to shorten the notation for
conversions like \pro{"Hello World"!} which is the same as
\pro{!("Hello World")}.

Postfix-operations are also known from mathematics;for example the
factorial is usually written as $3!$.

\medskip
Function applications may also be written as infix-operations:
\pro{(3)+(4)}.
Again you can omit the parentheses if the tuples only contain one
element. This results in the usual notation for mathematical
expressions: \pro{3+4}.

In general graphical symbols like \pro{+}, \pro{*}, \pro{!}, \pro{::}
or \pro{+\%} are used as function names if the function is intended to
be used as an infix- or postfix-operation, but this is not necessary.

The definition and the application of a function are independent.
You can define a function using pattern-based definitions and
infix-notation and you can still apply the function in traditional
prefix style.

\medskip
Infix-notation could also be used with more than two arguments:
\begin{prog}
     f(a,b,c); f(a, (b,c)); a f (b,c); (a) f (b,c); (a,b) f c; (a,b,c)f 
\end{prog}
are all the same function application.
\bigskip

You could also nest infix-notation, but there is  one little problem
with this very flexible notation. 
Because the analysis of arbitrary infix notations is very expensive,
you should use at most 5--6 infix operations without bracketing. 
If your expression is larger, you should enclose subexpressions in
parentheses.
This will  also enhance readability.

This problem seems ridiculous, especially as even traditional
imperative languages can deal with any number of infix-operations.
But remember a ``\pro{+}'' in \opal\ could be any function, including a
user-defined function or a function computed at run-time (in contrast to
traditional languages), and this much more general problem has not
been resolved satisfactorily to date.
\medskip

\advanced There is one more problem with arbitrary infix-notation. 
As the compiler cannot have any knowledge about precedence rules
between the different functions (in contrast to C, for example), 
it can use only
the various functionalities to detect  which applications were
intended by the programmer. 

A typical example is the construction of a list (using
\pro{FUN~::~:data**seq->seq}): 
\begin{prog}
          1 :: 2 :: 3 :: 4 :: <>
\end{prog}
\noindent Due to the functionalities, the compiler will recognize this as:
\begin{prog}
          1 :: (2 :: (3 :: (4 :: <> )))
\end{prog}

If the compiler cannot resolve this problem with a unique solution,
you will receive an error message like \pro{ambiguous infix
  application} and you should insert some parentheses to
help the compiler.

\important If you are using the same function several times in nested
infix-notation, the compiler will assume right-associativity.

\noindent The concatenation of several denotations
\begin{prog}
        "ab" ++ "cd" ++ "ef" ++ "gh"
\end{prog}
will be recognized as 
\begin{prog}
        "ab" ++ ("cd" ++ ("ef" ++ "gh"))
\end{prog}
{\bf WARNING:} This is contra-intuitive with respect to normal mathematics!
The expression \pro{20-10-5}
yields 15 instead of 5 as expected because the compiler assumes
right-associative functions and ``\pro{-}'' is not right associative!
In this case you must supply parenthesis: \pro{(20-10)-5}.
\medskip

\advanced
The application of higher-order functions is just the same as for
first-order functions.
Remember the functions
\begin{prog}
        FUN integral:(real->real) ** real ** real -> real
\end{prog}
and
\begin{prog}
        FUN doDyop: dyadicOp -> nat ** nat -> nat
\end{prog}

\pro{integral} will be applied as
\begin{prog}
        integral(sin, 0, 1)
\end{prog}
 for example to compute  the integral over the sinus function between 0 and
1.
Remember, you also can write this application as infix or postfix, e.g.
\begin{prog}
        sin integral (1, 2)
\end{prog}

The function \pro{doDyop} expects one argument of functionality
\pro{dyadicOp}. A correct application will be 
\begin{prog}
        doDyop(addOp)
\end{prog}
You could also write it as postfix\footnote{This time you cannot use
  infix-notation, because there is only one argument.}:
\begin{prog}
        addOp doDyop
\end{prog}

The result of this application is a function with functionality
\pro{nat**nat->nat}.
Thus you can apply this function to two numbers:
\begin{prog}
        doDyop(addOp)(3,4)
\end{prog}
Function application associates to the left, i.e.~the expression
above is the same as 
\begin{prog}
        (doDyop(addOp))(3,4)
\end{prog}

This application could also be written as
\begin{prog}
        (addOp doDyop)(3,4)
\end{prog}
for example.

But note that a function used as an infix-operator
must be an identifier. 
Therefore you cannot write
\begin{prog}
        3 (addOp doDyop) 4     -- this is illegal
\end{prog}
because  \pro{(addOp doDyop)} is no identifier.

\medskip

This example also illustrates another feature. 
The first expression (i.e., the function of the function application)
need not be a name. 
It can be any expression including another function
application (as above), a case distinction, a lambda abstraction or an
object declaration.
The only context condition is that it must have a functions
functionality.

\subsection{Case Distinctions}
\label{subsec:cases}
\novice Case distinctions are used to control the evaluation of the program.
Depending on the value of conditions, the program will continue to
evaluate different parts.

A case distinction in \opal\ is written as  
\begin{prog}
        IF cond_1 THEN part_1
        IF cond_2 THEN part_2
        ...
        IF cond_n THEN part_n
        FI
\end{prog}
\pro{cond\_1} \dots \pro{cond\_n} and \pro{part\_1} \dots \pro{part\_n}
are arbitrary expressions including function applications, other case
distinctions or local declarations.

\pro{cond\_1} \dots \pro{cond\_n} are called guards and all of them must
have functionality \pro{bool}.
\pro{part\_1} \dots \pro{part\_n} must have identical functionalities
and this common functionality is also the functionality of the case
distinction. 

Let us have a concrete example. The function \pro{rabbits} is defined
as:
\begin{prog}
          DEF rabbits(generation) ==
              IF generation = 0 THEN 1
              IF generation = 1 THEN 1
              IF generation > 1 THEN rabbits(generation - 1) 
                                     + rabbits(generation - 2)
              FI  
\end{prog}
The guards are infix function applications, each of functionality
\pro{bool}.
Each of the THEN-parts has functionality \pro{nat} which is the
functionality of the case distinction and also of the result of the
function.

The semantics of a case distinction is:      
evaluate one of the guards. 
If the guard yields false, forget about this case and try
another one.
As soon as a guard yields true, stop searching and evaluate the
corresponding THEN-part.

If none of the guards yields true, the value of the case distinction is
undefined and the program will terminate with an error message.

\important  The order the guards are evaluated in is selected by the
compiler! It does not depend on the order in which the guards are
denoted in the program!

\novice The example above may be evaluated as follows (supposing
\pro{n} has the value~\pro{0}):
first the program might check the last guard. \pro{0>1} yields
\pro{false}, therefore the next guard will be checked;  let us now
assume it to be the  first guard.
\pro{0=0} yields \pro{true}.  The corresponding THEN-part
will therefore be evaluated, yielding \pro{1} and the value of the case
distinction is computed as \pro{1}.

\medskip
If the guards are not disjoint there is no specification as to  which THEN-part
will be evaluated.
Sometimes this is desirable, as in
\begin{prog}
        DEF maximum(n, m) == 
                 IF n <= m THEN m
                 IF m <= n THEN n FI
\end{prog}
In this case the program will always check only one guard if the
values of \pro{n} and \pro{m} are the same.

But you should ensure that the result is the same in any possible
case. Otherwise your program might behave unexpectedly.

\medskip
Remember that \opal\ is a strict language and the guards are just
expressions.
If you have a condition consisting of two parts and the second is
valid only if the first yields \pro{true}, you cannot combine the two
parts with an \pro{and}-function!

There is a well-known example for this fault:
given a list of numbers, you want to compare the first element of the
list with some value.
The first part of the condition checks if the list is not empty
(because otherwise you can't access the first element) and the second
part does the comparison.

It is wrong to write
\begin{prog}
     IF (~(list empty?) and (ft(list) = 0)) THEN ... -- this is wrong
     ...
     FI
\end{prog}
because {\em both\/} arguments of the \pro{and} will always be
evaluated.
This results in a runtime-error if the list is empty.

Instead you have to split the condition into two guards in a nested
case distinction:
\begin{prog}
        IF ~(list empty?) THEN
                    IF  ft(list) = 0 THEN ...
                    ...
                    FI
        ...
        FI
\end{prog}
This could be abbreviated to the short-hand notation
\begin{prog}
     IF (~(list empty?) ANDIF (ft(list) = 0)) THEN ...-- this is correct
     ...
     FI
\end{prog}
There is also a corresponding \pro{ORIF}.
Both operations evaluate their first argument.
 Only if this evaluation yields \pro{true} for
\pro{ANDIF} or ~\pro{false} for \pro{ORIF}, the second argument will
be evaluated too.
\bigskip

\advanced There are two notational variations for case distinctions.
Firstly, you can add an ``\pro{OTHERWISE}'' between any two cases.
Then the program will first check all guards in front of the
\pro{OTHERWISE}, and 
only if all of these guards yield \pro{false} will it continue with
the guards following the \pro{OTHERWISE}.
\smallskip

\noindent The expression 
\begin{prog}
        IF cond_1 THEN part_1
        ...
        IF cond_n THEN part_n 
        OTHERWISE 
        IF cond_n+1 THEN part_n+1
        ...
        IF cond_n+m THEN part_n+m 
        FI 
\end{prog}
is exactly the same as 
\begin{prog}
        IF cond_1 THEN part_1
        ...
        IF cond_n THEN part_n 
        IF ~(cond_1 or ... or cond_n) 
           THEN 
              IF cond_n+1 THEN part_n+1
              ...
              IF cond_n+m THEN part_n+m
              FI
        FI  
\end{prog}

Secondly, you can add an ``\pro{ELSE expr}'' after the last case of a
case distinction. 
If the evaluation of all guards yields \pro{false}, the ELSE-expression
will be evaluated instead of the program terminating with an error.

\smallskip
\noindent The case distinction
\begin{prog}
        IF cond_1 THEN part_1
        ...
        IF cond_n THEN part_n 
        ELSE expr
        FI
\end{prog}
is the equivalent of 
\begin{prog}
        IF cond_1 THEN part_1
        ...
        IF cond_n THEN part_n 
        OTHERWISE
        IF true THEN expr
        FI
\end{prog}
\bigskip

\advanced In addition to the fundamental expressions  described above,
\opal\ offers three more constructs for defining expressions.
These are  lambda abstractions, sections and local declarations.
Lambda abstractions and local declarations also enlarge the number of
local objects by new objects  
which are visible inside their expressions (see below). 

\subsection{Lambda Abstraction}
\label{subsec:lambda}
\advanced
Lambda abstractions define auxiliary functions without a name.
The notation is
\begin{prog}
        \LAMBDA{}x,y. expr 
\end{prog}

This lambda expression defines a function which takes two arguments.
The parameters are called \pro{x} and \pro{y}.
These new local objects are visible only inside \pro{expr}.
The functionality of \pro{x} and \pro{y} could either be annotated or it
will be derived automatically from their usage in \pro{expr}.

\smallskip\noindent Let us have a concrete example: 
\begin{prog}
        \LAMBDA{}x.x=3
\end{prog}
defines a function with functionality \pro{nat -> bool}, which compares
a number to 3.

You can apply this function: for example \pro{(\LAMBDA x.x=3)(4)} yields
\pro{false}.
Or you can use it in an object declaration:
\begin{prog}
        LET equalThree == \LAMBDA{}x.x=3
        IN ...
\end{prog}
This way you receive a named auxiliary function, although
lambda abstractions cannot be recursive.

Very often  lambda abstractions are used to  submit an auxiliary
function to  higher-order functions as arguments, e.g. 
 \begin{prog}
         LET a == pi
             b == e
             c == "-5.0"!
         IN integral(\LAMBDA{}x.a*x*x + b*x + c, 0, 1)\end{prog}
computes the integral $\int_{0}^{1}a x^2 + b x + c$ with $a=\pi$,
$b=e$ and $c=-5$.
\bigskip

Lambda abstractions may also be nested to define higher-order
functions and they can be used to define ordinary named functions:
The definitions
\begin{prog}
        DEF f(a,b)(c) == expr
\end{prog}
and
\begin{prog}
        DEF f == \LAMBDA{}a,b. \LAMBDA{}c. expr
\end{prog}
are equivalent and each of them defines a function with functionality, \\
e.g.~\pro{FUN f:s1**s2->s3->s4}.

As in  pattern-matching (see Section \ref{sec:patternType}), an
underscore character may be used as a wildcard in a lambda
abstraction;  the meaning is that there is a parameter for a
(lambda-defined) function, 
but I am not interested in its value.


\subsection{Sections}
\label{sec:sections}
\advanced Sections are a shorthand notation for simple lambda
abstractions.
You will often  need a function where some arguments should not yet be
fixed.

As an example you may think of a function that adds the value of 3 to
each element of a sequence of numbers \pro{s}.
Using the apply-to-all-function ``\pro{*}'', this could be written with lambda
abstraction as
\begin{prog}
        * (\LAMBDA x. 3+x)(s)
\end{prog}
or in shorthand form with section as
\begin{prog}
        * (3 + _) (s)
\end{prog}
Note: don't forget the separator between ``\pro{+}'' and ``\pro{\_}''.
Otherwise you apply the (probably undefined) function ``\pro{+\_}'' on the
argument 3.

Underscores represent arguments of a function call which are
missing at the moment, but will be supplied later.

Sections could be regarded as a generalization of currying, because not
only trailing (as with currying) but also arbitrary arguments can be
postponed until later.
 
In detail, the expressions
\begin{prog}
        f(a,b,c,d,e);    f(a,b,\_,d,\_)(c,e);    (f(a,\_,\_,d,\_))(b,\_,e)(c)
\end{prog}
are all the same.


\subsection{Local Declarations}
\label{subsec:objdecl}
\advanced
Local declarations can be used to structure the definition of a
single function and to introduce abbreviations.

\smallskip\noindent
Local declarations are written as
\begin{prog}
        LET o_1 == expr_1
            o_2 == expr_2
            ...
            o_n == expr_n
        IN expr
\end{prog}
or
\begin{prog}
        expr WHERE o_1 == expr_1
                   o_2 == expr_2
                   ...
                   o_n == expr_n
\end{prog}
The two notations are equivalent.

The additional local objects \pro{o\_1} \dots \pro{o\_n} are visible in
the expression \pro{expr} and in all expressions \pro{expr\_1} \dots
\pro{expr\_n}.

A declaration \pro{o\_i == expr\_i} is said to depend on the
declaration \pro{o\_j == expr\_j} if \pro{o\_j} is used in \pro{expr\_i}.
The ordering of the declarations is irrelevant, but there must  not be
cyclic dependencies between the declarations.

\smallskip\noindent The expression \pro{expr} will be as long as
possible; in the local declaration
\begin{prog}
        LET a == ...
        IN f(a)(e_2)
\end{prog}
for example, the \pro{a} will be visible in the whole expression
\pro{f(a)(e\_2)}, not only in the expression \pro{f}.

The semantics of local declarations could be explained with lambda
abstraction. Assuming the declaration \pro{o\_1 == expr\_1} does not
depend on any of \pro{o\_2} to  \pro{o\_n},  then the local declaration
\begin{prog}
        LET o_1 == expr_1
            o_2 == expr_2
            ...
            o_n == expr_n
        IN expr
\end{prog}
is equivalent to 
\begin{prog}
        (\LAMBDA o_1. LET o_2 == expr_2
                 ...
                 o_n == expr_n
                 IN expr           ) (expr_1)
\end{prog}
The declaration that does not depend on any other must not necessarily be
the first one, because ordering of declarations is irrelevant.
But because the dependency between the declarations must be acyclic,
there is always a declaration, that does not depend on any other.

\important Note that local declarations are strict. Therefore a
declaration like 
\begin{prog}
        LET cond     == ...           -- dangerous pgm style
            thenpart == ...
            elsepart == ...
        IN IF cond THEN thenpart ELSE elsepart FI
\end{prog}

\noindent will not behave as expected, because \pro{thenpart} and
\pro{elsepart} are {\em always\/}  evaluated  due to the strict semantics
of local declarations.


% Local Variables: 
% mode: latex
% TeX-master: "tutorial"
% End: 

% LAST EDIT: Tue May  3 13:16:41 1994 by Juergen Exner (hektor!jue) 
% LAST EDIT: Fri Apr 29 16:01:13 1994 by Juergen Exner (hektor!jue) 
% LAST EDIT: Tue Feb 15 12:10:10 1994 by Juergen Exner (hektor!jue) 
% LAST EDIT: Mon Feb 14 16:01:55 1994 by Juergen Exner (hektor!jue) 
% LAST EDIT: Tue Jan 12 16:35:37 1993 by Juergen Exner (hektor!jue) 
\chapter{Input/Output in \opal}
\label{chap:IO}

\advanced
The problem of interactive input/output (I/O) in functional languages
is still a research topic.
\opal\ uses continuations to support I/O.
We won't discuss the theory of continuations here as we prefer to take a more
intuitive approach.


The fundamental issue of I/O in \opal\ is the {\em command\/}.
A command is a data object that describes an interaction with the environment,
e.g.~the terminal or the file system. Only commands do this.
The command itself is executed by the runtime-system, which
evaluates the commands at runtime according to the rules described in
this chapter.

Commands can be determinated by their functionality: a command is
always a constant of type \pro{com'Com}.  Functions which yield an object
of type \pro{com`Com} as a result are used very frequently. They can
be regarded as parameterized commands.

\section{Output}
\label{sec:output}
\advanced
We have already seen some examples of commands.
In the example program, \pro{HelloWorld} (see Chapter \ref{chap:example}), the
function \pro{write} is a predefined function from the library
structure \pro{Stream} with functionality
\begin{prog}
        FUN write : output ** denotation -> com
\end{prog}

The whole program itself interacts with the environment too.
Therefore it is also a command (the so-called top-level command which
is required for every program):
\begin{prog}
        FUN hello : com
\end{prog}

\medskip

Commands can be combined to form new commands with the functions
``\pro{;}'' and ``\pro{\&}'' from the library structure \pro{ComCompose}.
These functions are usually written as infix operations.
Both combination functions take two commands as arguments and combine
them to a new one by first executing the first command and afterwards
the second.
To combine some write-commands you could write
\begin{prog}
        writeLine(stdOut, "This will be the first line") \&
        writeLine(stdOut, "Some more output") \&
        writeLine(stdOut, "The last line")
\end{prog}
which will print the three lines to standard output, i.e.~usually the terminal.

We have already used command combination in the
\pro{Rabbits} example (see Chapter~\ref{chap:example}):
\begin{prog}
          write(stdOut,
                  "For which generation do you want to know ...? "!)\&
          (readLine(stdIn) \& 
          processInput)
\end{prog}
\medskip

Since commands describe interactions with the environment there is no
certainty  that the  evaluation of a
command will always succeed (for example a requested file can't be accessed or
the connection to an internet socket has been broken).
Therefore there must be some error-handling.

The functions ``\pro{;}'' and ``\pro{\&}'' differ in their handling of
errors. 
If an error occurs during the execution of the first command,  ``\pro{\&}''
does not execute the second command, whereas
``\pro{;}'' will.
In this case the programmer must check within the
second command whether an error has occurred during the execution of the first
 and continue with an appropriate alternative (e.g.~asking the user
for another filename if a file could not be opened).
How to access and analyze a possible error message will be explained
later on.

Note that ``\pro{;}'' and ``\pro{\&}'' are strict. 
The second argument will always be computed, but the resulting command
will be executed only if execution of the first  succeeds. 
If the execution of the first command fails, the corresponding error
is yielded. 


\section{Input}
\label{sec:input}
\advanced
Up to now we have only dealt with commands which produced some
output and combinations of them.
But for real computations we will need some input too.

For this reason commands are parameterized (see Section
\ref{sec:struct.param}) and can be instantiated with the sort they
should yield as result of an input operation.
The command \pro{readLine'Stream(stdIn)}, for instance, reads a string from
the terminal and delivers this string as  result:
\begin{prog}
        FUN readLine : input -> com[string]
\end{prog}

The result cannot be accessed directly, as \pro{readLine} yields a
command, not a string. And a command doesn't have a
``function result''.

To access the desired string the next command to be executed has to be
a parameterized command in the sense that is it has to
be a function that expects a string as argument, e.g.
\begin{prog}
        FUN foo : string -> com
\end{prog}

The  command \pro{readLine(stdIn)} and the function \pro{foo} can then be
combined with a variant of the ``\pro{\&}''-function:
\begin{prog}
        readLine(stdIn) \& foo
\end{prog}
This variant of the  ``\pro{\&}''-function passes the result of the
first command (the string read from standard input) as argument to the
function \pro{foo}, yielding a new command. 
The argument string itself can be accessed in \pro{foo} like any
other parameter of a function definition.
\medskip

Very often the second command will be written as a lambda-abstraction. 
For example, a command \pro{echo} which echoes the input to the output
can be written as
\begin{prog}
        FUN echo : () -> com[void]
        DEF echo () == readLine(stdIn)      \& (\LAMBDA x. 
                       writeLine(stdOut, x) \&
                       echo() ))
\end{prog}
or---if you don't like never terminating programs---as
\begin{prog}
        DEF echo() == 
              readLine(stdIn)               \& (\LAMBDA x. 
                IF x =  empty THEN writeLine(stdOut, "End of Program")
                IF x |= empty THEN writeLine(stdOut, x)  \& 
                                   echo() )
                FI
\end{prog}

The unusual functionality of \pro{echo} results from the restriction
that constants cannot be cyclic, i.e.~the definition of a constant
cannot be recursive.
Therefore \pro{echo} must be defined as a function with an empty
argument.


\section{Error-Handling}
\label{sec:ioerror}
\advanced While ``\pro{\&}'' does some error-handling automatically, it is
also possible to do error-handling by yourself with the corresponding
 function ``\pro{;}''.  When using the function
``\pro{;}'' instead of ``\pro{\&}'' the second command does not
receive the required value directly, but an answer that contains the
value if the command has been successfully executed.
 If the execution has failed the answer contains an error message.

The data type answer is declared as a parameterized data type in the library
structure \pro{Com[data]} as
\begin{prog}
     TYPE ans == okay(data:data)    -- result of successful command
                 fail(error:string) -- diagnostics of failing command
\end{prog}

\noindent In the example \pro{foo} the functionality has to be modified to 
\begin{prog}
        FUN foo : ans[string] -> com
\end{prog}
if you want to do error-checking by yourself.

\medskip
\noindent In the example \pro{echo} error-checking could be done like
\begin{prog}
        DEF echo ()== 
              readLine(stdIn) \&
                (\LAMBDA x.
                   IF x okay? THEN 
                      IF data(x) =  empty THEN 
                                writeLine(stdOut, "End of Program")
                      IF data(x) |= empty THEN 
                                writeLine(stdOut, data(x))
                                  \& echo () FI
                   IF x error? THEN 
                      writeLine(stdOut, "Some error has occurred")
                   FI
\end{prog}
Note that ``\pro{x}'' now has type \pro{ans}, whereas in the first
example ``\pro{x}'' has type \pro{string}.
Therefore  you have to use the selector \pro{data} to access the
desired string from the answer yielded by the command.

\medskip
This variant of command combination is also the proposed method for
supervising  commands that only produce output. 
Commands which don't do input (i.e.~which don't forward a value to
subsequent commands) have functionality \pro{com[void]}.
This means they always construct an answer of type \pro{ans} too, but
the data-item will be \pro{void}, i.e.~useless.
But you can check this answer to determine if an error has occurred or
not.

As an example, after executing a write-command you can check if this
command was successful as follows:
\begin{prog}
  writeLine(stdOut, "This will be the first line"!) ;   (\LAMBDA x.
        IF x okay? THEN <<<\mbox{everything all right}>>>
        IF x error? THEN <<<\mbox{some error occurred}>>>
        FI )
\end{prog}

The simple version of ``\pro{;}'' (as introduced at the beginning of
this chapter),
which combines two commands, simply ignores a failure of the first
command.
There is no way to check against failure, unless you use the second
version of  ``\pro{;}''.
% Local Variables: 
% mode: latex
% TeX-master: "tutorial"
% End: 

% LAST EDIT: Wed May  4 11:41:17 1994 by Juergen Exner (hektor!jue) 
% LAST EDIT: Tue May  3 13:19:15 1994 by Juergen Exner (hektor!jue) 
% LAST EDIT: Fri Apr 29 17:09:21 1994 by Juergen Exner (hektor!jue) 
% LAST EDIT: Tue Feb 15 14:20:00 1994 by Juergen Exner (hektor!jue) 
% LAST EDIT: Mon Feb 14 17:53:48 1994 by Juergen Exner (hektor!jue) 
\chapter{Types in \opal}
\label{chap:types}

{\bf Note:} This chapter describes how to define new types in \opal.
This knowledge is not vital for trivial programs, since \opal\ offers a
sophisticated set of predefined types in the standard library.

A really novice user may skip this chapter altogether. 
Nevertheless, types and typing are fundamental for efficient and correct
programming, so you should return to this chapter immediately after
writing your first few programs.
\bigskip


%Types are supported by nearly every modern programming language. 
%Nevertheless, we know from theory that types are not vital for
%programming and indeed there are several programming languages which
%have no typing at all, e.g.~LISP, Prolog and Assembler.
%
%Other languages offer types on a low level (only predefined types are
%allowed, the programmer can't define new types) or dynamic typing
%(the type of an expression or object is determined at the runtime of the
%program), e.g.~Smalltalk.
%\medskip

\advanced \opal\ is a strongly typed language with static typing which offers
powerful notations for user-defined types.
Each expression in a program will be associated with a unique type at
compile time%
\footnote{This associated type is described as the  functionality of
  an expression in
  Section \ref{sec:expr.simple}.}. 
If this fails, a context error will result.
\medskip

%What are the advantages of this concept given that  it complicates the compiler
%and also the writing of programs?
%It improves the quality of a program!
%The programmer is forced to think about the domain of arguments
%and results of  functions. 
%Furthermore typing is that  part of program verification which can
%be done automatically at compile time.
%
%A simple example: Let us have a function which runs on numbers.
%In general (as we know from mathematics) it won't make any difference
%if you submit real numbers or natural numbers as arguments.
%But in practice the function won't work on real numbers because of
%rounding errors. 
%So you will forbid the function to be used with real numbers.
%
%This will be enforced by restricting the arguments to natural numbers
%which is realized by adequate typing of this function.
%Whenever someone tries to use the function with real numbers he will
%receive a type mismatch error by the compiler.
%
%So typing is a great help for the programmer to write error-free
%programs.
%Furthermore static typing (analysis at compile time) is more sophisticated
%than dynamic typing (at runtime), because with dynamic typing the error
%``type mismatch'' is submitted to the user of a program instead of to
%the programmer, who has is the person to correct the failure.
%\bigskip


In the following sections we will first introduce the concept of free
types, as used in \opal, and the declaration of types (see
Section~\ref{sec:freeTypes}).  
Then we will explain how to define (i.e.~implement) free types (see Section
\ref{sec:definition.Types}).

Free types also offer an alternative way of defining functions using
pattern-matching. 
This results in style for function definitions
 that is similar to term-rewriting.
For details, see Section \ref{sec:patternType}.

We will finish this chapter with some concluding remarks about
parameterized types (see \ref{sec:paramType}) and type synonyms (see
\ref{sec:synonType}). 


\section{The Concept of Free Types}
\label{sec:freeTypes}
\advanced
Types in programming languages correspond to sets in mathematics.
Thus declaring and defining (in terms of programming) means to
define a set (in terms of mathematics).

In \opal\ there are no predefined types\footnote{Well, in fact there
  are two: booleans and denotations are actually built-ins of the
  compiler for obvious reasons.}, but the standard library offers
several frequently used types such as natural and real numbers, characters,
strings and also more complex types such as lists, arrays, mappings, trees
etc.

In this section we will introduce the concept of free types together
with  the {\em declaration\/} of \opal\ data types.
For the definition you should refer to the following section,
"Definition of Types".

\subsection{A First Example: Enumerated Types}
\label{sec:enumType}
\advanced
 The most simple way to define a set (in mathematics) is to
enumerate all of its elements. 
If you need colors you may declare a type \pro{color} by
enumerating all colors:
\begin{prog}
       TYPE color == red blue yellow green cyan orange
\end{prog}

By this declaration a new set named ``color'' is introduced; in terms of
programming, the sort ``color'' has been declared:
\begin{prog}
       SORT color
\end{prog}

This set consists of six elements, the colors red, blue, yellow,
green, cyan and  orange.
In terms of programming, six constants of type ``color'' have been
declared\footnote{Don't bother about the keyword \pro{FUN}. A constant is just a
function without arguments.}:
\begin{prog}
        FUN red : color
        FUN blue: color
        FUN yellow: color
        FUN green : color
        FUN cyan : color
        FUN orange : color
\end{prog}

And,  moreover, by this declaration six discriminator functions have
been declared\footnote{These discriminators have no
  counterpart in mathematics.}:
\begin{prog}
        FUN red? : color -> bool
        FUN blue? : color -> bool
        FUN yellow? : color -> bool
        FUN green? : color -> bool
        FUN cyan? : color -> bool
        FUN orange? : color -> bool
\end{prog}
As there is no predefined equality on types (see below) these discriminator
functions are the only way to distinguish between the six elements of
the set ``color''.  
The function call ``\pro{blue?(x)}'' yields ``\pro{true}'' if and only
if its argument ``\pro{x}'' is evaluated as the constant ``\pro{blue}''.


\subsubsection{Induced Signature}
\advanced
The sort, the constants and the discriminator functions are called the {\em
  induced signature\/} of a type declaration\footnote{Other elements
  of the induced signature are constructors and selectors which will
  be dealt with later.}.
The  type declaration not only declares the new sort, but also
all constants and functions of the induced signature.
In the example above you have declared twelve operations and one sort
in a single line! So data type declarations are a very powerful
concept.

\medskip
A type declaration declares by default all objects of the induced
signature too. 
Nevertheless, you may declare them explicitly as done above.
If a declaration of the induced signature is missing, the compiler
will add it by itself.

\experienced In fact a type declaration is not only a declaration,
 although   it is a specification, because it fixes the  behavior of the
objects of the induced signature as described above (and below).

 

\subsubsection{Equality and Ordering}
\advanced Note that {\em only(!)\/} the objects of the induced
signature are declared by the  type declaration.
This means there is no equality relation, no ordering or anything
else.

If, for example, you need equality of colors, you have to program it
yourself\footnote{Remember, you may use any identifier instead of the 
  equal sign.}:
\begin{prog}
       FUN = : color ** color -> bool
       DEF = (a,b) == ((a red?) and (b red?))
                      or (((a blue?) and (b blue?))
                      or ....
\end{prog}
This may be tedious, but  because data types in \opal\ may contain functions
 and there is no computable function which can check the equality of
 functions, there is no way the compiler can generate an
equality relation .

There is also no ordering on the elements of an enumeration type. 
There is no sense in saying ``blue is smaller then yellow''.
Thus, there are no relations like ``$\leq$'' or ``$>$'' between
elements of a type.
If you need any ordering relation you have to declare (and define) it
yourself. 


\subsection{A Second Example: Product Types}
\label{sec:productType}

\advanced
A very common problem is the combination of several different values in one
single data object. 
 The particulars of a person constitute a very typical example.
If you want to combine the surname (\pro{name}), the first name and a personal
identification number in one object ``person'' you may declare:
\begin{prog}
        TYPE person == person ( name : string,
                                firstName : string,
                                id : nat)
\end{prog}

Don't be worried about the two occurrences of the identifier
\pro{person}.
Since \opal\ supports overloading you may use different names or the same
name, just as you prefer.

The first \pro{person} (on the left-hand side of the ``\pro{==}'') declares
a new sort \pro{person}, similar to the \pro{color} example:
\begin{prog}
        SORT person  
\end{prog}

The second \pro{person} (on the right-hand side of the ``\pro{==}'') may be
compared with one of the concrete colors, only now it is not a constant,
but a function which takes three arguments (a string, another string
and a natural number) to construct a new element of the sort \pro{person}.
\begin{prog}
        FUN person: string ** string ** nat -> person
\end{prog}

As an example, by the function call \pro{person("Chaplin",
  "Charles", 1234)} the three elements ``Chaplin'', ``Charles'' and
1234 are merged into one object of type \pro{person}.

\smallskip
As in the first example with colors, the declaration above also declares
a discriminator function
\begin{prog}
        FUN person? : person -> bool
\end{prog}
which is rather useless in this example.
\medskip

Furthermore, for each component which is combined by the constructor,
there is a corresponding selector function:
\begin{prog}
        FUN name: person -> string
        FUN firstName: person -> string
        FUN id: person -> nat
\end{prog}
An object of type person could be decomposed by these selector
functions  into its elements. 
Let \pro{p} be an object of type person, for example declared by 
\begin{prog}
        LET p == person("Chaplin", "Charles", 1234)
\end{prog}
In this case the application of the first selector \pro{name} on
\pro{p} will select the first element of the triple; hence, the
evaluation of \pro{name(p)} yields \pro{"Chaplin"}.
 You may select the second and third element in the same way:
\pro{firstName(p)} yields \pro{"Charles"} and \pro{id(p)} yields
\pro{1234}.

Constructors and selectors are opposites.
A constructor composes components into a single object whereas
selectors decompose objects into their components or---more
precisely---they select a component from a composed object.


\subsubsection{Induced Signature}
\advanced
The description in the first example can now be expanded to include
selectors as another part of the induced signature.
The same rules apply to selectors as to the induced signature's
other components
%Expanding the description in the first example, the selectors are also
%part of the induced signature and the same rules apply to selectors
%as to the other components of the induced signature
 (e.g.~automatic declaration etc.).

\experienced Let us have a concluding note, why this kind of type
declaration is called a ``product type''.

Imagine you have only a very small computer with a limited number of
different strings and natural numbers.
Say the cardinality of the set  {\em string\/} is 1000 and the cardinality
of the natural numbers is limited to 65536.

The set person is the three-dimensional mathematical cross-product of
the set {\em string\/}, once more the set {\em string\/} and the set
{\em nat\/} 
 ({\em string\/} $\times$ {\em string} $\times$ {\em nat\/}), 
which means that the total cardinality of the 
set {\em person} is $1000 * 1000 * 65536 \approx 65*10^9$.  


\subsection{The General Concept: Sums of Products}
\advanced
Both  examples in the previous sections are only special cases of the general
concept \opal\ offers for declaring new types. Let us explain this
concept in another example about particulars. 

We are now no longer interested in the identification number, so we
will omit it.
But we want to  know if the person is single, married, widowed or
divorced. 
And if the person is married, we also want to know their
spouse's name.
On the other hand, if the person is  widowed or divorced, we are not
interested 
in the spouse's name but we  do want to know the date the spouse
died or the marriage was dissolved.

This could be expressed with  a free type like%
\footnote{The sort \pro{date} must be declared somewhere, but we won't
  bother about it now.} 
\begin{prog}
        TYPE person == single  ( name : string, firstName : string)
                       married ( name : string, firstName : string,
                                 spouse : string)
                       widowed ( name : string, firstName : string,
                                 dateOfDeath : date)
                       divorced( name : string, firstName : string,
                                 dateOfDivorce:date)
\end{prog}

Before discussing the details let us summarize the signature induced
by this  type declaration.
First of all there is the new sort \pro{person}:
\begin{prog}
        SORT person
\end{prog}
Then we have four constructors, one for each alternative:
\begin{prog}
        FUN single  : string ** string -> person
        FUN married : string ** string ** string -> person
        FUN widowed : string ** string ** date -> person
        FUN divorced: string ** string ** date -> person
\end{prog}

Each constructor has a corresponding discriminator:
\begin{prog}
        FUN single?  : person -> bool
        FUN married? : person -> bool 
        FUN widowed? : person -> bool 
        FUN divorced? : person -> bool 
\end{prog}
 Finally, there are several selectors:
\begin{prog}
        FUN name       : person -> string
        FUN firstName  : person -> string
        FUN spouse     : person -> string
        FUN dateOfDeath  : person -> date
        FUN dateOfDivorce: person -> date
\end{prog}

\bigskip
The  type consists of four alternatives (or variants).
An object of type \pro{person} will be constructed either by the
constructor \pro{single} or \pro{married} or \pro{widowed} or
\pro{divorced}\/  in the same way construction of objects was
explained in the previous section 
(see Section \ref{sec:productType}). 
But in the previous section there was only one alternative.
Note that the four constructors take different arguments, e.g.\
\pro{married} takes three strings whereas \pro{divorced} needs two strings
and an object of type \pro{date}.

If you have an object of type \pro{person}, you need to know which kind of
person it is, i.e.~ you want to know which constructor was used to compose
this object.
This could be tested using the discriminators  as outlined in Section
\ref{sec:enumType}. 
There the discriminators were used to distinguish between the
different colors, now they are used to distinguish between the different
kinds of a \pro{person}.

 For example, if an object \pro{p} is constructed by \pro{married}
\begin{prog}
        LET p == married(\dots, \dots, \dots)
\end{prog}
then the test \pro{married?(p)} yields \pro{true} and \pro{widowed?(p)}
yields \pro{false}.
\medskip

You have to be able to distinguish between the four variants not only
for algorithmic reasons (persons with different marital status
will need different algorithmic treatment), but also for
technical reasons.
As the variant \pro{married} does not contain a component about
\pro{dateOfDeath}, for example, you can't select this component from that
variant.
This means, assuming \pro{p} declared as above, the function call
\pro{dateOfDeath(p)} will result in a runtime-error with program
abortion.
On the other hand, the component \pro{name} is part of each
alternative, so you can use this selector in all cases.

This can't be checked by the compiler, so it is the programmer's
responsibility to ensure that he only uses a selector in those cases
where there is also a corresponding component.

In general this will lead to a commonly employed scheme. A function with an
argument of type \pro{person} will first distinguish the variant of
the argument and then do the real work:
\begin{prog}
  DEF foo(\dots, p, \dots) ==
          IF single?(p) THEN \dots
          IF married?(p) THEN \dots
          IF widowed?(p) THEN \dots
          IF divorced?(p) THEN \dots FI
\end{prog}
This scheme could be expressed very elegantly with pattern-matching (see
Section~\ref{sec:patternType} for details).

\subsubsection{Context Conditions}
\advanced
There are a few more interesting details concerning the  type declaration
discussed above.
As you can see, it is possible to use the same selector name in 
different variants (e.g.~\pro{name}). 
 The corresponding sorts need not be  the same (in
the case of \pro{name}, the sort \pro{string}) as they are just
overloaded identifiers, as permitted in \opal.

Of course you may use different selector names to select ``similar''
components as in \pro{dateOfDeath} and \pro{dateOfDivorce}.
But then you have to take care that the selector will only be applied
to its own variant!
 
\medskip
The ordering of variants doesn't matter at all.
 The ordering of the selectors is only relevant with respect to the
 functionality of the constructor.
You may also declare the variant \pro{divorced} as
\begin{prog}
  \dots
  divorced( dateOfDivorce : date, name : string, firstName : string)
\end{prog}
In this case only the functionality of the constructor \pro{divorced}
changes into \pro{FUN~divorced : date ** string ** string -> person};
everything else remains unchanged.
\medskip

The names of all constructors of a type must be different. 
Otherwise you won't be able to distinguish between the different
variants.

The name of the sort must be unique among all the sorts declared in this
structure, but there is no problem using the same identifier for a sort and
any function (including constructors and selectors; see the example in
Section \ref{sec:productType}). 

There is also no problem having the same identifier for constructors and
selectors, as long as they can be distinguished by their functionality.


\subsection{Recursive Types}
\label{sec:recType}
\advanced
In the imperative age of computing recursive types were regarded
as the ultimate data types, and they were notoriously difficult to
manage (dealing with pointers!).

In \opal\ there is nothing magic about recursive types at all, as they fit
naturally into the concept already presented.

Suppose you want to include the parents of a person in the
particulars.
Well, the parents are persons too, so it is very easy to declare the
further enlarged  type \pro{person}: 
\begin{prog}
       TYPE person == single  ( name : string, firstName : string,
                                father: person, mother : person)
                      married ( name : string, firstName : string,
                                spouse : string,
                                father: person, mother : person,
                                spousesFather: person, 
                                spousesMother : person)
                      widowed ( name : string, firstName : string,
                                dateOfDeath : date,
                                father: person, mother : person)
                      divorced( name : string, firstName : string,
                                dateOfDivorce : date,
                                father: person, mother : person)
\end{prog}
Now each object of type \pro{person} includes the father and the
mother of this person and in the case of a married person, also the parents
of the spouse.
Of course the induced signature changes a lot, but we won't write it
down explicitly any more.  

There is only one problem left.
To declare a person you need the particulars person's father and mother.
This means you  first have to declare two objects of the sort
\pro{person} as father and mother.
But to declare them you need four persons as grandparents and so on.

The nice consequence is that you finally get a whole family tree of
all ancestors. 
The ugly consequence is that this declaration results in an endless
data recursion (very similar to an endless algorithmic recursion in
function definitions), because you {\em always\/} need the parents of
a person to construct an object of type \pro{person}.

In reality each person has an infinite number of ancestors, but from some
point in time these are not known any more.
We will model this in our type by adding a new variant
\pro{unknown}:
\begin{prog}
       TYPE person == single  ( name : string, firstName : string,
                                father: person, mother : person)
                      married ( name : string, firstName : string,
                                spouse : string,
                                father: person, mother : person,
                                spousesFather: person, 
                                spousesMother : person)
                      widowed ( name : string, firstName : string,
                                dateOfDeath : date,
                                father: person, mother : person)
                      divorced( name : string, firstName : string,
                                dateOfDivorce:date,
                                father: person, mother : person)
                      unknown
\end{prog}
Now whenever we are missing information  about a person we can use
the constant 
\pro{unknown}, which terminates the data recursion.

\experienced The notion ``free type'' comes from algebra. 
All elements of the sort \pro{person} can be constructed using the
constructors and they form a model of the ``freely constructed term
algebra'' of the corresponding data type.

\advanced There are also examples of those data structures which
can't be expressed directly by free types.
A simple example is a type that includes the spouse of a person as
object of type \pro{person} again.
But the spouse of the spouse is the original person. 
This cyclic relation can't be expressed with a free type.

%Another example can be found in Appendix~\ref{app:graph}: ``An
%Arbitrary Directed Graph''. 

\section{Definition of Types}
\label{sec:definition.Types}
 \advanced
Definition (i.e.~implementation) of types is derived straightforwardly from
declaration of  types (see the previous section).
To implement a  type, just substitute the keyword \pro{TYPE} of a type
declaration with \pro{DATA} and you get an implementation of all
objects of the corresponding induced signature:
\begin{prog}
       DATA person == single  ( name : string, firstName : string,
                                father: person, mother : person)
                      married ( name : string, firstName : string,
                                spouse : string,
                                father: person, mother : person,
                                spousesFather: person, 
                                spousesMother : person)
                      widowed ( name : string, firstName : string,
                                dateOfDeath : date,
                                father: person, mother : person)
                      divorced( name : string, firstName : string,
                                dateOfDivorce : date,
                                father: person, mother : person)
                      unknown
\end{prog}

\noindent This implements a data type which fulfills all the
characteristics of 
the corresponding free type, as described in the previous section.
Therefore a type-definition is as powerful as a type-declaration.

\medskip
If a corresponding free type for the sort is missing, the compiler
automatically derives the free type from the definition.
Therefore you can use, for example, pattern-based function definitions even
without explicit declaration of a free type.

Note, however, that---in contrast to other objects---a type declaration in the
signature part is {\em not\/} submitted to the implementation part.
This means you have to define all objects of the induced signature (as for all
objects of the signature part), but the information about being a free
type is not available in the implementation part. 

\subsection{Implementation Differing from Declaration}

\experienced It is also possible to have an implementation which
differs from the declared free type.
This is very useful in cases where you want to hide implementation
details.

Imagine you want to write a structure for manipulating text.
Conceptually it is a good idea to represent texts as lists of
characters (similar to sequences):
\begin{prog}
  SIGNATURE Text
  IMPORT Char ONLY char
  TYPE text == :: (ft:char, rt : text)
               <>
 >>>{\em a lot of additional operations\/}<<<
\end{prog}

But in practice you can't implement a text as a sequence of
characters, because this implementation is slow (imagine, if you wanted to
select the 5000th character in a text) and wastes enormous amounts of
memory ( you need additional memory at
least four times as large as  character stored in order to refer the next
character). 

{\bf Note:\/} Although the implementation of texts as sequences is impractical,
it is a very quick and reliable method for rapid prototyping of a program.
\medskip

Arrays are known as fast and economical alternatives to sequences, but
they are limited in size.
Therefore we decided as a compromise to implement texts as a sequence
(unlimited length) of arrays with fixed length (efficient):
\begin{prog}
  DATA text == maketext(firstBlock: array[char],
                        firstFree : nat,
                        rest      : text)
               <>
  /* ASSURANCE: The array component has a constant size 
                     of 1024 Elements (1kB)
                The Array is filled with text up to but not including
                     the Element indexed by firstFree */
\end{prog}

In this case only the objects of the induced signature of the free
type corresponding to the  {\em type definition\/} are defined.
You as programmer are responsible for defining all objects of the
induced signature of the {\em type declaration\/}.
You must ensure that your functions behave just as if they
were defined by a \pro{DATA}-definition equivalent to the
\pro{TYPE}-declaration. 

In particular this means:
\begin{itemize}
\item The functions \pro{<>} and \pro{<>?} are already defined by the
  \pro{DATA}-definition. But you must ensure that an empty text is {\em
    always\/} represented by the constant \pro{<>}.
\item The discriminator \pro{::?} has to be defined by hand, e.g.~ as
  \begin{prog}
    DEF ::?(t) == {\raisebox{-1ex}{\Large \tt~}}(<>? (t))
  \end{prog}
\item The selector \pro{ft} could be defined as
  \begin{prog}
    DEF ft(t) == IF firstFree(t) > 0 THEN (firstBlock(t))[0] FI
  \end{prog}
\item The definition of the remaining functions \pro{::} and \pro{rt}
  will be left as an exercise.
\end{itemize}



%\section{Free Types}
%\label{sec:freeType}
%\advanced It is a reliable programming practice to encapsulate an abstract data
%type in a structure of its own and to export only a small set of
%operations which are required to deal with this data type. 
%All of those nasty details which are managed by auxiliary functions
%are hidden within this structure and invisible to the programmer, who
%uses this structure.
%
%In \opal\ you are able to only {\em declare\/} a type and to use a
%completely different implementation to define it
%.
%A well-known example for such a hidden realization is the UNIX-filesystem. 
%From the users point of view a Unix-file is just a sequence of
%characters.
%But the implementation is much more complex to deal with
%hardware-requirements (e.g.~blocks), efficiency (in space and time) and
%so on.
%
%This effect can be achieved very easily in \opal.
%In the signature part of the structure you declare a {\em free
%  type\/} (which is a completely analogy to Sequences):
%\begin{prog}
%  SIGNATURE UnixFile /*signature part*/
%  TYPE file == :: (ft:char, rt:file)
%               <>
% >>>{\em maybe a lot of additional operations\/}<<<
%\end{prog}
%Note the different keyword. 
%Whereas ``\pro{DATA}'' {\em defines\/} a data type (e.g.~it is a
%concrete implementation), the keyword \pro{TYPE} only {\em declares\/}
%a type.
%This means all operations (including the sort) of the induced
%signature are declared, but there is no implementation associated with
%them.
%
%Often you will choose to use the same structure of the abstract data
%type also for the implementation\footnote{This is also a good way to
%  quickly have a simple prototype version of a program.}:
%\begin{prog}
%  IMPLEMENTATION UnixFile /*implementation part, prototype version*/
%  DATA file == :: (ft:char, rt:file)
%               <>
%\end{prog}
%
%But you are free to choose a completely different implementation, which
%fits much better to the e.g.~hardware-requirements.
%
%In the concrete example of the  UNIX-file you might want to model the
%file as an array of datablocks, which is faster then the sequences and could
%also reflect the structure of data-blocks of real disk-drives.
%The idea is to have as much regularly structured datablocks as
%possible because these could be stored very efficiently.
%Only the very first block may be incomplete, because a file normally
%won't fit exactly into a multiple of 1024 characters:
%\begin{prog}
%  IMPLEMENTATION UnixFile /*implementation part, improved version*/
%  DATA file == toFile(firstBlock:datablock, regulars:array[datablock] )
%  DATA datablock == toData(arrayOf :array[char])
%  /* ASSURANCE: a regular datablock always has 1024 Elements (1kB) */
%\end{prog}
%
%In this case you must ensure, that for each function of the induced
%signature of the free type there will be an implementation, which
%behaves like the corresponding function of the induced signature of a
%data type definition. 
%
%This means, you must define the following function by yourself:
%\begin{prog}
%  FUN :: : char ** file -> file
%      <> : file
%  FUN ::? : file -> bool
%      <>? : file -> bool
%  FUN ft : file -> char
%      rt : file -> file
%\end{prog}
%
%The discriminator functions e.g.~could be implemented like
%\begin{prog}
%  DEF ::?(f) == (#(arrayOf(firstBlock(f))) = 0) and (#(regulars(f))=0)
%  DEF <>?(f) == (#(arrayOf(firstBlock(f))) > 0) or  (#(regulars(f))>0)
%\end{prog}
%where  \pro{\#} means length of an array and \pro{=},\pro{<} and \pro{0} are
%imported from the structure \pro{Nat}.
%
%The constructor \pro{<>} is trivial to implement (only empty arrays):
%\begin{prog}
%  DEF <> == LET dummy == chr(0) 
%                emptyBlock == toData(init(0,dummy))
%            IN toFile(emptyBlock, init(0,emptyBlock))
%\end{prog}
%
%The other functions are a little bit more difficult to implement, but
%the principle is straight forward:
%\begin{prog}
%  DEF ft(f) ==
%     IF #(arrayOf(firstBlock(f))) > 0 THEN arrayOf(firstBlock(f))!0
%     IF #(arrayOf(firstBlock(f))) = 0 THEN arrayOf(regulars(f)!0)!0
%     FI
%
%  DEF ::(ch,old) == 
%     LET oldFB == arrayOf(firstBlock(old))
%     IN
%     IF #(oldFB) < 1024 THEN /* still space in the first block */
%        LET newFB == init(#(oldFB)+1, 
%                          \back\back i. IF i=0 THEN ch
%                               IF i>0 THEN oldFB!(i-1) FI)
%         IN toFile(toData(newFB), regulars(old))
%     IF #(oldFB) = 1024 THEN /* first block full, create new one */
%        LET oldRegs == regulars(old)
%            newRegs == init(#(oldRegs)+1, 
%                            \back\back i. IF i=0 THEN toData(oldFB)
%                                 IF i>0 THEN oldRegs!(i-1) FI)
%            dummy == chr(0) 
%            newFB == toData(init(0,dummy))
%         IN toFile(newFB, newRegs)
%     FI
%\end{prog}
%The implementation of \pro{rt} will be left as an exercise.
%


\section{Pattern-Matching}
\label{sec:patternType}

\advanced 
Pattern matching is a syntactic alternative for defining functions.
Instead of only giving a formal parameter name at the left-hand side of a
function definition, 
you supply a pattern; this means this function definition will only
be used if the argument matches the pattern.

Let us take the  type \pro{person}  declared above as an example.
Then the
function \pro{FUN knownAncestors:person->nat}, which should compute the
number of ancestors including the person itself, could be defined as
follows:
\begin{prog}
DEF knownAncestors(p AS unknown) == 0                /* definition 1 */
DEF knownAncestors(p AS single(n, fn, fa, ma)) ==    /* definition 2 */
              1 + knownAncestors(fa) + knownAncestors(ma)
\end{prog}
In this case a definition only matches if the argument to the usual
parameter \pro{p} has 
a shape corresponding to the term behind the keyword \pro{AS}.
More specifically this pattern-based definition will be interpreted as
follows: 
\begin{itemize}
\item if the argument of a call of \pro{knownAncestors} is the
  constant \pro{unknown}, then the value is \pro{0} (as defined by
  Definition 1)
\item if the argument of a call of \pro{knownAncestors} is constructed
  with \pro{single}, then the arguments of the constructor can be
  accessed by the newly introduced parameters \pro{n}, \pro{fn},
  \pro{fa}, \pro{ma}, and the value 
  results from the right-hand side of the second definition.
\item in all other cases the application of this function will result
  in an runtime-error, but the compiler will check whether all variants are
  covered and warn you beforehand. 
  You may complete the definition by yourself. 
\end{itemize}

If you don't need the parameter itself, but only the arguments of the
pattern, you can omit the parameter and the keyword \pro{AS}:
\begin{prog}
DEF knownAncestors(unknown) == 0                /* definition 1.1 */
DEF knownAncestors(single(n, fn, fa, ma)) ==    /* definition 2.1 */
              1 + knownAncestors(fa) + knownAncestors(ma)
\end{prog}
This has a touch of term rewriting. A term on the left
side (as parameter of a function definition) is substituted by the
right side.
\medskip

The patterns may be nested. So we can write:
\begin{prog}
        DEF knownAncestors(single(n, fn, unknown, unknown)) == 1   
                                                 /* definition 3 */
\end{prog}
This means that if both parents of an unmarried person are unknown, the
number of ancestors is 1, just the person itself.

 Note that Definition 3 does not conflict with Definition 2 as you
might expect. 
Definition 3 is a more specialized version of Definition 2 and the
program will first check whether the most specific version is appropriate
and only if this fails will it use the more general one.

During compilation, pattern-based definitions are collected and
transformed into one single definition with a large case distinction.
Therefore in cases of ambiguity (which of two (or more) patterns is
more specific) you should ensure that their definitions deliver the
same results anyway.


\important Pattern matching can only be done on the constructors of a
free type (see Section \ref{sec:freeTypes}: ``The Concept of Free
Types'').
 But since a data type definition also induces a corresponding free type if
the free type is missing, this is not a serious restriction. 
Nevertheless, pattern matching cannot be done on arbitrary functions;
only constructors (and formal parameters) are allowed as elements of patterns.

It is a common error to write, e.g.
\begin{prog}
  DEF f(0) == ...
  DEF f(1) == ...
  DEF f(2) == ...
  DEF f(a) == ...
\end{prog}
This is wrong because only the natural number $0$ is a constructor,
while $1$ and $2$ are {\em not\/}! 
In this case you have simply introduced local names for parameters  and
it is all the same whether you call them $1$ and $2$ or $n$ and $m$ or $x$
and $y$.  


\subsection{Using Wildcards in Pattern-Based Definitions}
\label{sec:wildcard}
\advanced
In the second definition of the example in the previous section 
\begin{prog}
  DEF knownAncestors(single(n, fn, fa, ma)) ==    /* definition 2 */
              1 + knownAncestors(fa) + knownAncestors(ma)
\end{prog}

\noindent the values of the parameters \pro{n} and \pro{fn} are never used.
In this case \opal\ allows the use of an underscore as a wildcard with the
meaning: I know there should be a parameter, but I am not interested
in its value at all.
\begin{prog}
  DEF knownAncestors(single(_, _, fa, ma)) ==    /* definition 2.1 */
              1 + knownAncestors(fa) + knownAncestors(ma)
\end{prog}
 Using wildcards has the advantage that you don't need to invent new
 names for objects never used. 
This also improves the readability of the definition.

\section{Parameterized Types}
\label{sec:paramType}
\advanced
A very frequent problem in programming is to construct the same data type over
different objects.
A classical example for this problem are  sequences. 
It does not matter whether you are implementing sequences of natural numbers, of
characters or of persons. 
The  algorithms and the data type will be the same in all cases.
Only the sort the sequence is based on will change.

One can say the data type sequence is parameterized with the concrete
basic set.
 In \opal\ this could be expressed using parameterized structures.

For example, the parameterized data type \pro{seq} (as included in the
standard library) could be declared and implemented as:
\begin{prog}
  SIGNATURE Seq[data]
  SORT data
  
  IMPORT Nat ONLY nat

  TYPE seq == ::(ft :data, rt : seq)       /* as free type */
              <>
  FUN # : seq -> nat                     /* length of a seq */
  >>>{\em a lot of additional operations\/}<<<
\end{prog}

\medskip
\begin{prog}
  IMPLEMENTATION Seq[data]

  IMPORT Nat ONLY 1 + 

  DATA seq == ::(ft :data, rt : seq)           /* implementation of */
              <>                               /* free type         */

  DEF # (s) == IF ::? (s) THEN 1 + #(rt(s)) 
               IF <>?(s) THEN 0 FI

  >>>{\em much more definitions\/}<<<

\end{prog}

This declares and defines a parameterized structure with a
parameterized data type \pro{seq}.

This can be used, for example, to  declare a function \pro{convert} which
converts a sequence of numbers into a sequence of characters by using
the  structure \pro{Seq} and instantiating the sort \pro{seq} with the
concrete sorts \pro{nat} and \pro{char}: 

\begin{prog}
  IMPORT Seq ONLY seq
  FUN convert : seq[nat] -> seq[char]
\end{prog}
We won't bother with what this function will do.

For more details about parameterization and instantiation, see Section
\ref{sec:struct.param}: ``Parameterized Structures and Instantiations''.


\section{No Type Synonyms}
\label{sec:synonType}
Sometimes you may want to rename a sort  or use different names
for the same sort. This is called type synonyms. 
Unfortunately, \opal\ does not support type synonyms.

For example, if you don't like the standard strings based on arrays,
you may want to substitute them with an implementation based on the
predefined structure of sequence:
\begin{prog}
  DATA myString == seq[char]     /* illegal construction! */
\end{prog}

This is not allowed in \opal\ (in fact it is a syntactic error).
One possible circumvention  is to use \pro{seq[char]} instead of \pro{myString}
everywhere. Nevertheless, this ``solution'' negates your intention
to explicitly distinguish between strings and sequences of characters.

The other solution is to use embedding instead of synonyms:
\begin{prog}
  DATA myString == asMyString(asSeq : seq[char]) /* this is legal */
\end{prog}
In this case you really introduce a new type and the constructor and
selector only serve as type conversions or---from a different point of
view---as embedding operations.

The new type \pro{myString} doesn't inherit any functions from
\pro{seq}. Therefore you have to program all functions yourself,
e.g.
\begin{prog}
  FUN \# : myString -> nat
  DEF \#(mS) == \#(asSeq(mS))
\end{prog}
where the \pro{\#} at the right-hand side of the definition is the
well-known function from structure \pro{Seq}.
% Local Variables: 
% mode: latex
% TeX-master: "tutorial"
% End: 

% LAST EDIT: Tue May  3 12:18:14 1994 by Juergen Exner (hektor!jue) 
% LAST EDIT: Tue Feb 15 16:12:21 1994 by Juergen Exner (hektor!jue) 
% LAST EDIT: Tue Nov 16 10:43:28 1993 by Mario Suedholt (kreusa!fish) 
% LAST EDIT: Tue Jan 12 16:30:57 1993 by Juergen Exner (hektor!jue) 
\chapter{Programming in the Large}
\label{chap:large}
\novice
In contrast to programming in the small---which covers how to define
single functions and data types---programming in the large describes
how several functions and data types can be combined in separate units to
emphasize the abstract structure of a program.
These units are often called modules, structures or classes.
Programming in the large also involves some other features, such as separate
compilation or re usability of parts of a program.

In general, \opal\ follows the same principles of modular programming
as other modern programming languages, e.g.~Modula-2.
The types and functions, etc.~are collected in {\em structures\/}
(similar to modules in Modula-2), which may be combined by 
import interfaces to form a complete program (for details, see below). 
Information-hiding is realized by explicit export-interfaces,
analogously to Modula-2.

In the following we will describe structures and their combination by
import and export interfaces (Section \ref{sec:struct}), how to build
complete programs (Section \ref{sec:struct}) and we will
introduce parameterized structures and their instantiation
(Section~\ref{sec:struct.param}).

\section{Structures in \opal, Import and Export}
\label{sec:struct}
\novice
In \opal\ the basic compilation unit is the structure.
In this section we explain the relations between structures and
describe how structures can be combined.
\medskip

Each structure can be compiled independently from other structures.
Only the export interface (i.e.~the signature part) of imported
structures (directly or transitively imported) is required
and---provided the imported structure is not compiled already---it will
also be analyzed.
The compilation process is transparent for the user.
For details see ``A User's Guide to the \opal\ Compilation System''\cite{Ma}.

\medskip
As already mentioned in Chapter \ref{chap:example}, ``A First
Example'', a structure in \opal\ consists of two parts, the signature part
and the implementation part, which are physically stored in two files.
A signature part only contains declarations; all definitions are
 delegated to the implementation part.

Objects which are declared in the implementation part are local to
this structure (i.e.~they can't be used in other structures) and
therefore they are of no interest with respect to inter-structural relations.


\subsection{The Export of a Structure: The Signature Part}
\novice
The signature part of a structure defines the export interface of this
structure. 
Each object declared or imported in the signature part is said to be
exported by the structure. 
Only exported objects can be accessed from other structures by using
imports (see below).

\advanced To export a data type you can declare either a free type or the
objects of the induced signature in the signature part. 
If you don't include the type declaration in the signature part,
the information about being a free type won't be included in the
export interface.  
Therefore you can't use this information in other structures, it
is not possible, for example,  to use pattern-based definitions (on
this data type) outside this structure.
With the exception of very special applications, it is a good idea
to  always export the free type.
\medskip

\novice The signature part must be consistent on its own.
Specifically, if a sort is used to describe the functionality of
an object, this sort must be declared (maybe as part of the induced
signature of a free type) or imported in the signature part too.

All imports in a signature part must be selective, i.e.~you are not
allowed to use complete imports (see below) in the export interface.
\medskip

\advanced Furthermore, all parameterized structures which are imported in the
signature part must be instantiated (for details, see
Section~\ref{sec:struct.param}). Uninstantiated imports are allowed
only in the implementation part of a structure.


\subsubsection{Transitive Exports}
\novice
\opal\ supports  transitive exports. Transitive exports are objects
you have imported in the {\em signature part\/} of a structure and
therefore they are re-exported again by your own structure.

The consequences can  best be explained in an example.
Imagine you have a structure \pro{Mystruct} which only exports the
function \pro{FUN~foo~:~nat~->~nat}.
This requires the import of \pro{nat`Nat:SORT} (remember annotations!)
in the signature part, because otherwise the signature part won't be
correct:
\begin{prog}
           SIGNATURE Mystruct
           IMPORT Nat ONLY nat
           FUN foo : nat -> nat
\end{prog}

In this case the sort \pro{nat} is  exported by \pro{Mystruct} too,
although it's origin is the structure \pro{Nat} and it is not declared in
\pro{Mystruct}.

When importing \pro{Mystruct} somewhere else, you have to import
\pro{nat`Nat:SORT} additionally, either from the structure \pro{Nat}
or---if you prefer---from structure \pro{Mystruct}.
Otherwise the sort for the argument and the result of \pro{foo} will be
missing.


\subsection{The Import of a Structure}
\novice
To use objects declared and exported by another structure you have to
import them into your structure. 
An import can be complete or selective.
In the first case you write, e.g.
\begin{prog}
          IMPORT Nat COMPLETELY
\end{prog}
This will import all exported objects of structure \pro{Nat}. 
It could be regarded as enlarging your structure by the signature part
of \pro{Nat}.
All objects declared in the signature part of \pro{Nat} are now known
in the importing structure too.
Of course you can't redefine an imported object, because it has
already been implemented in  structure \pro{Nat}.

Remember that a complete import is not allowed in the signature part
of a structure.
\medskip

The import could be restricted to particular objects by naming them
explicitly:
\begin{prog}
          IMPORT Nat ONLY nat + - 0 1
\end{prog}
In this case only the sort \pro{nat} and the object \pro{+},
\pro{-},  \pro{0} and  \pro{1} are imported.

It is possible to have multiple imports from the same structure (and
even of the same object, although this is not very useful):
\begin{prog}
          IMPORT Nat ONLY nat + - 0 1
                 Nat ONLY * /
\end{prog}
It might be helpful in importing  additional objects for auxiliary
functions, for example, which should not appear in the main imports. 

\important The import hierarchy must be acyclic, i.e. if a structure
\pro{Struct\_a} imports a
structure \pro{Struct\_b}, then \pro{Struct\_b} can't import
\pro{Struct\_a}, neither directly nor transitively from other structures.


\subsubsection{Overloaded Names in Imports}
\novice
The handling of overloaded names in imports is different from the
general scheme for overloading in \opal.
Whereas normally the compiler complains if it can't identify an object
unambiguously, in the case of imports {\em all\/} possible objects are
imported.

The import of
\begin{prog}
         IMPORT Int ONLY int -
\end{prog}
for example, imports the sort \pro{int} and both functions \pro{-},
the unary as well as the dyadic:
\begin{prog}
         FUN - : int->int
         FUN - : int ** int -> int
\end{prog}

This generally simplifies the imports, but it can  sometimes also lead to an
unexpected error.
Imagine a structure which exports two functions \pro{is}:
\begin{prog}
          SIGNATURE Mystruct
          IMPORT ...
          FUN is : nat -> bool
          FUN is : char -> bool
\end{prog}
Then you do an import
\begin{prog}
          IMPORT Mystruct ONLY is
                 Nat      ONLY nat
\end{prog}
because you want to use the first function.

The compiler will complain with an error message such as ``\probreak{ERROR: application of is needs
  import of char}'', because {\em both\/} functions are imported and the second
one requires the sort \pro{char}.

If you only want to import the first function you must be more
specific and annotate the imported objects:
\begin{prog}
          IMPORT Mystruct   ONLY is:nat->bool
                 Nat        ONLY nat
\end{prog}

\bigskip
\novice Generally it is a good idea to import only those objects which are
really required by your algorithm.
You should prefer selective imports to complete imports for
several reasons: 
\begin{itemize}
\item Compilation time decreases because the compiler has less
  objects to manage.
\item The documentary expressiveness of the program text increases.
\item The quality of the program increases, as the programmer is
  forced to think more carefully about the imports.
\end{itemize}

Nevertheless, sometimes during program development it is useful to
do a complete import if you don't want to think too much about
the import.

There is a tool called ``browser'' available which transforms a complete
import into an appropriate selective import with respect to the
objects really needed, see \cite{browser} for details.


\subsection{Systems of Structures}
\advanced
\opal\ structures can be combined in subsystems and systems, thus 
comprising  private libraries. 
The handling of these ``super-large'' programming structures is the task
of the \opal\ Compilation System, not of the language \opal\ itself.

For more information on how to create subsystems with
several structures and use different libraries, refer to ``A User's
Guide to the \opal\ Compilation System'' \cite{Ma}.


\subsection{Importing Foreign Languages}
\experienced \opal\ supports the integration of modules written in
foreign languages into \opal\ programs.
In the case of C this is called hand-coding (because C is also the
target language of the compiler).

The import of foreign modules is recommended only for very special
purposes.
In the current release, e.g.~some few structures of the library have
been hand-coded due to efficiency reasons (e.g.~numbers and texts) or
because they need  routines which depend on the operating
system(e.g.~access to  the file system) and therefore cannot be
expressed in \opal\ itself. 

If you are thinking about doing hand-coding by yourself, {\em don't do it.\/} 
If you are prepared to go to any length, you should refer to
``Hand-coder's Guide to OCS Version 2'' \cite{handcoding}.

\section{\opal\ Programs}
\label{sec:opalProg}
\novice
A complete \opal\ program consists of several \opal\ structures which
may reside in various subsystems and in the actual working
directory. 
There is no syntactic item which identifies the ``main structure'', the
``entry point'' or the ``start function'' of a program.
Instead, you have to tell the \opal\ compilation system (OCS)
which structure should be considered the root of the program (i.e.~the
``main structure'') and which function the entry point.

The OCS call (cf. Chapter \ref{chap:example}: ``A first
Example: Hello World'')
\begin{prog}
      >  \underline{ocs -top HelloWorld hello}
\end{prog}
will compile the structure \pro{HelloWorld} and---as far as
necessary---all imported structures of \pro{HelloWorld} (directly
imported as well 
as transitively imported) and then link the resulting object code.

%\begin{sloppypar}
The compilation process is optimized such that structures are only
 recompiled when their former compilations are obsolete
because of changes in the program text.
This optimized compilation is handled automatically by OCS
and the user doesn't need to worry about the order of the several
compilation steps.
Submitting one compile command with the root structure of the program
will compile all necessary structures.
%\end{sloppypar}
\medskip

The call above tells OCS to use the function
\pro{hello} as the entry point of the program and this name will also
be the name of the generated executable binary file.
When calling the generated binary program \pro{hello}, the function
\pro{hello} of the root structure \pro{HelloWorld} will be evaluated.

\important This entry point must be a constant of type \pro{com[void]}
and must be exported by the signature part of the root structure.
\begin{prog}
          SIGNATURE HelloWorld
          ...
          FUN hello : com[void]
\end{prog}

There can be several functions with this functionality in the root
structure, so the entry point of the program can easily be altered;
 but for each
generated binary you must explicitly state which one should be used
as entry point.


\section{Parameterized Structures}
\label{sec:struct.param}
\advanced
 As already mentioned in Section \ref{sec:paramType}, a
common difficulty is  writing a data structure or algorithm which is more
general than the usual typing restrictions allow.

Sets are a typical example.
The algorithms for enlarging a set, checking if a data item is a member
of the set or computing the cardinality of a set are the same
whether it is a set of numbers, a set of strings or even a set
of a sets  of persons.
In traditional programming languages (e.g. Modula-2) you would have to
write the data structure set several times as set of numbers, set of
characters and so on.

\opal\ offers the concept of parameterized structures to avoid this
nasty and boring repetition.
You could write the structure set once using a parameterized structure
and the concrete instantiation (i.e. set of numbers, set of characters
etc.) will be fixed when using the structure.

In the following we describe how to write (Section \ref{sec:write.param}) and how to
use (Section \ref{sec:use.param}) parameterized structures.


\subsection{How to write Parameterized Structures}
\label{sec:write.param}
\advanced
 Writing parameterized structures is easy.
You just have to add the names of the parameters to the structure name
as annotations in the signature part:
\begin{prog}
          SIGNATURE Set[data, <]
\end{prog}
You also have to declare what these parameters should be:
\begin{prog}
          SORT data
          FUN < : data ** data -> bool
\end{prog}

In this example the first parameter \pro{data} is a sort and the second
``\pro{<}'' a dyadic operation which takes two elements of this sort
and yields a boolean value.
Of course you cannot define the objects of the parameter list
(\pro{data}, \pro{<}) in the implementation part.
But you can use them as if they were declared (and in fact they are)
in the structure \pro{Set} just like any other object.

As an example you could write a function
\begin{prog}
          FUN foo : data ** data -> data
          DEF foo(a,b) == IF a < b THEN a
                          IF b < a THEN b FI
\end{prog}
or declare a new type
\begin{prog}
          TYPE okOrError == ok(value:data)
                            error(msg:string)
\end{prog}
Because you can use the parameters anywhere in the structure, not only
the structure as a whole but also each single function and sort is
parameterized with \pro{data} and \pro{<}.

Remember however that \pro{<} is just a name---nothing more---and can be
substituted by any other name.
In the case of the structure \pro{Set} the  intention is, that the
function should be 
a total strict order, though this can only be expressed in comments
because this is a semantic requirement that can't be checked by the compiler.


\subsection{How to use Parameterized Structures}
\label{sec:use.param}
\advanced
To use a parameterized structure you have to substitute the abstract
parameters by concrete values. This is called instantiation.
To use sets of natural numbers you could import
\begin{prog}
      IMPORT Set[nat'Nat, <'Nat] ONLY set in #
\end{prog}

In this import the parameter \pro{data} is instantiated with \pro{nat}
and the parameter \pro{<} with the function \pro{<} from the structure
\pro{Nat}.
Note that the fact that parameter and instance have same identifier
(\pro{<}) is  pure chance. 

The signatures of the imported objects are
\begin{prog}
      SORT set
      FUN in : set ** nat -> bool
      FUN #  : set -> nat
\end{prog}
Remember, in the structure \pro{Set} the function  \pro{in} has been
declared as \\ \pro{FUN~in~:~set~**~data~->bool}. 
This \pro{data} has been substituted by \pro{nat} due to the
instantiation.

Using this import you can write a function
\begin{prog}
      FUN foo : set -> bool
      DEF foo(s) == (#(s)) in s
\end{prog}
for example, which checks if the number of elements of a set is a 
member of the set itself. 
\bigskip

You can also use uninstantiated imports of parameterized structures.
Conceptually you import all possible instances of the structure; of
course these will be  quite numerous.

In the import
\begin{prog}
      IMPORT Set ONLY set in #
\end{prog}
the parameters \pro{data} and \pro{<} are still undefined.

 Uninstantiated imports are allowed only in the implementation part, in
signature parts you always have to use instantiated imports.

 Note that in case of an uninstantiated or more then one
instantiated import the application of \pro{set} is still ambiguous
and therefore has to be annotated to resolve this ambiguity.

If you want to declare a function \pro{transform}, which converts a
set of numbers into a set of characters, you have to instantiate the
sort \pro{set} by annotations:
\begin{prog}
      FUN transform : set[nat, <] -> set[char, <]
\end{prog}

Uninstantiated imports are useful if you need several different
instantiations. In the case above you may also do two
instantiated imports:
\begin{prog}
      IMPORT Set[nat, <] ONLY set in #
      IMPORT Set[char, <] ONLY set in #
\end{prog}
But this doesn't help because now you have two sorts named \pro{set}
and they too have to be distinguished in the declaration of
\pro{transform} by annotations.



% Local Variables: 
% mode: latex
% TeX-master: "tutorial"
% End: 


\appendix
% LAST EDIT: Tue May  3 13:43:01 1994 by Juergen Exner (hektor!jue) 
% LAST EDIT: Tue Feb 15 17:37:14 1994 by Juergen Exner (hektor!jue) 
% LAST EDIT: Thu Oct 14 20:34:30 1993 by Juergen Exner (hektor!jue) 
% LAST EDIT: Tue Jan 12 16:37:24 1993 by Juergen Exner (hektor!jue) 
\chapter{The Standard Library}
\label{chap:lib}
\label{sec:predef}
\novice
Conceptually \opal\ offers no built-in data types. There are no
numbers, no texts, no complex data types like arrays or anything else
as is customary in other programming languages.

In \opal\ these issues are deferred to the standard library, which
offers a large number of simple as well as powerful data structures
and algorithms.
At the moment there are more then one hundred structures and the
number continues to  increase as new features are added to the library.

In this appendix we will give a short survey of the features
offered by the standard library. 
It won't be a complete guide, just a glance.
For more detailed information you should refer to the guide
``Bibliotheca Opalica'' \cite{Di}, the library itself
(in subdirectory \pro{.../lib}) and  to the online documentation
system ``olm''.

\medskip
The library has a two-dimensional structure with several naming
conventions. In general there are several auxiliary structures for
each data type, which offer additional operations for the data
type in question.
These conventions are straigthforward.
They are explained in \cite{Di} and should help to prevent the user
getting lost in the huge library.

\bigskip
The library is divided into five subsystems: Internal, Basic Types,
Functions, Aggregate Types and System.
In the following sections we provide a short overview of each of
these subsystems.


\section{Internal}
\label{denotations}
\novice There are only two  structures in this subsystem which are
of interest to the user: \pro{BOOL} and \pro{DENOTATION}. 

Because boolean values and denotations are essential for the compiler,
they are (in contrast to all theory) in fact not realized with library
structures but are built-ins of the compiler. 
The two structures \pro{BOOL} and \pro{DENOTATION} only ensure correct
management during compilation.

These two structures are always imported automatically.
Therefore these two structures cannot be substituted by user-written
structures as it is possible for all other structures.

Note: There is a second data type (called \pro{string}) for
representation of text beside denotations; see \ref{sec:aggrTypes} for
differences. 

\section{Basic Types}
\novice In this subsystem the customary data types are declared. This encompasses
natural and integral numbers, real numbers, characters and also some
additional operations for bools and denotations. 



\section{Functions}
\novice The subsystem \pro{Functions} contains several structures which
support the combination of functions in various ways, e.g.~for the
well-known function-composition, for iterating functions, for
combining 
predicates or for defining ordering relations on arbitrary data
types.


\section{Aggregate Types}
\label{sec:aggrTypes}

\novice In the subsystem \pro{Aggregate Types} a bunch of more or less complex
data types is defined. 
These range from the very familiar, like \pro{strings}, to the
extremely complex, e.g. arbitrary graphs.

In detail, there are
\begin{itemize}
\item products, which realize cartesian products with up to four dimensions;
\item unions, which realize disjoint unions with up to four data types;
\item sequences of arbitrary elements as the most frequently used data
  structure in functional programming;
\item strings as an alternative method for representing text; Strings
  could be thought of as sequences of characters, whereas denotations are
  more like arrays of characters. 
  You should use denotations if the text does not change very often
  (e.g.~for fixed messages), whereas strings are better if you are
  continually modifying the text.

  There are also structures to scan and to format strings.
\item sets of arbitrary elements; Sets could also be defined with
  predicates, which describe the members of the set or as bitsets
  which are fast but limited in size.
\item bags as sets with multiple elements;
\item maps as implementations of arbitrary mappings;
\item arrays as a special (and efficient) form of mappings with a
  domain restricted to natural numbers;
\item graph-like data structures; This subsystem is still under
  construction. At the moment there are only AVL-trees available.
\end{itemize}

\section{System}
\novice This subsystem consists of four part:
\begin{itemize}
\item Debugging offers some functions we hope you will never need;
\item Commands realize the basis for input and output;
\item Streams offer a simple communication protocol, which is independent form
  the operating system used;
\item Unix supports access to UNIX-specific features like file-system,
  environment and processes;
\end{itemize}



%\subsection{Basic Data Types}
%\label{denotations}
%The standard library contains several structures which implement
%simple data\-structures:
%\begin{description}
%\item[BOOL:] The Boolean values \pro{true} and \pro{false} and several
%      standard operations.
%\item[DENOTATION:] The data type \pro{denotation}, a method for
%  denoting arbitrary objects, see below.
%\item[Nat:] Natural numbers with the usual operations.
%\item[Int:] Positive and negative numbers with usual operations and
%  conversions between \pro{int} and \pro{nat}.
%\item[Real:] The usual floating point numbers with their operations
%  and conversions between \pro{real} and \pro{int}.
%  Remember that floating point numbers are only an approximation of the
%  mathematical real numbers.
%\item[Char:] All characters, but based on the numerical
%  ASCII-representation.
%Offers also some constants and classification functions,
%e.g.~\pro{blank}, \pro{newline}, \pro{digit?}, \pro{lower?} etc.
%\item[PrintableChar:] Defines constants for all printable
%  ASCII-characters.
%\end{description}
%Denotations are just  lists of characters. They can't be modified and
%their only business is to be converted into other data types.
%They are very important as denotations are the only possibility for
%denoting arbitrary objects.  
%For any data type you could write conversion functions which convert
%denotations to the required data type, thus a denotation can be used
%to represents objects of any data type.
%
%For the standard data types like numbers, characters and also strings,
%these conversion routines are included in the corresponding
%structures (usually named ``\pro{!}'').
%This is also the appropriate solution for denoting arbitrary numbers,
%as the library structures only offer very few---often used---numbers
%as predefined constants. 
%
%
%\section{Complex Data Structures}
%The more complex data structures  are in general parameterized,
%e.g.~sequences, sets, bags, arrays and AVL-trees.
%We won't discuss them in detail.
%
%In general there is more than one structure for one abstract data
%type, such as sets.
%For technical reasons the structure parameter has to be
% enlarged with a second and even a third parameter sort for several
% higher order  functions like apply-to-all, filter, reduce and zip.
%For details you should read ``The \opal-Library''.
%
%Mappings (which belongs to this group too) can be used to define
%functions with an finite domain explicitly by enumerating all values.
%
%
%\section{Input and Output}
%Input and output are realized in \opal\ with continuations.
%The library structures for I/O contain the definition of the sort
%\pro{com} for commands, several functions for string oriented I/O,
%combining commands, checking the success of I/O-operations and, finally,
%low-level binary I/O.
%
%
%\section{Unix}
%The library also offers access to several Unix-specific features such as
%the environment, creating and manipulating processes and dealing with
%sockets.
%
%
%\section{Auxiliary Structures}
%There are also structures for combining up to four sorts in one data
%type, for function composition and even for control structures, so
%that an imperative approach is simulated.

% Local Variables: 
% mode: latex
% TeX-master: "tutorial"
% End: 

% LAST EDIT: Tue May  3 13:53:01 1994 by Juergen Exner (hektor!jue) 
% LAST EDIT: Tue Feb 15 19:03:01 1994 by Juergen Exner (hektor!jue) 
% LAST EDIT: Thu Nov  4 15:34:23 1993 by Juergen Exner (hektor!jue) 
% LAST EDIT: Mon Oct 18 15:11:21 1993 by Juergen Exner (hektor!jue) 
% LAST EDIT: Wed Oct 13 12:58:22 1993 by Juergen Exner (hektor!jue) 

\chapter{More Examples of \opal-Programs}
This appendix will summarize some examples of \opal\ programs.

\section{Rabbits}
\label{prog:rabbits}
The program Rabbits has already been presented in Chapter \ref{chap:intro}.
We include it here only for the sake of completeness.

It computes the Fibbonacci-numbers and it is an example of simple
interactive I/O. Also the conversions between denotations, strings and
natural numbers (``\pro{`}'' and ``\pro{!}'') are worth a  look.

\subsubsection{Rabbits.sign}
\begin{prog}
  SIGNATURE Rabbits
  IMPORT  Com[void]       ONLY com
          Void            ONLY void
  FUN main : com[void]  -- top level command
\end{prog}

\subsubsection{Rabbits.impl}
\begin{prog}
IMPLEMENTATION Rabbits

IMPORT  Denotation      ONLY  ++
        Nat             ONLY nat ! 0 1 2 - + > = 
        NatConv         ONLY `
        String          ONLY string 
        StringConv      ONLY `
        Com             ONLY ans:SORT
        ComCompose      COMPLETELY
        Stream          ONLY input stdIn readLine 
                             output stdOut writeLine
                             write:output**denotation->com[void]

-- FUN main : com[void] -- already declared in signature part
DEF main == 
    write(stdOut, 
 "For which generation do you want to know the number of rabbits? ") &
    (readLine(stdIn)    & (\LAMBDA in.
     processInput(in`)
    ))

FUN processInput: denotation -> com[void]
DEF processInput(ans) == 
          LET generation == !(ans)
              bunnys     == rabbits(generation)
              result     == "In the " 
                            ++ (generation`)
                            ++ ". generation there are " 
                            ++ (bunnys`) 
                            ++ " couples of rabbits." 
          IN writeLine(stdOut, result)
-- --------------------------------------------------------------

FUN rabbits : nat -> nat
DEF rabbits(generation) ==
        IF generation = 0 THEN 1
        IF generation = 1 THEN 1
        IF generation > 1 THEN rabbits(generation - 1) 
                               + rabbits(generation - 2)
        FI
\end{prog}

\section{An Interpreter for Expressions}
\label{app:expressions}

This interpreter for simple mathematical expressions was developed as
an exercise for students in the first semester.
 After two months of programming experience they had to
implement the evaluation and formatting functions, while the parser
and the I/O environment were implemented by the teaching staff.

The goal was to format a simple arithmetic expression in pre- and
infix-order and to compute the value of the expression. 

The whole program is divided into three structures:
\begin{itemize}
\item \pro{Expr}: This structure does the formatting and evaluation of
  expressions, i.e.~this was the student's part.
\item \pro{Parser}: The parser transforms the textual input into an object
  of type expression, which can be processed by the functions of
  the structure \pro{Expr}. 
  Because the implementation part is rather lengthy and not so
  interesting, we omit it in this paper.
\item \pro{ExprIO}: This structure controls the input and output and
  calls the routines of the parser, the formatters and  the evaluator
 in an the required sequence.
\end{itemize}


\subsection{ExprIO.sign}
\begin{prog}
SIGNATURE ExprIO
IMPORT  Com[void]       ONLY com
        Void            ONLY void
FUN interpreter : com[void]     -- top level command
\end{prog}

\subsection{ExprIO.impl}
\begin{prog}
IMPLEMENTATION ExprIO

IMPORT  String          ONLY string  ! ++ slice   = |=  % # empty
        Char            ONLY char = |= space? newline

        Com             ONLY ans:SORT okay? data fail?
        ComCompose      COMPLETELY
        Stream          ONLY input stdIn readLine 
                             output stdOut write writeLine 
        Nat             ONLY nat + - 0 1 2 3 4 5 6 7 8 9 10 * = |=  >
        Expr            COMPLETELY
        Parser          COMPLETELY

DEF interpreter == 
        writeLine(stdOut, 
   "Bitte einen mathematischen Ausdruck eingeben, z.B: 4+(3^2)!-2") & 
       (readLine(stdIn) & (\LAMBDA in. 
        processInput(in) )


FUN processInput: string -> com[void]
DEF processInput(ans) == 
   IF ans = empty THEN writeLine(stdOut, "Ende des Programms"!)
   IF ans |= empty THEN
      LET res == parse(ans)
      IN
      IF res error? THEN
        writeLine(stdOut, msg(res)) & 
        (writeLine(stdOut, 
               "Bitte noch einmal versuchen (leere Eingabe -> Ende):")&
        (readLine(stdIn) &
         processInput))
      IF res ok? THEN
        doWork(exprOf(res)) &
        (writeLine(stdOut, 
                 "Naechster Ausdruck (leere Eingabe -> Ende):")&
        (readLine(stdIn)&   (\LAMBDA in.
         processInput))
      FI
   FI



FUN doWork : expr -> com[void]
DEF doWork(e) ==
  writeLine(stdOut, "Der Ausdruck in Prefix-Notation ist:")&
  (write(stdOut, ">>>")&
  (write(stdOut, prefix(e))&
  (writeLine(stdOut, "<<<")&
  (writeLine(stdOut, "Der Ausdruck in Infix-Notation ist:")&
  (write(stdOut, ">>>")&
  (write(stdOut, infix(e))&
  (writeLine(stdOut, "<<<"!)&
  (write(stdOut, "Der Wert des Ausdrucks ist: "!)&
  (writeLine(stdOut, eval(e)))))))))))
\end{prog}

\subsection{Parser.sign}
\begin{prog}
SIGNATURE Parser

IMPORT  Expr    ONLY expr
        String  ONLY string

TYPE parseRes == ok(exprOf : expr)
                 error(msg : denotation)
 
FUN parse : string -> parseRes
\end{prog}

\subsection{Expr.sign}
\begin{prog}
-- -----------------------------------------------------------------
--      Textual Representation and Evaluation of Expressions 
--                              
-- -----------------------------------------------------------------
SIGNATURE Expr

IMPORT Nat ONLY nat

TYPE  expr ==   number (val: nat)
                dyadic (left: expr, dyop: dyadicOp, right:expr)
                monadic(monop : monadicOp, exprof :expr)

TYPE dyadicOp == addOp multOp minusOp divOp powerOp

TYPE monadicOp == facOp

-- -----------------------------------------------------------------

FUN prefix : expr -> denotation
FUN infix : expr -> denotation
FUN eval : expr -> denotation
\end{prog}

\subsection{Expr.impl}
\begin{prog}
-- ------------------------------------------------------------------
 
IMPLEMENTATION Expr

IMPORT Nat ONLY = > 0 1 + - * / 
IMPORT Denotation  ONLY ++ 
IMPORT NatConv ONLY `

DATA  expr ==   number (val: nat)
                dyadic (left: expr, dyop: dyadicOp, right:expr)
                monadic(monop : monadicOp, exprof :expr)

DATA dyadicOp == addOp multOp minusOp divOp powerOp

DATA monadicOp == facOp

-- --------------------------------------------------------------------

DEF prefix(number(e)) == e`
DEF prefix(monadic(op,e)) ==  monop2textPrefix(op) ++
                                "(" ++ prefix(e) ++ ")"
DEF prefix(dyadic(l,op,r)) == dyop2textPrefix(op) ++
                              "(" ++ prefix(l) ++ ", " ++ 
                              prefix(r) ++ ")"

DEF infix(number(v)) == v`
DEF infix(monadic(op,e)) == "(" ++ infix(e) ++ ")" 
                               ++ monop2textInfix(op)
DEF infix(dyadic(le,ope,re)) == 
                LET l ==  "(" ++ infix(le) ++ ")"
                    r ==  "(" ++ infix(re) ++ ")"
                    op == dyop2textInfix(ope)
                IN l ++ op ++ r

-- Auxiliary functions for textual representation

FUN monop2textPrefix : monadicOp -> denotation
DEF monop2textPrefix(op) ==
        IF op facOp? THEN "fac"
        FI

FUN dyop2textPrefix : dyadicOp -> denotation
DEF dyop2textPrefix(op) ==
        IF op addOp? THEN "add"
        IF op minusOp? THEN "minus"
        IF op multOp? THEN "mult"
        IF op divOp? THEN "div"
        IF op powerOp? THEN "pow"
        FI


FUN monop2textInfix : monadicOp -> denotation
DEF monop2textInfix(op) ==
        IF op facOp? THEN "!"
        FI


FUN dyop2textInfix : dyadicOp -> denotation
DEF dyop2textInfix(op) ==
        IF op addOp? THEN "+"
        IF op minusOp? THEN "-"
        IF op multOp? THEN "*"
        IF op divOp? THEN "/"
        IF op powerOp? THEN "^"
        FI

-- ----------------------------------------------------------------
-- evaluation of expressions

DATA result == ok(val:nat)
                error(msg:denotation)

FUN ok : nat -> result
    error : denotation -> result

FUN val : result -> nat
    msg : result -> denotation

FUN ok? error? : result -> bool



DEF eval(e) ==
        LET res == eval1(e)
        IN 
        IF res ok? THEN val(res)`
        IF res error? THEN  msg(res)
        FI

-- ------------------------------------------------------------------
-- Bei der Fehlermeldung werden alle gefundenen Fehler gemeldet

FUN eval1 : expr -> result
DEF eval1(e) ==
        IF e number? THEN ok(val(e))
        IF e monadic? THEN evalMonadic(monop(e), eval1(exprof(e)))
        IF e dyadic? THEN evalDyadic(dyop(e), 
                                     eval1(left(e)), 
                                     eval1(right(e)))
        FI

FUN evalMonadic : monadicOp ** result -> result
DEF evalMonadic (op, arg AS error(msg)) == arg
DEF evalMonadic (op, arg AS ok(val))    == ok( doMonop(op)(val))

FUN doMonop : monadicOp -> nat -> nat
DEF doMonop(op) == 
        IF op facOp? THEN fac
        FI


FUN evalDyadic : dyadicOp ** result ** result -> result
DEF evalDyadic (op, arg1, arg2) ==
        IF (arg1 error?) and (arg2 error?) 
                   THEN error(msg(arg1) ++ msg(arg2))
        IF (arg1 error?) and (arg2 ok?) THEN arg1
        IF (arg1 ok?) and (arg2 error?) THEN arg2
        IF (arg1 ok?) and (arg2 ok?) THEN 
          IF (op divOp? ) and (val(arg2)=0) 
                   THEN error("Division durch Null ")
          IF (op minusOp? ) and (val(arg2)> val(arg1))
                          THEN error("Minuend kleiner Subtrahend ")
          IF (op powerOp?) and ((val(arg1)=0) and (val(arg2)=0)) 
                          THEN error("Basis und Exponent gleich Null")
           ELSE ok(doDyop(op)(val(arg1), val(arg2)))
          FI
        FI

FUN doDyop : dyadicOp -> nat ** nat -> nat
DEF doDyop(op) == 
        IF op addOp? THEN +
        IF op minusOp? THEN -
        IF op multOp? THEN *
        IF op divOp? THEN /
        IF op powerOp? THEN pow
        FI 

-- ------------------------------------------------------------------
FUN pow : nat ** nat -> nat
DEF pow(x, y) == IF y = 0 THEN 1
                 IF y = 1 THEN x
                 IF y > 1 THEN x * pow(x, y-1)
                 FI
-- --------------------------------------------------------------------
FUN fac : nat -> nat
DEF fac(x)==    IF (x =0) or (x=1) THEN 1
                IF x > 1 THEN x * fac(x-1) 
                FI
\end{prog}


\section{An Arbitrary Directed Graph}
\label{app:graph}

Sorry, the directed Graph is not available yet.

% Local Variables: 
% mode: latex
% TeX-master: "tutorial"
% End: 

% LAST EDIT: Tue May  3 14:00:21 1994 by Juergen Exner (hektor!jue) 
% LAST EDIT: Tue Feb 15 18:19:54 1994 by Juergen Exner (hektor!jue) 
% LAST EDIT: Mon Nov 15 11:29:29 1993 by Juergen Exner (hektor!jue) 
% LAST EDIT: Tue Jan 12 16:36:57 1993 by Juergen Exner (hektor!jue) 
\chapter{Common Errors and What To Do}
\label{chap:errors}

This appendix will describe some common programming errors and how
to avoid them. 
The collection is incomplete as  we are still looking for new
errors.
If you have found a typical programming error and you think it should
be included here, we would encourage you to send this error (together
with the solution) to the author (email: jue@cs.tu-berlin.de).

\begin{errorlist}
\error{Expected was FI ... instead of OTHERWISE}{Behind an \pro{OTHERWISE}
  there must be another guard; no immediate object declaration (\pro{LET
  ...}) is allowed.}

\error{Expected was IN ... instead of IF}{\pro{IF}-expressions within
  other expressions have to be enclosed in parantheses.}

\error{Missing Operand}{Missing delimiter (e.g.~space) between
  graphical symbols; Example: \pro{a::b::c::<>} must be written as
  \pro{a::b::c:: <>}}

\error{Undefined identification}{Typically wrong typing: missing
  import of the operation, import with the wrong functionality (in
  cases of overloaded functions), wrong functionality in declaration.}

\error{Undefined identification}{Function application in prefix
  notation without paranthesss, compiler will assume postfix notation:
  \pro{\% c} must be written as \pro{\%(c)}}
  
\end{errorlist}

% Local Variables: 
% mode: latex
% TeX-master: "tutorial"
% End: 

% LAST EDIT: Tue May  3 13:15:39 1994 by Juergen Exner (hektor!jue) 
% LAST EDIT: Tue Feb 15 16:12:19 1994 by Juergen Exner (hektor!jue) 
% LAST EDIT: Sun Feb 13 17:03:32 1994 by Juergen Exner (hektor!jue) 
% LAST EDIT: Thu Nov 11 11:03:17 1993 by Juergen Exner (hektor!jue) 

%
%\chapter{Bibliography}
%\label{sec:biblio}
%The following papers offers you additional information about the
%language \opal\ itselfs and surrounding topics.

\begin{thebibliography}{XXXXX}
\bibitem[Pe91]{Pe} P. Pepper: {\em The Programming Language \opal\/};
Technical Report No.\ 10-91, FB Informatik, TU Berlin; third corrected
edition Dec.~92
\bibitem[SchGr92]{SchGr}
W. Schulte and W. Grieskamp: {\em Generating {E}fficient {P}ortable
  {C}ode for a {S}trict {A}pplicative {L}anguage\/} in: Proceedings of 
Phoenix Seminar and Workshop on Declarative Programming; 
Springer 1992
\bibitem[Di94]{Di}K. Didrich: {\em Bibliotheca Opalica\/}; TU Berlin,
1994

\bibitem[Ma93]{Ma} Ch. Maeder: {\em A User's Guide to the \opal\  Compilation
System\/}; TU Berlin, 1993

\bibitem[Le94]{Le}A. Lennartz: {\em Entwurf und Implementierung eines
  Pseudo-Interpreters f\"ur die Programmiersprache \opal\/}; Thesis,
TU Berlin, 1994

\bibitem[Dz93]{browser}R. Dziallas: {\em Entwicklung von
  Anzeigefunktionen f\"ur analysierte \opal Quellen\/}; Thesis, TU
Berlin, 1993

\bibitem[GrS\"u94]{handcoding}W. Grieskamp and M. S\"udholt: {\em
  Handcoder's Guide to OCS Version 2\/}; TU Berlin, 1994

\end{thebibliography}
% Local Variables: 
% mode: latex
% TeX-master: "tutorial"
% End: 

\end{document}

